This section contains info to setup the target hardware, and install Legato on the host P\+C including making and deploying your first target app.





If you\textquotesingle{}re new to Legato, checkout our \href{http://www.sierrawireless.com/productsandservices/airprime_wireless_modules/smart_modules/legato/}{\tt product info}.





There are two things to do before you can develop a Legato app\+:

\hyperlink{getstartedTargetMain}{Setup Target Device} ~\newline
 \hyperlink{getstartedSetupHost}{Setup Dev Host P\+C}





Each release, we publish\+:

\hyperlink{legatoRelNotesBeta}{Release Notes} ~\newline
 \hyperlink{legatoUpgradeBeta}{Upgrade Notes}





All Get Started info is also available in a \href{http://source.sierrawireless.com/Resources/Resources/Legato/GettingStarted.aspx}{\tt P\+D\+F doc}.





Copyright (C) Sierra Wireless Inc. Use of this work is subject to license. \hypertarget{getstartedTargetMain}{}\section{Setup Target Device}\label{getstartedTargetMain}
Legato comes ready to develop on the \hyperlink{getstartedMangOH}{Mang\+O\+H} dev kit.

There is also info available on how to assemble and setup the \href{http://source.sierrawireless.com/resources/legato/wp7xdevkit/}{\tt W\+P7} dev kit.





These \hyperlink{getstartedAdvanced}{Advanced} topics about rebuilding your target are also available.





Copyright (C) Sierra Wireless Inc. Use of this work is subject to license. \hypertarget{getstartedMangOH}{}\subsection{Mang\+O\+H}\label{getstartedMangOH}
It\textquotesingle{}s easy to get ready to develop on the Mang\+O\+H board with a W\+P85 module\+:

\hyperlink{getstartedMangOHAssemble}{Assemble} ~\newline
 \hyperlink{getstartedMangOHInstallSoftware}{Install Software}

Complete information about Mang\+O\+H is available from \href{https://mangoh.io/}{\tt mangoh.\+io} and the Sierra Wireless Source \href{http://source.sierrawireless.com/mangoh/}{\tt mang\+O\+H} page.





Copyright (C) Sierra Wireless Inc. Use of this work is subject to license. \hypertarget{getstartedMangOHAssemble}{}\subsubsection{Assemble}\label{getstartedMangOHAssemble}
This topic provides high-\/level info only.

Download the \href{http://source.sierrawireless.com/mangoh/}{\tt Mang\+O\+H Get Started Guide} if you need detailed info about the board or available add-\/ons (e.\+g., antenna, Io\+T module).



 There are only a few steps to assemble the Mang\+O\+H board with W\+P85 for Legato.

Connect\+:
\begin{DoxyItemize}
\item power supply
\item U\+A\+R\+T cable
\item either micro-\/\+U\+S\+B or Ethernet cable
\item W\+P85 module.
\end{DoxyItemize}

After you\textquotesingle{}ve assembled the board, you can start \hyperlink{getstartedTargetHostCom}{Target/\+Host Communications}.



Here\textquotesingle{}s a side view of the power and micro-\/\+U\+S\+B connections\+:







Copyright (C) Sierra Wireless Inc. Use of this work is subject to license. \hypertarget{getstartedTargetHostCom}{}\subsection{Target/\+Host Communications}\label{getstartedTargetHostCom}
We use a serial communication program like Minicom to access the target through the U\+A\+R\+T port to for console communications.

The micro-\/\+U\+S\+B or Ethernet ports are used for development. See \hyperlink{getstartedMangOHAssemble}{Assemble} mang\+O\+H for Legato.

If you\textquotesingle{}re using Minicom, you access the config and initialize menus through the {\ttfamily C\+T\+R\+L+\+A} Z command.

You\textquotesingle{}ll need to \hyperlink{getstartedConfigIP}{Configure I\+P Address}.

Always remember to run the target as {\bfseries root} user, so no password will be required.

Once the host and target are communicating, you can \hyperlink{getstartedMangOHInstallSoftware}{Install Software}.

\begin{DoxyNote}{Note}
Use the \hyperlink{toolsTarget_setNet}{set\+Net} tool to set your M\+A\+C address so you don\textquotesingle{}t have to go through {\ttfamily ifconfig} every time you reboot the target.
\end{DoxyNote}




Copyright (C) Sierra Wireless Inc. Use of this work is subject to license. \hypertarget{getstartedConfigIP}{}\subsubsection{Configure I\+P Address}\label{getstartedConfigIP}
Before you can start working with your target, you need to setup host-\/target communications to use an I\+P address.

There are a few ways to configure a target I\+P address\+:

\hyperlink{getstartedConfigIPusb}{Micro-\/\+U\+S\+B} ~\newline
 \hyperlink{getstartedConfigIPeth}{Ethernet} 



Copyright (C) Sierra Wireless Inc. Use of this work is subject to license. \hypertarget{getstartedConfigIPusb}{}\paragraph{Micro-\/\+U\+S\+B}\label{getstartedConfigIPusb}
The W\+P85 module on a mang\+O\+H board is pre-\/configured to use the micro-\/\+U\+S\+B cable.

Connect serial, micro-\/\+U\+S\+B, and power cables.\hypertarget{getstarted_config_i_pusb_getstartedConfigIPusb_target}{}\subparagraph{Target}\label{getstarted_config_i_pusb_getstartedConfigIPusb_target}
In a terminal, run\+: 
\begin{DoxyCode}
minicom /dev/ttyUSB0 
\end{DoxyCode}


Press enter when Minicom starts.

Sign on as {\bfseries root} {\bfseries user} (no password required).

Press enter when prompted for a password.

Run {\ttfamily ifconfig}. It should look something like this\+:

\begin{DoxyVerb}root@swi-mdm9x15:~# ifconfig
lo        Link encap:Local Loopback
          inet addr:127.0.0.1  Mask:255.0.0.0
          inet6 addr: ::1/128 Scope:Host
          UP LOOPBACK RUNNING  MTU:16436  Metric:1
          RX packets:0 errors:0 dropped:0 overruns:0 frame:0
          TX packets:0 errors:0 dropped:0 overruns:0 carrier:0
          collisions:0 txqueuelen:0
          RX bytes:0 (0.0 B)  TX bytes:0 (0.0 B)

usb0      Link encap:Ethernet  HWaddr 32:E4:5E:6F:C4:3C
          inet addr:192.168.1.2  Bcast:192.168.1.255  Mask:255.255.255.0
          inet6 addr: fe80::30e4:5eff:fe6f:c43c/64 Scope:Link
          UP BROADCAST RUNNING MULTICAST  MTU:1500  Metric:1
          RX packets:197 errors:0 dropped:0 overruns:0 frame:0
          TX packets:57 errors:0 dropped:0 overruns:0 carrier:0
          collisions:0 txqueuelen:1000
          RX bytes:32111 (31.3 KiB)  TX bytes:6725 (6.5 KiB)
\end{DoxyVerb}


Note that usb0 has an inet addr value. If it doesn\textquotesingle{}t display an inet value, try rebooting the target and rerun {\ttfamily ifconfig} \hypertarget{getstarted_config_i_pusb_getstartedConfigIPusb_host}{}\subparagraph{Host}\label{getstarted_config_i_pusb_getstartedConfigIPusb_host}
Open a new terminal on the host.

{\ttfamily cd} to your Legato directory.

Run {\ttfamily bin/legs} to set the Legato environment.

Run\+: 
\begin{DoxyCode}
ifconfig usb0 <target ip addr> up 
\end{DoxyCode}


where the I\+P address {\bfseries subnet} matches the target.

You may have to use {\ttfamily sudo} 

\begin{DoxyVerb}local@LocalLinux:~$ sudo ifconfig usb0 192.168.1.20 up
[sudo] password for local:
local@LocalLinux:~$
\end{DoxyVerb}


Run {\ttfamily ifconfig} again, it should look something like this\+:

\begin{DoxyVerb}eth0      Link encap:Ethernet  HWaddr 28:d2:44:36:bd:f6
          UP BROADCAST MULTICAST  MTU:1500  Metric:1
          RX packets:0 errors:0 dropped:0 overruns:0 frame:0
          TX packets:0 errors:0 dropped:0 overruns:0 carrier:0
          collisions:0 txqueuelen:1000
          RX bytes:0 (0.0 B)  TX bytes:0 (0.0 B)

lo        Link encap:Local Loopback
          inet addr:127.0.0.1  Mask:255.0.0.0
          inet6 addr: ::1/128 Scope:Host
          UP LOOPBACK RUNNING  MTU:65536  Metric:1
          RX packets:6407 errors:0 dropped:0 overruns:0 frame:0
          TX packets:6407 errors:0 dropped:0 overruns:0 carrier:0
          collisions:0 txqueuelen:0
          RX bytes:618870 (618.8 KB)  TX bytes:618870 (618.8 KB)

usb0      Link encap:Ethernet  HWaddr 72:ee:fc:17:f3:61
          inet6 addr: fe80::70ee:fcff:fe17:f361/64 Scope:Link
          UP BROADCAST RUNNING MULTICAST  MTU:1500  Metric:1
          RX packets:25 errors:0 dropped:0 overruns:0 frame:0
          TX packets:210 errors:0 dropped:0 overruns:0 carrier:0
          collisions:0 txqueuelen:1000
          RX bytes:5229 (5.2 KB)  TX bytes:42991 (42.9 KB)

wlan0     Link encap:Ethernet  HWaddr 00:c2:c6:13:a2:f9
          inet addr:192.168.0.17  Bcast:192.168.0.255  Mask:255.255.255.0
          inet6 addr: fe80::2c2:c6ff:fe13:a2f9/64 Scope:Link
          UP BROADCAST RUNNING MULTICAST  MTU:1500  Metric:1
          RX packets:657272 errors:0 dropped:0 overruns:0 frame:0
          TX packets:114077 errors:0 dropped:0 overruns:0 carrier:0
          collisions:0 txqueuelen:1000
          RX bytes:244778069 (244.7 MB)  TX bytes:15624212 (15.6 MB)\end{DoxyVerb}


Ping target to host and vice versa to check everything is working.





Copyright (C) Sierra Wireless Inc. Use of this work is subject to license. \hypertarget{getstartedConfigIPeth}{}\paragraph{Ethernet}\label{getstartedConfigIPeth}
The W\+P85 module on a Mang\+O\+H board can use an Ethernet connection to a local area network to configure the I\+P addresses.

Connect the serial, Ethernet, and power cables

In a terminal, run\+:


\begin{DoxyCode}
minicom /dev/ttyUSB0 
\end{DoxyCode}


Press enter when Minicom starts.

Sign on as {\bfseries root} {\bfseries user} (no password required).

Press enter when prompted for a password.

Run {\ttfamily ifconfig} (or ip addr). It should display something like this\+:

\begin{DoxyVerb}root@swi-mdm9x15:~# ifconfig
eth0      Link encap:Ethernet  HWaddr 9E:90:7C:6D:BE:9B
          inet addr:10.1.28.211  Bcast:10.1.28.255  Mask:255.255.255.0
          inet6 addr: fe80::9c90:7cff:fe6d:be9b/64 Scope:Link
          UP BROADCAST RUNNING MULTICAST  MTU:1500  Metric:1
          RX packets:43357 errors:0 dropped:107 overruns:0 frame:0
          TX packets:3975 errors:0 dropped:0 overruns:0 carrier:0
          collisions:0 txqueuelen:1000
          RX bytes:14481736 (13.8 MiB)  TX bytes:718334 (701.4 KiB)

lo        Link encap:Local Loopback
          inet addr:127.0.0.1  Mask:255.0.0.0
          inet6 addr: ::1/128 Scope:Host
          UP LOOPBACK RUNNING  MTU:16436  Metric:1
          RX packets:65 errors:0 dropped:0 overruns:0 frame:0
          TX packets:65 errors:0 dropped:0 overruns:0 carrier:0
          collisions:0 txqueuelen:0
          RX bytes:3608 (3.5 KiB)  TX bytes:3608 (3.5 KiB)\end{DoxyVerb}


If it doesn\textquotesingle{}t display an eth0 inet address, try rebooting the target (easiest is to disconnect and reconnect power cable), and run {\ttfamily ifconfig} again.

Or you can run {\ttfamily ifdown} {\ttfamily eth0} and then {\ttfamily if} up {\ttfamily eth0}, to stop and restart the Ethernet connection.

You should be ready to ssh to your target I\+P -\/ the subnets should match.





Copyright (C) Sierra Wireless Inc. Use of this work is subject to license. \hypertarget{getstartedMangOHInstallSoftware}{}\subsubsection{Install Software}\label{getstartedMangOHInstallSoftware}
After you have your Mang\+O\+H board communicating with your host, you can install Legato software\+:

\hyperlink{getstartedDSinstall}{Developer Studio} ~\newline
 \hyperlink{getstartedCLinstallMain}{Command-\/line}





Copyright (C) Sierra Wireless Inc. Use of this work is subject to license. \hypertarget{getstartedAdvanced}{}\subsection{Advanced}\label{getstartedAdvanced}
These topics have info on how to rebuild or customize your Legato target\+: ~\newline
 \hyperlink{getstartedRestoreUpgrade}{Upgrade or Restore} ~\newline
 \hyperlink{yoctoMain}{Yocto Info}





Copyright (C) Sierra Wireless Inc. Use of this work is subject to license. \hypertarget{getstartedRestoreUpgrade}{}\subsubsection{Upgrade or Restore}\label{getstartedRestoreUpgrade}
See \href{http://www.legato.io/legato-docs/15_10/legatoUpgrade15_10.html}{\tt Upgrade Notes} for specifics on upgrading to the current Legato version.

Update or restore target\+:

\hyperlink{getstartedWindowsHost}{Windows Dev Host} ~\newline
 \hyperlink{getstartedLinuxAndroid}{Linux Android S\+D\+K} ~\newline


\hyperlink{getstartedUninstall}{Uninstall Legato}





Copyright (C) Sierra Wireless Inc. Use of this work is subject to license. \hypertarget{getstartedWindowsHost}{}\subsection{Windows Dev Host}\label{getstartedWindowsHost}
You can set up a Windows host to upgrade or recover a device.\hypertarget{getstarted_windows_host_getstartedWindowsHost_instAndroidtools}{}\subparagraph{Windows Android Tools}\label{getstarted_windows_host_getstartedWindowsHost_instAndroidtools}
The Android S\+D\+K tools can be downloaded \href{http://developer.android.com/sdk/index.html}{\tt here}.

To use the Android tools, first download the bundle, run the S\+D\+K Manager, then install a single Android platform and the Google U\+S\+B driver.

The following additions are needed in the android\+\_\+winusb.\+inf file, which is located at\+: 
\begin{DoxyCode}
%install\_dir% sdk\(\backslash\)extras\(\backslash\)google\(\backslash\)usb\_driver\(\backslash\)android\_winusb.inf.
\end{DoxyCode}


The additions are in two sections that already contain entries for Nexus devices (install\+\_\+dir\% refers to the location where you unzipped the Android S\+D\+K bundle).

\begin{DoxyVerb}In section [Google.NTx86]

;Qualcomm SURF/FFA
%SingleAdbInterface%        = USB_Install, USB\VID_05C6&PID_9025
%CompositeAdbInterface%     = USB_Install, USB\VID_05C6&PID_9025&MI_01
%SingleBootLoaderInterface% = USB_Install, USB\VID_18D1&PID_D00D

%SingleAdbInterface%        = USB_Install, USB\VID_05C6&PID_901C
%CompositeAdbInterface%     = USB_Install, USB\VID_05C6&PID_901C&MI_01
%SingleBootLoaderInterface% = USB_Install, USB\VID_18D1&PID_D00D

;Sierra
%SingleAdbInterface%        = USB_Install, USB\VID_1199&PID_68A2
%CompositeAdbInterface%     = USB_Install, USB\VID_1199&PID_68A2&MI_01
%SingleAdbInterface%        = USB_Install, USB\VID_1199&PID_68C0
%CompositeAdbInterface%     = USB_Install, USB\VID_1199&PID_68C0&MI_01
%CompositeAdbInterface%     = USB_Install, USB\VID_1199&PID_9041&MI_01
%CompositeAdbInterface%     = USB_Install, USB\VID_1199&PID_9051&MI_01
%CompositeAdbInterface%     = USB_Install, USB\VID_1199&PID_9053&MI_01
%CompositeAdbInterface%     = USB_Install, USB\VID_1199&PID_9054&MI_01
%CompositeAdbInterface%     = USB_Install, USB\VID_1199&PID_9056&MI_01
%CompositeAdbInterface%     = USB_Install, USB\VID_1199&PID_9061&MI_01

In section [Google.NTamd64]

;Qualcomm SURF/FFA
%SingleAdbInterface%        = USB_Install, USB\VID_05C6&PID_9025
%CompositeAdbInterface%     = USB_Install, USB\VID_05C6&PID_9025&MI_01
%SingleBootLoaderInterface% = USB_Install, USB\VID_18D1&PID_D00D

%SingleAdbInterface%        = USB_Install, USB\VID_05C6&PID_901C
%CompositeAdbInterface%     = USB_Install, USB\VID_05C6&PID_901C&MI_01
%SingleBootLoaderInterface% = USB_Install, USB\VID_18D1&PID_D00D

;Sierra
%SingleAdbInterface%        = USB_Install, USB\VID_1199&PID_68A2
%CompositeAdbInterface%     = USB_Install, USB\VID_1199&PID_68A2&MI_01
%SingleAdbInterface%        = USB_Install, USB\VID_1199&PID_68C0
%CompositeAdbInterface%     = USB_Install, USB\VID_1199&PID_68C0&MI_01
%CompositeAdbInterface%     = USB_Install, USB\VID_1199&PID_68C0&REV_0006&MI_01
%CompositeAdbInterface%     = USB_Install, USB\VID_1199&PID_9041&MI_01
%CompositeAdbInterface%     = USB_Install, USB\VID_1199&PID_9051&MI_01
%CompositeAdbInterface%     = USB_Install, USB\VID_1199&PID_9053&MI_01
%CompositeAdbInterface%     = USB_Install, USB\VID_1199&PID_9054&MI_01
%CompositeAdbInterface%     = USB_Install, USB\VID_1199&PID_9056&MI_01
%CompositeAdbInterface%     = USB_Install, USB\VID_1199&PID_9061&MI_01

In addition, ensure there are matching entries under the [Strings] section:
[Strings]
SingleAdbInterface          = "Android ADB Interface"
CompositeAdbInterface       = "Android Composite ADB Interface"
SingleBootLoaderInterface   = "Android Bootloader Interface"

Edit the ini file from android:
•    A device VID must be added to be recognized by ADB.
•    Create a %USERPROFILE%\.android directory, if it does not exist.
•    Navigate to the %USERPROFILE%\.android directory.
In the %USERPROFILE%\.android directory, create or edit the file adb_usb.ini. If the file already exists, it will contain a DO NOT EDIT message.
Disregard this message and edit the file anyway. To edit, add a line containing 0x05C6 and a line containing 0x1199 to the end of the file.
Note: Don’t run android update adb or it will reset the contents of this file and overwrite the line just added.
After editing, the adb_usb.ini file should look like this:
# ANDROID 3RD PARTY VENDOR ID LIST -- DO NOT EDIT.
# USE 'android update adb' TO GENERATE.
# 1 USB VENDOR ID PER LINE
0x05C6
0x1199
\end{DoxyVerb}






Copyright (C) Sierra Wireless Inc. Use of this work is subject to license. \hypertarget{getstartedLinuxAndroid}{}\subsection{Linux Android S\+D\+K}\label{getstartedLinuxAndroid}
You can setup a Linux Android host to upgrade or recover a device.

Only a 64-\/bit Ubuntu 12.\+04 distribution is currently being tested. Other distributions are unknown at this time, but can be tried. If you intend to rebuild the Yocto images, don\textquotesingle{}t use Ubuntu 13.\+10\+: it does not work.\hypertarget{getstarted_linux_android_instLinuxAndroid}{}\subparagraph{Setup fastboot}\label{getstarted_linux_android_instLinuxAndroid}
Android tools use fastboot to replace the kernel and root filesystem on devices.

Download Android S\+D\+K tools from \href{http://developer.android.com/sdk/index.html}{\tt here}.

Extract the file and add the directory containing the fastboot command to your path.\hypertarget{getstarted_linux_android_createDevRules}{}\subparagraph{Create Udev Rules}\label{getstarted_linux_android_createDevRules}
By default, fastboot on Ubuntu can\textquotesingle{}t communicate with a U\+S\+B-\/connected device because of the default permissions. Create udev rules to overcome this. In the file /etc/udev/rules.d/51-\/android.\+rules insert the following two lines and save\+: 
\begin{DoxyCode}
SUBSYSTEM==\textcolor{stringliteral}{"usb"}, ATTRS\{idVendor\}==\textcolor{stringliteral}{"1199"}, MODE=\textcolor{stringliteral}{"0666"} 
\end{DoxyCode}
 
\begin{DoxyCode}
SUBSYSTEM==\textcolor{stringliteral}{"usb"}, ATTRS\{idVendor\}==\textcolor{stringliteral}{"18d1"}, MODE=\textcolor{stringliteral}{"0666"} 
\end{DoxyCode}
 Then reboot your P\+C.





Copyright (C) Sierra Wireless Inc. Use of this work is subject to license. \hypertarget{getstartedUninstall}{}\subsection{Uninstall Legato}\label{getstartedUninstall}
To uninstall Legato from your dev host, delete the directory where you extracted Legato.

To uninstall the toolchain, delete the toolchain directory {\ttfamily /opt/swi}.





Copyright (C) Sierra Wireless Inc. Use of this work is subject to license. \hypertarget{getstartedSetupHost}{}\section{Setup Dev Host P\+C}\label{getstartedSetupHost}
Legato can be installed on a host P\+C using Sierra Wireless\textquotesingle{} Developer Studio or by working with the command line\+:

\hyperlink{getstartedDSinstall}{Developer Studio} --- install and develop Legato with Developer Studio.

\hyperlink{getstartedCLinstallMain}{Command-\/line} --- install and develop Legato source files from the command-\/line.





Copyright (C) Sierra Wireless Inc. Use of this work is subject to license. \hypertarget{getstartedDSinstall}{}\subsection{Developer Studio}\label{getstartedDSinstall}
Please review the \hyperlink{getstartedDSrequirements}{Requirements}.

You can install Legato with Developer Studio on two platforms\+:

\hyperlink{getstartedDSwindows}{Windows Host} ~\newline
 \hyperlink{getstartedDSlinux}{Linux Host}





Copyright (C) Sierra Wireless Inc. Use of this work is subject to license. \hypertarget{getstartedDSrequirements}{}\subsubsection{Requirements}\label{getstartedDSrequirements}
These are the requirements for a Legato Developer Studio host P\+C.\hypertarget{getstarted_d_srequirements_getstartedDSrequirements_hardware}{}\paragraph{Hardware}\label{getstarted_d_srequirements_getstartedDSrequirements_hardware}
Please make sure the following hardware requirements are fulfilled by the system where Developer Studio aims to be installed.

Minimum\+: ~\newline
 C\+P\+U\+: Dual Core @ 2,6\+G\+Hz ~\newline
 R\+A\+M\+: 4\+G\+B ~\newline
 This configuration has been identified as the minimum to run Developer Studio in normal conditions, but some latencies should be experienced when perspectives are loaded for the first time, or when heavy traffic load is received from the target.

Recommended\+: ~\newline
 C\+P\+U\+: Dual Hyperthreaded or Quad Core @ 2,8\+G\+Hz ~\newline
 R\+A\+M\+: 6\+G\+B ~\newline


Ideal\+: ~\newline
 C\+P\+U\+: Quad Hyperthreaded Core (8x) @ 3,4\+G\+Hz ~\newline
 R\+A\+M\+: 8\+G\+B ~\newline


This configuration has been identified as the recommended one to run Developer Studio in comfortable conditions, with high performances \& reduced latencies.

{\bfseries System display}

Minimum\+: ~\newline
 19" display (e.\+g. 1280x1024px) ~\newline


Recommended\+: ~\newline
 24" display (e.\+g. 1920x1080px) ~\newline
\hypertarget{getstarted_d_srequirements_getstartedDSrequirements_software}{}\paragraph{Software}\label{getstarted_d_srequirements_getstartedDSrequirements_software}
Developer Studio requires Java 7 to be installed on the host system in order to function properly. Please pay attention to the use the Java version (32 bits / 64 bits) consistent with the one you chose for Developer Studio.

On Linux hosts\+: Java is automatically installed as dependency of Developer Studio. On Windows hosts\+:

Legato Application Development Kit supports 64-\/bit hosts only; 64-\/bit Java version is required.





Copyright (C) Sierra Wireless Inc. Use of this work is subject to license. \hypertarget{getstartedDSwindows}{}\subsubsection{Windows Host}\label{getstartedDSwindows}
Legato can be installed on a Windows host to use the Developer Studio I\+D\+E.\hypertarget{getstarted_d_swindows_getstartedDSwindows_adkInstaller}{}\paragraph{Legato A\+D\+K Installer}\label{getstarted_d_swindows_getstartedDSwindows_adkInstaller}
Download the Legato Application Development Kit (A\+D\+K) Installer for Windows from the Sierra Wireless Source \href{http://source.sierrawireless.com/resources/legato/downloads/}{\tt Legato Downloads} page.

Run the {\bfseries Legato A\+D\+K Installer} and follow the steps to install the Legato A\+D\+K\+:

These items will be installed\+:
\begin{DoxyItemize}
\item Developer Studio
\item drivers
\item latest Legato version
\end{DoxyItemize}\hypertarget{getstarted_d_swindows_getstartedDSwindows_cheatSheet}{}\paragraph{Cheat Sheet}\label{getstarted_d_swindows_getstartedDSwindows_cheatSheet}
After installation completes, launch Developer Studio, and follow the provided interactive cheat sheet to run your first Legato Hello World application on your target.





Copyright (C) Sierra Wireless Inc. Use of this work is subject to license. \hypertarget{getstartedDSlinux}{}\subsubsection{Linux Host}\label{getstartedDSlinux}
There are two ways to install Legato with Developer Studio on Linux host\+:

\hyperlink{getstartedDSlinuxIDE}{Developer Studio I\+D\+E} ~\newline
 \hyperlink{getstartedDSspm}{Software Package Manager}





Copyright (C) Sierra Wireless Inc. Use of this work is subject to license. \hypertarget{getstartedDSlinuxIDE}{}\subsection{Developer Studio I\+D\+E}\label{getstartedDSlinuxIDE}
These are the steps required to setup Legato for Developer Studio I\+D\+E on a Linux host.\hypertarget{getstarted_d_slinux_i_d_e_getstartedDSlinuxIDE_debian}{}\subparagraph{Setup apt sources}\label{getstarted_d_slinux_i_d_e_getstartedDSlinuxIDE_debian}
It\textquotesingle{}s recommended to use Debian packets (Debian/\+Ubuntu distributions or any other compatible with A\+P\+T repositories).

Run the follow to reference the Developer Studio A\+P\+T repository\+: 
\begin{DoxyCode}
wget http:\textcolor{comment}{//updatesite.sierrawireless.com/developerStudio3/debian/devstudio.list -O - | sudo tee
       /etc/apt/sources.list.d/devstudio.list}
wget http:\textcolor{comment}{//updatesite.sierrawireless.com/developerStudio3/debian/devstudio.key -O - | sudo apt-key add -}
sudo apt-\textcolor{keyword}{get} update
\end{DoxyCode}
\hypertarget{getstarted_d_slinux_i_d_e_getstartedDSlinuxIDE_packets}{}\subparagraph{Install Packages}\label{getstarted_d_slinux_i_d_e_getstartedDSlinuxIDE_packets}
Install Developer Studio for Legato by running\+:

\begin{DoxyVerb}sudo apt-get install devstudio-legato
\end{DoxyVerb}
\hypertarget{getstarted_d_slinux_i_d_e_getstartedDSlinuxIDE_latestVersion}{}\subparagraph{Get Latest Version}\label{getstarted_d_slinux_i_d_e_getstartedDSlinuxIDE_latestVersion}
To get the latest Legato version, open Developer Studio, go to the Package Manager perspective, and install it from the official Sierra Wireless repository.\hypertarget{getstarted_d_slinux_i_d_e_getstartedDSlinuxIDE_choice}{}\subparagraph{Native or Docker Version}\label{getstarted_d_slinux_i_d_e_getstartedDSlinuxIDE_choice}
You have the choice between two Legato versions\+:


\begin{DoxyItemize}
\item {\bfseries Native\+:} the native G\+C\+C cross compiler Toolchain will be installed to build Legato applications.
\item {\bfseries Docker\+:} build will be addressed through Docker (as it is done on Windows).
\end{DoxyItemize}

Follow the provided interactive cheat sheet in Developer Studio to run your first Legato Hello World application on your device.





Copyright (C) Sierra Wireless Inc. Use of this work is subject to license. \hypertarget{getstartedDSspm}{}\subsection{Software Package Manager}\label{getstartedDSspm}
You can install Legato using the Developer Studio Software Package Manager (S\+P\+M).\hypertarget{getstarted_d_sspm_getstartedDSspm_installSPMapt}{}\subparagraph{Setup apt sources}\label{getstarted_d_sspm_getstartedDSspm_installSPMapt}
Run the follow to reference the Developer Studio A\+P\+T repository\+:


\begin{DoxyCode}
wget http:\textcolor{comment}{//updatesite.sierrawireless.com/developerStudio3/debian/devstudio.list -O - | sudo tee
       /etc/apt/sources.list.d/devstudio.list}
wget http:\textcolor{comment}{//updatesite.sierrawireless.com/developerStudio3/debian/devstudio.key -O - | sudo apt-key add -}
sudo apt-\textcolor{keyword}{get} update
\end{DoxyCode}
\hypertarget{getstarted_d_sspm_getstartedDSspm_installSPM}{}\subparagraph{Install Packages}\label{getstarted_d_sspm_getstartedDSspm_installSPM}
Install Software Package Manager for Legato by running\+: \begin{DoxyVerb}sudo apt-get install devstudio-legato-spm
\end{DoxyVerb}
\hypertarget{getstarted_d_sspm_getstartedDSspm_useSPM}{}\subparagraph{Use S\+P\+M}\label{getstarted_d_sspm_getstartedDSspm_useSPM}
Here\textquotesingle{}s how to use the S\+P\+M.\hypertarget{getstarted_d_sspm_getstartedComLine_useSPMoptions}{}\subparagraph{Run Options}\label{getstarted_d_sspm_getstartedComLine_useSPMoptions}
{\bfseries {\ttfamily legato-\/spm \mbox{[}O\+P\+T\+I\+O\+N\mbox{]}}}

Options\+:

\begin{DoxyVerb}tree \end{DoxyVerb}
 \begin{quote}
Displays repository tree of available releases and modules. \end{quote}


\begin{DoxyVerb}-r <release> -m <module> browse\end{DoxyVerb}
 \begin{quote}
Displays available packages that can be installed from the repository for a specified release and module. \end{quote}


\begin{DoxyVerb}-r <release> -m <module> install\end{DoxyVerb}
 \begin{quote}
Installs packages from the repository for a specified release and module. \end{quote}


\begin{DoxyVerb}list\end{DoxyVerb}
 \begin{quote}
Lists currently installed Legato packages. \end{quote}
\hypertarget{getstarted_d_sspm_getstartedDSspm_useSPMpkgs}{}\subparagraph{Packages Installed}\label{getstarted_d_sspm_getstartedDSspm_useSPMpkgs}
Packages are installed by default to your {\ttfamily $\sim$/}.devstudio/packages directory\+: \begin{DoxyVerb}legato.framework.* - Prebuilt Legato framework packages
legato.toolchain.* - GCC cross compiler Toolchains
legato.devimg.* - Device Images allowing to upgrade Legato devices system software
\end{DoxyVerb}
\hypertarget{getstarted_d_sspm_getstartedDSspm_useSPMconfig}{}\subparagraph{Config Legato}\label{getstarted_d_sspm_getstartedDSspm_useSPMconfig}
Use this script to confirgure Legato\+:

\begin{DoxyVerb}source ~/.devstudio/packages/legato.sdk.VERSION/resources/configlegatoenv
\end{DoxyVerb}


Your environment is now ready to \hyperlink{getstartedSampleApps}{Use Sample Apps}. ~\newline






Copyright (C) Sierra Wireless Inc. Use of this work is subject to license. \hypertarget{getstartedCLinstallMain}{}\subsection{Command-\/line}\label{getstartedCLinstallMain}
You can install Legato source packages and build it from scratch through the command line.

Please review the \hyperlink{getstartedCLrequirements}{Requirements}.

Downloads and other info available from the Sierra Wireless Source \href{http://source.sierrawireless.com/legato/}{\tt Legato} page.





\hyperlink{getstartedPrepDevHost}{Prep Dev Host} ~\newline
 \hyperlink{getstartedDwnLd}{Download Files} ~\newline
 \hyperlink{getstartedCLinstall}{Install Source Files} ~\newline
 \hyperlink{getstartedCLbuild}{Build Legato} ~\newline
 \hyperlink{getstartedTargetSW}{Setup Target Software} ~\newline
 \hyperlink{getstartedSampleApps}{Use Sample Apps}

See \hyperlink{yoctoMain}{Yocto Info} if you\textquotesingle{}re a Yocto whiz, and you want to customize or rebuild the target.





Copyright (C) Sierra Wireless Inc. Use of this work is subject to license. \hypertarget{getstartedCLrequirements}{}\subsubsection{Requirements}\label{getstartedCLrequirements}
The following are the recommended system requirements for a Linux host P\+C running from the command-\/line\+:\hypertarget{getstarted_c_lrequirements_getstartedCLrequirements_system}{}\paragraph{System}\label{getstarted_c_lrequirements_getstartedCLrequirements_system}
If you’re installing the development environment and application framework on a development host, recommended system requirements for the Ubuntu desktop is at least 768 M\+B of R\+A\+M and 5 G\+B of disk space.

Ubuntu version 15.\+04 is officially supported.

If you’re downloading Legato source and toolchain files, you’ll need a minimum Internet connection speed of 4 Mpbs.\hypertarget{getstarted_c_lrequirements_getstartedCLrequirements_Host}{}\paragraph{Dev Host P\+C}\label{getstarted_c_lrequirements_getstartedCLrequirements_Host}
Different options are available for development host P\+Cs depending on needs. You can build, install, and work with Legato entirely from a Linux dev host, but target device recovery is only possible from Windows at this time. See \hyperlink{getstartedWindowsHost}{Windows Dev Host}.\hypertarget{getstarted_c_lrequirements_getstartedCLrequirements_tasks}{}\paragraph{Linux vs Windows tasks}\label{getstarted_c_lrequirements_getstartedCLrequirements_tasks}
Linux dev host is used to compile system software, flash Linux images, develop and debug applications.

Windows Host is used to flash Linux images, flash new modem firmware, recover device.\hypertarget{getstarted_c_lrequirements_getstartedCLrequirements_target}{}\paragraph{Target Device}\label{getstarted_c_lrequirements_getstartedCLrequirements_target}
Legato is comes preloaded on the Mang\+O\+H dev kit. You can also use the Air\+Prime – A\+R and W\+P Series dev kit. ~\newline






Copyright (C) Sierra Wireless Inc. Use of this work is subject to license. \hypertarget{getstartedPrepDevHost}{}\subsubsection{Prep Dev Host}\label{getstartedPrepDevHost}
The Linux development host needs a few things to be set up.\hypertarget{getstarted_prep_dev_host_instLinuxPackages}{}\paragraph{Install Linux Host Packages}\label{getstarted_prep_dev_host_instLinuxPackages}
Run the following to install necessary packages\+:

\begin{DoxyVerb}$ sudo apt-get install bison build-essential chrpath cifs-utils cmake coreutils curl desktop-file-utils diffstat docbook-utils doxygen fakeroot flex g++ gawk gcc git-core gitk graphviz help2man libgmp3-dev libmpfr-dev libreadline6-dev libsdl-dev libtool libxml2-dev libxml-libxml-perl make m4 ninja-build python-pip python-jinja2 python-pysqlite2 quilt samba scons sed subversion texi2html texinfo unzip wget
\end{DoxyVerb}
\hypertarget{getstarted_prep_dev_host_uninstCache}{}\paragraph{Uninstall ccache}\label{getstarted_prep_dev_host_uninstCache}
If {\ttfamily ccache} is installed, you must uninstall it so you can build some target images\+: \begin{DoxyVerb}$ sudo apt-get remove ccache
\end{DoxyVerb}
\hypertarget{getstarted_prep_dev_host_defSysShell}{}\paragraph{Set Default System Shell}\label{getstarted_prep_dev_host_defSysShell}
Some versions of Ubuntu default to dash system shell instead of bash, which will cause builds to fail. Run {\ttfamily echo} {\ttfamily \$0} to check your shell.

If you need to change it to use bash, run\+: \begin{DoxyVerb}$ sudo dpkg-reconfigure dash
\end{DoxyVerb}


Answer \char`\"{}\+No\char`\"{} to the question.





Copyright (C) Sierra Wireless Inc. Use of this work is subject to license. \hypertarget{getstartedDwnLd}{}\subsubsection{Download Files}\label{getstartedDwnLd}
You have to download the Legato app framework and Legato packages to install Legato from the command-\/line. The Sierra Wireless Source \href{http://source.sierrawireless.com/legato/}{\tt Legato} page has useful links.\hypertarget{getstarted_dwn_ld_getstartedDwnLdPkgs_legato_io}{}\paragraph{Framework}\label{getstarted_dwn_ld_getstartedDwnLdPkgs_legato_io}
Download Legato application framework and sample apps (with open source code) from ~\newline
 \href{http://www.legato.io/}{\tt legato.\+io}.\hypertarget{getstarted_dwn_ld_getstartedDwnLdPkgs_framework}{}\paragraph{Packages}\label{getstarted_dwn_ld_getstartedDwnLdPkgs_framework}
Legato Packages are available at \href{http://source.sierrawireless.com/resources/legato/downloads/}{\tt Legato Downloads}.

The following packages are available\+:

\href{http://source.sierrawireless.com/resources/legato/legatolinuxdist/}{\tt Legato Linux source distribution} ~\newline
 \href{http://source.sierrawireless.com/resources/legato/linuxtoolchain32bit/}{\tt Linux Toolchain -\/ Linux P\+C i686 32-\/bit} ~\newline
 \href{http://source.sierrawireless.com/resources/legato/linuxtoolchain64bit/}{\tt Linux Toolchain -\/ Linux P\+C x86 64-\/bit} ~\newline
 \href{http://source.sierrawireless.com/resources/legato/wp7firmware/}{\tt W\+P710x firmware package} ~\newline
 \href{http://source.sierrawireless.com/resources/legato/wp7withoutLegato/}{\tt W\+P710x package (without Legato)}





Copyright (C) Sierra Wireless Inc. Use of this work is subject to license. \hypertarget{getstartedCLinstall}{}\subsubsection{Install Source Files}\label{getstartedCLinstall}
After you’ve downloaded the current software packages, setup the target, and you a have Git\+Hub account, you\textquotesingle{}re ready to install Legato.\hypertarget{getstarted_c_linstall_getstartedCLinstall_toolchain}{}\paragraph{Toolchain}\label{getstarted_c_linstall_getstartedCLinstall_toolchain}
You\textquotesingle{}ll need to install the cross-\/toolchain for building A\+R\+M applications. Copy either the 64-\/bit or 32-\/bit \mbox{[}Legato-\/\+Toolchain-\/file\mbox{]} to /opt/swi and then extract it.


\begin{DoxyCode}
$ mkdir -p /opt/swi
$ chmod u+x [toolchain.sh file]
$ [toolchain.sh file]
\end{DoxyCode}


\begin{DoxyNote}{Note}
You may need to use {\ttfamily sudo} to run {\ttfamily mkdir}.
\end{DoxyNote}
\hypertarget{getstarted_c_linstall_getstartedCLinstall_framework}{}\paragraph{Framework}\label{getstarted_c_linstall_getstartedCLinstall_framework}
Install the Legato application framework on your dev host.

Create a directory where you\textquotesingle{}ll run Legato (e.\+g., mkdir Legato).

Then run (you\textquotesingle{}ll be prompted for your Git\+Hub credentials)\+: \begin{DoxyVerb}git clone https://github.com/legatoproject/legato-af.git
\end{DoxyVerb}






Copyright (C) Sierra Wireless Inc. Use of this work is subject to license. \hypertarget{getstartedCLbuild}{}\subsubsection{Build Legato}\label{getstartedCLbuild}
After you\textquotesingle{}ve installed the source files, you build the Legato framework for your host and target.\hypertarget{getstarted_c_lbuild_getstartedCLbuild_makeLegato}{}\paragraph{Make Legato}\label{getstarted_c_lbuild_getstartedCLbuild_makeLegato}
{\bfseries cd} to your Legato directory

{\bfseries Run} {\ttfamily make}.

These are \hyperlink{getstartedHostDirs}{Host Directories} installed.~\newline
\hypertarget{getstarted_c_lbuild_getstartedCLbuild_binlegs}{}\paragraph{bin/legs}\label{getstarted_c_lbuild_getstartedCLbuild_binlegs}
To set the your Legato environment, you must run {\ttfamily bin/legs} {\itshape every} time you open a new shell\+:

{\bfseries Run} {\ttfamily \$ bin/legs }\hypertarget{getstarted_c_lbuild_getstartedCLbuild_makeTarget}{}\paragraph{Make for Target}\label{getstarted_c_lbuild_getstartedCLbuild_makeTarget}
{\bfseries Run} {\ttfamily make} to build the framework and Air\+Vantage agent for the target (substitute {\ttfamily ar7} or {\ttfamily wp7} if building those targets)\+: 
\begin{DoxyCode}
$ make wp85 
\end{DoxyCode}


Or run this to build the framework for the target without the Air\+Vantage agent\+: 
\begin{DoxyCode}
$ make wp85 INCLUDE\_AIRVANTAGE=0 
\end{DoxyCode}






Copyright (C) Sierra Wireless Inc. Use of this work is subject to license. \hypertarget{getstartedHostDirs}{}\subsection{Host Directories}\label{getstartedHostDirs}
These are the top level Legato Dev Host directories installed\+:

{\bfseries 3rd\+Party } -\/ C\+Unit system used for automated unit testing.

{\bfseries apps } -\/ sample applications and test applications.

{\bfseries bin } -\/ created by build system and populated with executable files run on the development host that ran the build. Contains files like {\ttfamily mkapp} and {\ttfamily mkexe}.

{\bfseries build} -\/ results of a framework build. Includes {\ttfamily build/tools} (tools used by the build system) and {\ttfamily build/target} (output of a build for a specific target e.\+g., /build/wp85).

{\bfseries cmake } -\/ C\+Make scripts used by the build system.

{\bfseries components } -\/ pre-\/built audio, data, modem and positioning components.

{\bfseries framework} -\/ source code for the Legato framework.

{\bfseries interfaces } -\/ pre-\/built audio, data, modem and positioning interfaces.

{\bfseries platform\+Services } pre-\/built app definition files for audio, data, modem and positioning services.

{\bfseries target\+Files } -\/ files for installation on target devices.





Copyright (C) Sierra Wireless Inc. Use of this work is subject to license. \hypertarget{getstartedTargetSW}{}\subsubsection{Setup Target Software}\label{getstartedTargetSW}
To setup target software, you copy startup scripts from the host to the target, and then install libraries and executables.\hypertarget{getstarted_target_s_w_getstartedTargetSetup_copyStartScripts}{}\paragraph{Copy Startup Scripts}\label{getstarted_target_s_w_getstartedTargetSetup_copyStartScripts}
Startup scripts are copied automatically to the target when {\ttfamily instlegato} is run the first time (see \hyperlink{getstarted_target_s_w_getstartedTargetSetup_copyLibExe}{instlegato Copy Libs \& Exes}) .

You can also copy the scripts manually. Using ssh, access the target\+:


\begin{DoxyCode}
ssh root@<target ip address> 
\end{DoxyCode}
 When prompted for a password, press enter.

Then create the startup directory\+: 
\begin{DoxyCode}
mkdir -p  /mnt/flash/startup 
\end{DoxyCode}


Then copy the startup files to the target by running the following from the dev host\+: 
\begin{DoxyCode}
scp targetFiles/mdm-9x15/startup/fg\_* root@<target ip addr>:/mnt/flash/startup
\end{DoxyCode}


Scripts beginning with {\ttfamily fg\+\_\+} are run first during startup, and then everything else is runs in the background. You can add more scripts to customize your target. {\ttfamily fg\+\_\+} scripts must have executable permissions or they will not run.

\begin{DoxyNote}{Note}
In the rare case you need to prevent legato starting automatically, run 
\begin{DoxyCode}
touch /mnt/flash/startup/STOPLEGATO 
\end{DoxyCode}

\end{DoxyNote}
\hypertarget{getstarted_target_s_w_getstartedTargetSetup_rebootTarget}{}\paragraph{Reboot Target}\label{getstarted_target_s_w_getstartedTargetSetup_rebootTarget}
Wait for the target to fully reboot. The target I\+P address may have changed. Check it using {\ttfamily ifconfig} through the serial console before proceeding.\hypertarget{getstarted_target_s_w_getstartedTargetSetup_copyLibExe}{}\paragraph{instlegato Copy Libs \& Exes}\label{getstarted_target_s_w_getstartedTargetSetup_copyLibExe}
To copy libraries and executables to the target, run\+:


\begin{DoxyCode}
$ instlegato build/wp85 <target ip address>
\end{DoxyCode}


It will automatically start the Legato runtime components.

If target startup scripts don\textquotesingle{}t exist or don\textquotesingle{}t match the ones in\+: 
\begin{DoxyCode}
targetFiles/mdm-9x15/startup 
\end{DoxyCode}
 they\textquotesingle{}ll be copied to the target.

Old startup files from the target will be copied to the host here\+: 
\begin{DoxyCode}
targetFiles/mdm-9x15/backup 
\end{DoxyCode}


After the startup files are finished updating, the target will reboot.

After the target restarts, run {\ttfamily instlegato} again to finish installation.

Then you can build and run apps.





Here\textquotesingle{}s \hyperlink{getstartedTargetIPv6}{Config E\+C\+M I\+Pv6} info.

These are \hyperlink{getstartedTargetDirs}{Target Directories} installed.





Copyright (C) Sierra Wireless Inc. Use of this work is subject to license. \hypertarget{getstartedTargetIPv6}{}\subsection{Config E\+C\+M I\+Pv6}\label{getstartedTargetIPv6}
Setting up target networking for I\+Pv6 on E\+C\+M can be done through the host or directly on the target.

This host code sample uses a /127 prefix (only 2 addresses, 0 \& 1)\+:

\begin{DoxyVerb}$ sudo ip -6 addr add fd42::0 dev usb0
$ sudo ip -6 route add fd42::/127 dev usb0
\end{DoxyVerb}


To test the connection from the host to the target, run\+:

\begin{DoxyVerb} # ping6 fd42::0\end{DoxyVerb}


Here\textquotesingle{}s the same code sample that you run on the target\+:

\begin{DoxyVerb}# ip -6 addr add fd42::1 dev usb0
# ip -6 route add fd42::/127 dev usb0
\end{DoxyVerb}


To test the connection from the target to the host, run

\begin{DoxyVerb}$ ping6 fd42::1
\end{DoxyVerb}


To permanently setup addresses on both sides, add the following to {\ttfamily /etc/network/interfaces} on the target\+:

\begin{DoxyVerb}auto usb0
iface usb0 inet6 static
    address fd42::0
    netmask 127
\end{DoxyVerb}






Copyright (C) Sierra Wireless Inc. Use of this work is subject to license. \hypertarget{getstartedTargetDirs}{}\subsection{Target Directories}\label{getstartedTargetDirs}
These are the top level Legato Dev Target directories installed\+:

{\bfseries logs} -\/ logs for all target sessions.

{\bfseries \mbox{[}app directories\mbox{]} }-\/ any installed applications.

{\bfseries root } has these directories -\/ ~\newline
 bin ~\newline
 boot ~\newline
 cdrom ~\newline
 dev ~\newline
 etc ~\newline
 home ~\newline
 lib ~\newline
 lib64 ~\newline
 lost+found ~\newline
 media ~\newline
 mnt ~\newline
 opt ~\newline
 proc ~\newline
 rofs ~\newline
 root ~\newline
 run ~\newline
 sbin ~\newline
 srv ~\newline
 sys ~\newline
 tmp ~\newline
 usr ~\newline
 var ~\newline






Copyright (C) Sierra Wireless Inc. Use of this work is subject to license. \hypertarget{getstartedSampleApps}{}\subsubsection{Use Sample Apps}\label{getstartedSampleApps}
There are a few easy ways to start developing Legato apps.\hypertarget{getstarted_sample_apps_getstartedSampleApps_preBuilt}{}\paragraph{Pre-\/\+Built}\label{getstarted_sample_apps_getstartedSampleApps_preBuilt}
Use the pre-\/built sample apps\+: ~\newline

\begin{DoxyItemize}
\item \hyperlink{sampleApps}{Sample Apps} available in the app/sample/ directories.
\item \href{http://www.legato.io}{\tt legato.\+io} to clone the sample apps from Git\+Hub.
\end{DoxyItemize}\hypertarget{getstarted_sample_apps_getstartedSampleApps_source}{}\paragraph{Source}\label{getstarted_sample_apps_getstartedSampleApps_source}
Develop Legato apps from scratch\+:

\hyperlink{basicApps}{Build Apps} ~\newline
 \hyperlink{howToMain}{How To}





Copyright (C) Sierra Wireless Inc. Use of this work is subject to license. \hypertarget{legatoRelNotesBeta}{}\section{Release Notes}\label{legatoRelNotesBeta}
\href{http://www.legato.io/legato-docs/15_10/legato_rel_notes_15_10.html}{\tt 15.\+10 Beta Release} ~\newline
 \href{http://www.legato.io/legato-docs/15_08/legato_rel_notes_15_08.html}{\tt 15.\+08 Beta Release} ~\newline
 \href{http://www.legato.io/legato-docs/15_05/legato_rel_notes_15_05.html}{\tt 15.\+05 Beta Release} ~\newline
 \href{http://www.legato.io/legato-docs/15_01/legato_rel_notes_15_01.html}{\tt 15.\+01 Beta Release} ~\newline
 \href{http://www.legato.io/legato-docs/14_10/legato_rel_notes_14_10.html}{\tt 14.\+10 Beta Release} ~\newline
 \href{http://www.legato.io/legato-docs/14_07/legato_rel_notes_14_07.html}{\tt 14.\+07 Beta Release} ~\newline
 \href{http://www.legato.io/legato-docs/14_04/legatoRelNotesBeta14_04.htm}{\tt 14.\+04 Beta Release} ~\newline






Copyright (C) Sierra Wireless Inc. Use of this work is subject to license. \hypertarget{legatoUpgradeBeta}{}\section{Upgrade Notes}\label{legatoUpgradeBeta}
Here\textquotesingle{}s important upgrade info for Legato Beta\+:

\href{http://www.legato.io/legato-docs/15_10/legatoUpgrade15_10.html}{\tt Beta 15.\+08 to 15.\+10} ~\newline
 \href{http://www.legato.io/legato-docs/15_08/legatoUpgrade15_08.html}{\tt Beta 15.\+05 to 15.\+08} ~\newline
 \href{http://www.legato.io/legato-docs/15_05/legatoUpgrade15_05.html}{\tt Beta 15.\+01 to 15.\+05} ~\newline
 \href{http://www.legato.io/legato-docs/15_01/legatoUpgrade15_01.html}{\tt Beta 14.\+10 to 15.\+01} ~\newline
 \href{http://www.legato.io/legato-docs/14_10/legatoUpgrade14_10.html}{\tt Beta 14.\+07 to 14.\+10} ~\newline
 \href{http://www.legato.io/legato-docs/14_07/legatoUpgradeBeta14_07.html}{\tt Beta 14.\+04 to 14.\+07} ~\newline
 \href{http://www.legato.io/legato-docs/14_04/upgradeLegato.htm}{\tt Alpha\+B to Beta 14.\+04} ~\newline






Copyright (C) Sierra Wireless Inc. Use of this work is subject to license. 