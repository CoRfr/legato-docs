This section contains detailed info about Legato\textquotesingle{}s C Language library used for low-\/level routines like commonly used data structures and O\+S services A\+P\+Is.





The C A\+P\+Is\textquotesingle{} \hyperlink{c__a_p_is_cApiOverview}{Overview} has high-\/level info.





\hyperlink{c_le_build_cfg}{Legato Build Configuration} ~\newline
 \hyperlink{c_basics}{Basic Type and Constant Definitions} ~\newline
 \hyperlink{c_args}{Command Line Arguments A\+P\+I} ~\newline
 \hyperlink{c_dir}{Directory A\+P\+I} ~\newline
 \hyperlink{c_doublyLinkedList}{Doubly Linked List A\+P\+I} ~\newline
 \hyperlink{c_memory}{Dynamic Memory Allocation A\+P\+I} ~\newline
 \hyperlink{c_eventLoop}{Event Loop A\+P\+I} ~\newline
 \hyperlink{c_fdMonitor}{File Descriptor Monitor A\+P\+I} ~\newline
 \hyperlink{c_flock}{File Locking A\+P\+I} ~\newline
 \hyperlink{c_hashmap}{Hash\+Map A\+P\+I} ~\newline
 \hyperlink{c_hex}{Hex string A\+P\+I} ~\newline
 \hyperlink{c_logging}{Logging A\+P\+I} ~\newline
 \hyperlink{c_messaging}{Low-\/\+Level Messaging A\+P\+I} ~\newline
 \hyperlink{c_mutex}{Mutex A\+P\+I} ~\newline
 \hyperlink{c_path}{Path A\+P\+I} ~\newline
 \hyperlink{c_pathIter}{Path Iterator A\+P\+I} ~\newline
 \hyperlink{c_print}{Print A\+P\+Is} ~\newline
 \hyperlink{c_clock}{System Clock A\+P\+I} ~\newline
 \hyperlink{c_safeRef}{Safe References A\+P\+I} ~\newline
 \hyperlink{c_semaphore}{Semaphore A\+P\+I} ~\newline
 \hyperlink{c_signals}{Signals A\+P\+I} ~\newline
 \hyperlink{c_singlyLinkedList}{Singly Linked List A\+P\+I} ~\newline
 \hyperlink{c_threading}{Thread Control A\+P\+I} ~\newline
 \hyperlink{c_timer}{Timer A\+P\+I} ~\newline
 \hyperlink{c_test}{Unit Testing A\+P\+I} ~\newline
 \hyperlink{c_utf8}{U\+T\+F-\/8 String Handling A\+P\+I} ~\newline
\hypertarget{c__a_p_is_cApiOverview}{}\section{Overview}\label{c__a_p_is_cApiOverview}
Here is some background info on Legato\textquotesingle{}s C Language A\+P\+Is.\hypertarget{c__a_p_is_Object-Oriented}{}\subsection{Design}\label{c__a_p_is_Object-Oriented}
The Legato framework is constructed in an object-\/oriented manner.

The C programming language was created before object-\/oriented programming was popular so it doesn\textquotesingle{}t have native support for O\+O\+P features like inheritance, private object members, member functions, and overloading. But object-\/oriented designs can still be implemented in C.

In the Legato C A\+P\+Is, classes are hidden behind opaque \char`\"{}reference\char`\"{} data types. You can get references to objects created behind the scenes in Legato, but you can never see the structure of those objects. The implementation is hidden from view. Access to object properties is made available through accessor functions.\hypertarget{c__a_p_is_Opaque}{}\subsection{Types}\label{c__a_p_is_Opaque}
The basic \char`\"{}opaque data type\char`\"{} offered by the C programming language is the \char`\"{}void pointer\char`\"{} (void $\ast$). The idea is that a pointer to an object of type {\itshape T} can be cast to point to a void type before being passed outside of the module that implements {\itshape T}.

This makes it impossible for anyone outside of the module that implements {\itshape T} to dereference the pointer or access anything about the implementation of {\itshape T}. This way, the module that implements {\itshape T} is free to change the implementation of {\itshape T} in any way needed without worrying about breaking code outside of that module.

The problem with the void pointer type is that it throws away type information. At compile time, this makes it impossible to detect that a variable with opaque type {\itshape T} has been passed into a function with some other pointer type {\itshape P}.

To overcome this, Legato uses incomplete types to implement its opaque types. For example, there are declarations similar to the following in Legato C A\+P\+I header files\+:


\begin{DoxyCode}
\textcolor{comment}{// Declare a reference type for referring to Foo objects.}
\textcolor{keyword}{typedef} \textcolor{keyword}{struct }le\_foo* le\_foo\_Ref\_t;
\end{DoxyCode}


But \char`\"{}struct le\+\_\+foo\char`\"{} would {\itshape not} be defined in the A\+P\+I header or {\itshape anywhere} outside of the hypothetical Foo A\+P\+I\textquotesingle{}s implementation files. This makes \char`\"{}struct le\+\_\+foo\char`\"{} an \char`\"{}incomplete type\char`\"{} for all code outside of the Foo A\+P\+I implementation files. Incomplete types can\textquotesingle{}t be used because the compiler doesn\textquotesingle{}t have enough information about them to generate any code that uses them. But {\itshape pointers} to incomplete types {\itshape can} be passed around because the compiler always knows the pointer size. The compiler knows that one incomplete type is {\itshape not} necessarily interchangeable with another. It won\textquotesingle{}t allow a pointer to an incomplete type to be used where a pointer to another type is expected.





Copyright (C) Sierra Wireless Inc. Use of this work is subject to license. \hypertarget{c_le_build_cfg}{}\section{Legato Build Configuration}\label{c_le_build_cfg}
In the file {\ttfamily le\+\_\+build\+\_\+conifg.\+h} are a number of preprocessor macros. Uncommenting these macros enables a non-\/standard feature of the framework.



\hypertarget{c_le_build_cfg_bld_cfg_mem_trace}{}\subsection{L\+E\+\_\+\+M\+E\+M\+\_\+\+T\+R\+A\+C\+E}\label{c_le_build_cfg_bld_cfg_mem_trace}
When {\ttfamily L\+E\+\_\+\+M\+E\+M\+\_\+\+T\+R\+A\+C\+E} is defined, the memory subsystem will create a trace point for every memory pool created. The name of the tracepoint will be the same of the pool, and is of the form \char`\"{}component.\+pool\+Name\char`\"{}.\hypertarget{c_le_build_cfg_bld_cfg_mem_valgrind}{}\subsection{L\+E\+\_\+\+M\+E\+M\+\_\+\+V\+A\+L\+G\+R\+I\+N\+D}\label{c_le_build_cfg_bld_cfg_mem_valgrind}
When {\ttfamily L\+E\+\_\+\+M\+E\+M\+\_\+\+V\+A\+L\+G\+R\+I\+N\+D} is enabled the memory system doesn\textquotesingle{}t use pools anymore but in fact switches to use malloc/free per-\/block. This way, tools like valgrind can be used on a Legato executable.





Copyright (C) Sierra Wireless Inc. Use of this work is subject to license. \hypertarget{c_basics}{}\section{Basic Type and Constant Definitions}\label{c_basics}
Cardinal types and commonly-\/used constants form the basic foundation on which everything else is built. These include error codes, portable integer types, and helpful macros that make things easier to use.

See \hyperlink{le__basics_8h}{le\+\_\+basics.\+h} for basic cardinal types and commonly-\/used constants info.





Copyright (C) Sierra Wireless Inc. Use of this work is subject to license. \hypertarget{c_args}{}\section{Command Line Arguments A\+P\+I}\label{c_args}
\hyperlink{le__args_8h}{A\+P\+I Reference}





When a program starts, arguments may be passed from the command line.

\begin{DoxyVerb}$ foo bar baz
\end{DoxyVerb}


In a traditional C/\+C++ program, these arguments are received as parameters to {\ttfamily main()}. The Legato framework makes these available to component via function calls instead.\hypertarget{c_args_c_args_by_index}{}\subsection{Argument Access By Index}\label{c_args_c_args_by_index}
The arguments can be fetched by index using {\ttfamily \hyperlink{le__args_8h_a5ebca8229facd069785639cb3c1e273a}{le\+\_\+arg\+\_\+\+Get\+Arg()}}. The first argument has index 0, the second argument has index 1, etc. In the above example, {\bfseries bar} has index 0 and {\bfseries baz} has index 1.

The number of available arguments is obtained using {\ttfamily \hyperlink{le__args_8h_a6fbbeb423104e6eb92fe47ef42b7310a}{le\+\_\+arg\+\_\+\+Num\+Args()}}.

The name of the program is obtained using {\ttfamily \hyperlink{le__args_8h_add0db0cb93135a6f18f336bd7885cf75}{le\+\_\+arg\+\_\+\+Get\+Program\+Name()}}.

The program name and all arguments are assumed to be Null-\/terminated U\+T\+F-\/8 strings. For more information about U\+T\+F-\/8 strings see \hyperlink{c_utf8}{U\+T\+F-\/8 String Handling A\+P\+I}.\hypertarget{c_args_c_args_options}{}\subsection{Options}\label{c_args_c_args_options}
Options are arguments that start with a \char`\"{}-\/\char`\"{} or \char`\"{}-\/-\/\char`\"{}.

To search for a specific option, the following functions are provided\+:
\begin{DoxyItemize}
\item \hyperlink{le__args_8h_af3b6949dd9d93b8461f3bb64d565fa93}{le\+\_\+arg\+\_\+\+Get\+Flag\+Option()} -\/ Searches for a given flag (flags don\textquotesingle{}t have values).
\item \hyperlink{le__args_8h_aac66ccbb038e10c117a685d6eae5a684}{le\+\_\+arg\+\_\+\+Get\+Int\+Option()} -\/ Searches for a given option with an integer value.
\item \hyperlink{le__args_8h_af096ac39dfa56aedaa4490653891c222}{le\+\_\+arg\+\_\+\+Get\+String\+Option()} -\/ Searches for a given option with a string value.
\end{DoxyItemize}

\begin{DoxyNote}{Note}
A \char`\"{}-\/\char`\"{} or \char`\"{}-\/-\/\char`\"{} by itself is not considered an option. These are treated as positional arguments.
\end{DoxyNote}
\hypertarget{c_args_c_args_positional}{}\subsection{Positional Arguments}\label{c_args_c_args_positional}
Positional arguments are arguments that {\bfseries do not} start with a \char`\"{}-\/\char`\"{} or \char`\"{}-\/-\/\char`\"{}; except for \char`\"{}-\/\char`\"{} or \char`\"{}-\/-\/\char`\"{} by itself (these are positional arguments).

For example, the following command line has four positional arguments (\char`\"{}foo\char`\"{}, \char`\"{}bar\char`\"{}, \char`\"{}-\/\char`\"{}, and \char`\"{}-\/-\/\char`\"{}). A flag option (\char`\"{}-\/x\char`\"{}), and two string options (\char`\"{}-\/f ./infile\char`\"{} and \char`\"{}-\/-\/output=/tmp/output file\char`\"{}) are intermixed with the positional arguments.


\begin{DoxyCode}
$ myExe -x foo -f ./infile - \textcolor{stringliteral}{"--output=/tmp/output file"} bar --
\end{DoxyCode}


In this example, \char`\"{}foo\char`\"{} is the first positional argument, \char`\"{}-\/\char`\"{} is the second, \char`\"{}bar\char`\"{} is the third, and \char`\"{}-\/-\/\char`\"{} is the fourth.

Positional arguments are retrieved using the \hyperlink{c_args_c_args_scanner}{Argument Scanner} and \hyperlink{le__args_8h_a525bef6095a4655e97008e27a4829d44}{le\+\_\+arg\+\_\+\+Add\+Positional\+Callback()}.\hypertarget{c_args_c_args_scanner}{}\subsection{Argument Scanner}\label{c_args_c_args_scanner}
If you\textquotesingle{}re building a command-\/line application with a complex argument list, you may want to use the Legato framework\textquotesingle{}s argument scanner feature. It supports many options commonly seen in command-\/line tools and performs a lot of the error checking and reporting for you.

For example, the {\ttfamily command\+Line} sample application implements a tool called {\ttfamily file\+Info} that prints information about files or directories. It is flexible about the order of appearance of options on the command-\/line. For example, the following are equivalent\+:

\begin{DoxyVerb}# fileInfo -x -mc 20 permissions *
\end{DoxyVerb}


\begin{DoxyVerb}# fileInfo permissions --max-count=20 * -x
\end{DoxyVerb}


Note that
\begin{DoxyItemize}
\item \char`\"{}-\/mc 20\char`\"{} and \char`\"{}-\/-\/max-\/count=20\char`\"{} are different ways of specifying the same option;
\item the order of appearance of the options can change;
\item options (which start with \textquotesingle{}-\/\textquotesingle{} or \textquotesingle{}--\textquotesingle{}) and other arguments can be intermixed.
\end{DoxyItemize}\hypertarget{c_args_c_args_scanner_usage}{}\subsubsection{Usage}\label{c_args_c_args_scanner_usage}
A program (typically inside a {\ttfamily C\+O\+M\+P\+O\+N\+E\+N\+T\+\_\+\+I\+N\+I\+T}) can call functions to register variables to be set or call-\/back functions to be called when certain arguments are passed to the program.

After registering the variables and call-\/back functions, \hyperlink{le__args_8h_af44485fc914a7ac6f562d23d66c3410c}{le\+\_\+arg\+\_\+\+Scan()} is called to parse the argument list.

The following functions can be called before \hyperlink{le__args_8h_af44485fc914a7ac6f562d23d66c3410c}{le\+\_\+arg\+\_\+\+Scan()} is called to register variables to be set or call-\/back functions to be called by \hyperlink{le__args_8h_af44485fc914a7ac6f562d23d66c3410c}{le\+\_\+arg\+\_\+\+Scan()}\+:


\begin{DoxyItemize}
\item \hyperlink{le__args_8h_a889bb72c62d8590d61170a069219e852}{le\+\_\+arg\+\_\+\+Set\+Flag\+Var()}
\item \hyperlink{le__args_8h_a27f1486b1e855559158e218a7d93ce73}{le\+\_\+arg\+\_\+\+Set\+Int\+Var()}
\item \hyperlink{le__args_8h_a56d0b80e404966a00c87ec662fea23a8}{le\+\_\+arg\+\_\+\+Set\+String\+Var()}
\item \hyperlink{le__args_8h_a4594892b35d4e0a6d7551e9c371919fc}{le\+\_\+arg\+\_\+\+Set\+Flag\+Callback()}
\item \hyperlink{le__args_8h_a40e96c54132708b0637c3d696e3d060d}{le\+\_\+arg\+\_\+\+Set\+Int\+Callback()}
\item \hyperlink{le__args_8h_a41b845bab467f4b1e7fcae3d600e88b2}{le\+\_\+arg\+\_\+\+Set\+String\+Callback()}
\item \hyperlink{le__args_8h_a525bef6095a4655e97008e27a4829d44}{le\+\_\+arg\+\_\+\+Add\+Positional\+Callback()}
\end{DoxyItemize}

There are essentially 3 forms of function\+:


\begin{DoxyItemize}
\item le\+\_\+arg\+\_\+\+Set\+Xxx\+Var() -\/ Registers a variable to be set by \hyperlink{le__args_8h_af44485fc914a7ac6f562d23d66c3410c}{le\+\_\+arg\+\_\+\+Scan()} when it sees a certain argument starting with \textquotesingle{}-\/\textquotesingle{} or \textquotesingle{}--\textquotesingle{}.
\item le\+\_\+arg\+\_\+\+Set\+Xxx\+Callback() -\/ Registers a call-\/back function to be called by \hyperlink{le__args_8h_af44485fc914a7ac6f562d23d66c3410c}{le\+\_\+arg\+\_\+\+Scan()} when it sees a certain argument starting with \textquotesingle{}-\/\textquotesingle{} or \textquotesingle{}--\textquotesingle{}.
\item \hyperlink{le__args_8h_a525bef6095a4655e97008e27a4829d44}{le\+\_\+arg\+\_\+\+Add\+Positional\+Callback()} -\/ Registers a call-\/back function to be called by \hyperlink{le__args_8h_af44485fc914a7ac6f562d23d66c3410c}{le\+\_\+arg\+\_\+\+Scan()} when it sees an argument that does not start with either \textquotesingle{}-\/\textquotesingle{} or \textquotesingle{}--\textquotesingle{}.
\end{DoxyItemize}

\hyperlink{le__args_8h_a525bef6095a4655e97008e27a4829d44}{le\+\_\+arg\+\_\+\+Add\+Positional\+Callback()} can be called multiple times. This constructs a list of call-\/back functions, where the first function in that list will be called for the first positional argument, the second function in the list will be called for the second positional argument, etc.

Normally, an error will be generated if there are not the same number of positional arguments as there are positional callbacks in the list. However, this behaviour can be changed\+:


\begin{DoxyItemize}
\item If \hyperlink{le__args_8h_ab646cfcb831e13312bff496221e74acc}{le\+\_\+arg\+\_\+\+Allow\+More\+Positional\+Args\+Than\+Callbacks()} is called, then the last callback in the list will be called for each of the extra positional arguments on the command-\/line.
\item If \hyperlink{le__args_8h_aedcaae9ee7e7cc9cf83c30435f6ae653}{le\+\_\+arg\+\_\+\+Allow\+Less\+Positional\+Args\+Than\+Callbacks()} will allow shorter argument lists, which will result in one or more of the last callbacks in the list not being called.
\end{DoxyItemize}

\hyperlink{le__utf8_8h_a680a92fafea1ed72dedb80b52be32a06}{le\+\_\+utf8\+\_\+\+Parse\+Int()} can be used by a positional callback to convert the string value it receives into an integer value, if needed.\hypertarget{c_args_c_args_parser_example}{}\subsubsection{Example}\label{c_args_c_args_parser_example}

\begin{DoxyCode}
\textcolor{comment}{// Set IsExtreme to true if the -x or --extreme appears on the command-line.}
\hyperlink{le__args_8h_a889bb72c62d8590d61170a069219e852}{le\_arg\_SetFlagVar}(&IsExtreme, \textcolor{stringliteral}{"x"}, \textcolor{stringliteral}{"extreme"});

\textcolor{comment}{// Set Count to the value N given by "-mc N" or "--max-count=N".}
\hyperlink{le__args_8h_a27f1486b1e855559158e218a7d93ce73}{le\_arg\_SetIntVar}(&MaxCount, \textcolor{stringliteral}{"mc"}, \textcolor{stringliteral}{"max-count"});

\textcolor{comment}{// Register a function to be called if -h or --help appears on the command-line.}
\hyperlink{le__args_8h_a4594892b35d4e0a6d7551e9c371919fc}{le\_arg\_SetFlagCallback}(PrintHelp, \textcolor{stringliteral}{"h"}, \textcolor{stringliteral}{"help"});

\textcolor{comment}{// The first argument that doesn't start with '-' or '--' should be a command.}
\hyperlink{le__args_8h_a525bef6095a4655e97008e27a4829d44}{le\_arg\_AddPositionalCallback}(SetCommand);

\textcolor{comment}{// All other arguments that don't start with '-' or '--' should be file paths.}
\hyperlink{le__args_8h_a525bef6095a4655e97008e27a4829d44}{le\_arg\_AddPositionalCallback}(SetFilePath);
\hyperlink{le__args_8h_ab646cfcb831e13312bff496221e74acc}{le\_arg\_AllowMorePositionalArgsThanCallbacks}();

\textcolor{comment}{// Perform command-line argument processing.}
\hyperlink{le__args_8h_af44485fc914a7ac6f562d23d66c3410c}{le\_arg\_Scan}();
\end{DoxyCode}
\hypertarget{c_args_c_args_parser_errorHandling}{}\subsubsection{Error Handling}\label{c_args_c_args_parser_errorHandling}
If a program wishes to try to recover from errors on the command-\/line or to generate its own special form of error message, it can use \hyperlink{le__args_8h_a5128be1cbe2c7b30f1f697f8b5594479}{le\+\_\+arg\+\_\+\+Set\+Error\+Handler()} to register a callback function to be called to handle errors.

If no error handler is set, the default handler will print an error message to the standard error stream and terminate the process with an exit code of E\+X\+I\+T\+\_\+\+F\+A\+I\+L\+U\+R\+E.

Error conditions that can be reported to the error handler are described in the documentation for \hyperlink{le__args_8h_a52331d81cc8f4a20089598b0ab362786}{le\+\_\+arg\+\_\+\+Error\+Handler\+Func\+\_\+t}.


\begin{DoxyCode}
\textcolor{comment}{// Set Count to the value N given by "-mc N" or "--max-count=N".}
\hyperlink{le__args_8h_a27f1486b1e855559158e218a7d93ce73}{le\_arg\_SetIntVar}(&MaxCount, \textcolor{stringliteral}{"mc"}, \textcolor{stringliteral}{"max-count"});

\textcolor{comment}{// Register my own error handler.}
\hyperlink{le__args_8h_a5128be1cbe2c7b30f1f697f8b5594479}{le\_arg\_SetErrorHandler}(HandleArgError);

\textcolor{comment}{// Perform command-line argument processing.}
\hyperlink{le__args_8h_af44485fc914a7ac6f562d23d66c3410c}{le\_arg\_Scan}();
\end{DoxyCode}
\hypertarget{c_args_c_args_writingYourOwnMain}{}\subsection{Writing Your Own main()?}\label{c_args_c_args_writingYourOwnMain}
If you are not using a main() function that is auto-\/generated by the Legato application framework\textquotesingle{}s build tools ({\ttfamily mksys}, {\ttfamily mkapp}, or {\ttfamily mkexe} ), then you must call \hyperlink{le__args_8h_aefd062c124811c5de122a06907e19b56}{le\+\_\+arg\+\_\+\+Set\+Args()} to pass {\ttfamily argc} and {\ttfamily argv} to the argument parsing system before using any other {\ttfamily le\+\_\+arg} functions.





Copyright (C) Sierra Wireless Inc. Use of this work is subject to license. \hypertarget{c_dir}{}\section{Directory A\+P\+I}\label{c_dir}
\hyperlink{le__dir_8h}{A\+P\+I Reference}\hypertarget{c_dir_c_dir_create}{}\subsection{Creating Directories}\label{c_dir_c_dir_create}
To create a directory at a specific location, call {\ttfamily \hyperlink{le__dir_8h_a7ac7d25b67f2e47127084677626d5344}{le\+\_\+dir\+\_\+\+Make()}} passing in the path name of the directory to create. All directories in the path name except the last directory (the directory to be created) must exist prior to calling \hyperlink{le__dir_8h_a7ac7d25b67f2e47127084677626d5344}{le\+\_\+dir\+\_\+\+Make()}.

To create all directories in a specified path use {\ttfamily \hyperlink{le__dir_8h_a41fc915e2a21ea91dabe335f1316df74}{le\+\_\+dir\+\_\+\+Make\+Path()}}.

With both \hyperlink{le__dir_8h_a7ac7d25b67f2e47127084677626d5344}{le\+\_\+dir\+\_\+\+Make()} and \hyperlink{le__dir_8h_a41fc915e2a21ea91dabe335f1316df74}{le\+\_\+dir\+\_\+\+Make\+Path()} the calling process must have write and search permissions on all directories in the path.\hypertarget{c_dir_c_dir_delete}{}\subsection{Removing Directories}\label{c_dir_c_dir_delete}
To remove a directory and everything in the directory (including all files and sub-\/directories) use {\ttfamily \hyperlink{le__dir_8h_a9eca51d0e3031f9dee7b875a62c8b1e0}{le\+\_\+dir\+\_\+\+Remove\+Recursive()}}.\hypertarget{c_dir_c_dir_read}{}\subsection{Reading Directories}\label{c_dir_c_dir_read}
To read the contents of a directory use the P\+O\+S\+I\+X function {\ttfamily open\+Dir()}.





Copyright (C) Sierra Wireless Inc. Use of this work is subject to license. \hypertarget{c_doublyLinkedList}{}\section{Doubly Linked List A\+P\+I}\label{c_doublyLinkedList}
\hyperlink{le__doubly_linked_list_8h}{A\+P\+I Reference}





A doubly linked list is a data structure consisting of a group of nodes linked together linearly. Each node consists of data elements with links to the next node and previous nodes. The main advantage of linked lists (over simple arrays) is the nodes can be inserted and removed anywhere in the list without reallocating the entire array. Linked list nodes don\textquotesingle{}t need to be stored contiguously in memory, but nodes then you can\textquotesingle{}t access by index, you have to be access by traversing the list.\hypertarget{c_doubly_linked_list_dls_createList}{}\subsection{Creating and Initializing Lists}\label{c_doubly_linked_list_dls_createList}
To create and initialize a linked list the user must create a \hyperlink{structle__dls___list__t}{le\+\_\+dls\+\_\+\+List\+\_\+t} typed list and assign L\+E\+\_\+\+D\+L\+S\+\_\+\+L\+I\+S\+T\+\_\+\+I\+N\+I\+T to it. The assignment of L\+E\+\_\+\+D\+L\+S\+\_\+\+L\+I\+S\+T\+\_\+\+I\+N\+I\+T can be done either when the list is declared or after its declared. The list {\bfseries must} be initialized before it can be used.


\begin{DoxyCode}
\textcolor{comment}{// Create and initialized the list in the declaration.}
\hyperlink{structle__dls___list__t}{le\_dls\_List\_t} MyList = \hyperlink{le__doubly_linked_list_8h_a68f28b61cdfd004591f24730b4d5a740}{LE\_DLS\_LIST\_INIT};
\end{DoxyCode}


Or


\begin{DoxyCode}
\textcolor{comment}{// Create list.}
\hyperlink{structle__dls___list__t}{le\_dls\_List\_t} MyList;

\textcolor{comment}{// Initialize the list.}
MyList = \hyperlink{le__doubly_linked_list_8h_a68f28b61cdfd004591f24730b4d5a740}{LE\_DLS\_LIST\_INIT};
\end{DoxyCode}


{\bfseries  Elements of \hyperlink{structle__dls___list__t}{le\+\_\+dls\+\_\+\+List\+\_\+t} M\+U\+S\+T N\+O\+T be accessed directly by the user. }\hypertarget{c_doubly_linked_list_dls_createNode}{}\subsection{Creating and Accessing Nodes}\label{c_doubly_linked_list_dls_createNode}
Nodes can contain any data in any format and is defined and created by the user. The only requirement for nodes is that it must contain a {\ttfamily \hyperlink{structle__dls___link__t}{le\+\_\+dls\+\_\+\+Link\+\_\+t}} link member. The link member must be initialized by assigning L\+E\+\_\+\+D\+L\+S\+\_\+\+L\+I\+N\+K\+\_\+\+I\+N\+I\+T to it before it can be used. Nodes can then be added to the list by passing their links to the add functions (\hyperlink{le__doubly_linked_list_8h_a90f9072a55ef0cb573bbdad91e34d368}{le\+\_\+dls\+\_\+\+Stack()}, \hyperlink{le__doubly_linked_list_8h_a264df63b847a9c485df0bf9050ac5deb}{le\+\_\+dls\+\_\+\+Queue()}, etc.). For example\+:


\begin{DoxyCode}
\textcolor{comment}{// The node may be defined like this.}
\textcolor{keyword}{typedef} \textcolor{keyword}{struct}
\{
     dataType someUserData;
     ...
     \hyperlink{structle__dls___link__t}{le\_dls\_Link\_t} myLink;

\}
MyNodeClass\_t;

\textcolor{comment}{// Create and initialize the list.}
\hyperlink{structle__dls___list__t}{le\_dls\_List\_t} MyList = \hyperlink{le__doubly_linked_list_8h_a68f28b61cdfd004591f24730b4d5a740}{LE\_DLS\_LIST\_INIT};

\textcolor{keywordtype}{void} foo (\textcolor{keywordtype}{void})
\{
    \textcolor{comment}{// Create the node.  Get the memory from a memory pool previously created.}
    MyNodeClass\_t* myNodePtr = \hyperlink{le__mem_8h_af7c289c73d4182835a26a9099f3db359}{le\_mem\_ForceAlloc}(MyNodePool);

    \textcolor{comment}{// Initialize the node's link.}
    myNodePtr->myLink = \hyperlink{le__doubly_linked_list_8h_a616b17e10af5ce6dcc49e34b6ab927c2}{LE\_DLS\_LINK\_INIT};

    \textcolor{comment}{// Add the node to the head of the list by passing in the node's link.}
    \hyperlink{le__doubly_linked_list_8h_a90f9072a55ef0cb573bbdad91e34d368}{le\_dls\_Stack}(&MyList, &(myNodePtr->myLink));
\}
\end{DoxyCode}


The links in the nodes are actually added to the list and not the nodes themselves. This allows a node to be included on multiple lists through links added to different lists. It also allows linking different type nodes in a list.

To obtain the node itself, use the {\ttfamily C\+O\+N\+T\+A\+I\+N\+E\+R\+\_\+\+O\+F} macro defined in \hyperlink{le__basics_8h}{le\+\_\+basics.\+h}. Here\textquotesingle{}s a code sample using C\+O\+N\+T\+A\+I\+N\+E\+R\+\_\+\+O\+F to obtain the node\+:


\begin{DoxyCode}
\textcolor{comment}{// Assuming mylist has been created and initialized and is not empty.}
le\_dls\_link\_t* linkPtr = \hyperlink{le__doubly_linked_list_8h_ab0a2a83f476727f6aa875e98b213f05c}{le\_dls\_Peek}(&MyList);

\textcolor{comment}{// Now we have the link but still need the node to access user data.}
\textcolor{comment}{// We use CONTAINER\_OF to get a pointer to the node given the node's link.}
\textcolor{keywordflow}{if} (linkPtr != NULL)
\{
    MyNodeClass\_t* myNodePtr = \hyperlink{le__basics_8h_a3616d3fd5b502150b643ddc769f71188}{CONTAINER\_OF}(linkPtr, MyNodeClass\_t, myLink);
\}
\end{DoxyCode}


The user is responsible for creating and freeing memory for all nodes; the linked list module only manages the links in the nodes. The node must be removed from all lists before its memory can be freed.

{\bfseries The elements of {\ttfamily \hyperlink{structle__dls___link__t}{le\+\_\+dls\+\_\+\+Link\+\_\+t}} M\+U\+S\+T N\+O\+T be accessed directly by the user.}\hypertarget{c_doubly_linked_list_dls_add}{}\subsection{Adding Links to a List}\label{c_doubly_linked_list_dls_add}
To add nodes to a list, pass the node\textquotesingle{}s link to one of these functions\+:


\begin{DoxyItemize}
\item {\ttfamily \hyperlink{le__doubly_linked_list_8h_a90f9072a55ef0cb573bbdad91e34d368}{le\+\_\+dls\+\_\+\+Stack()}} -\/ Adds the link to the head of the list.
\item {\ttfamily \hyperlink{le__doubly_linked_list_8h_a264df63b847a9c485df0bf9050ac5deb}{le\+\_\+dls\+\_\+\+Queue()}} -\/ Adds the link to the tail of the list.
\item {\ttfamily \hyperlink{le__doubly_linked_list_8h_ad93394ff686d2fe93f5f4ce73c7034cd}{le\+\_\+dls\+\_\+\+Add\+After()}} -\/ Adds the link to a list after another specified link.
\item {\ttfamily \hyperlink{le__doubly_linked_list_8h_a6b68837b42fc2c68885db0857e7c71bf}{le\+\_\+dls\+\_\+\+Add\+Before()}} -\/ Adds the link to a list before another specified link.
\end{DoxyItemize}\hypertarget{c_doubly_linked_list_dls_remove}{}\subsection{Removing Links from a List}\label{c_doubly_linked_list_dls_remove}
To remove nodes from a list, use one of these functions\+:


\begin{DoxyItemize}
\item {\ttfamily \hyperlink{le__doubly_linked_list_8h_a4bd942822ffc97004f46f9d062f62270}{le\+\_\+dls\+\_\+\+Pop()}} -\/ Removes and returns the link at the head of the list.
\item {\ttfamily \hyperlink{le__doubly_linked_list_8h_a31d98b6cfe8de4e618c07bb06e983e81}{le\+\_\+dls\+\_\+\+Pop\+Tail()}} -\/ Removes and returns the link at the tail of the list.
\item {\ttfamily \hyperlink{le__doubly_linked_list_8h_ac5e1d4687e04c4e44359ce697ea9eeb2}{le\+\_\+dls\+\_\+\+Remove()}} -\/ Remove a specified link from the list.
\end{DoxyItemize}\hypertarget{c_doubly_linked_list_dls_peek}{}\subsection{Accessing Links in a List}\label{c_doubly_linked_list_dls_peek}
To access a link in a list without removing the link, use one of these functions\+:


\begin{DoxyItemize}
\item {\ttfamily \hyperlink{le__doubly_linked_list_8h_ab0a2a83f476727f6aa875e98b213f05c}{le\+\_\+dls\+\_\+\+Peek()}} -\/ Returns the link at the head of the list without removing it.
\item {\ttfamily \hyperlink{le__doubly_linked_list_8h_a366bc41775b9dfe265e79a9ffecd0a86}{le\+\_\+dls\+\_\+\+Peek\+Tail()}} -\/ Returns the link at the tail of the list without removing it.
\item {\ttfamily \hyperlink{le__doubly_linked_list_8h_a3a1a15d3922ec53770f0fde3c8eef9f1}{le\+\_\+dls\+\_\+\+Peek\+Next()}} -\/ Returns the link next to a specified link without removing it.
\item {\ttfamily \hyperlink{le__doubly_linked_list_8h_ad43e69f235920323d725115cb166de34}{le\+\_\+dls\+\_\+\+Peek\+Prev()}} -\/ Returns the link previous to a specified link without removing it.
\end{DoxyItemize}\hypertarget{c_doubly_linked_list_dls_swap}{}\subsection{Swapping Links}\label{c_doubly_linked_list_dls_swap}
To swap two links, use\+:


\begin{DoxyItemize}
\item {\ttfamily \hyperlink{le__doubly_linked_list_8h_ad317fec42c2474b8bef3654c89f3d239}{le\+\_\+dls\+\_\+\+Swap()}} -\/ Swaps the position of two links in a list.
\end{DoxyItemize}

The \hyperlink{le__doubly_linked_list_8h_ad317fec42c2474b8bef3654c89f3d239}{le\+\_\+dls\+\_\+\+Swap()} function can be used to sort a list.\hypertarget{c_doubly_linked_list_dls_query}{}\subsection{Querying List Status}\label{c_doubly_linked_list_dls_query}
These functions can be used to query a list\textquotesingle{}s current status\+:


\begin{DoxyItemize}
\item {\ttfamily \hyperlink{le__doubly_linked_list_8h_ab6068e41fca76311c0eeab36d9e23504}{le\+\_\+dls\+\_\+\+Is\+Empty()}} -\/ Checks if a given list is empty.
\item {\ttfamily \hyperlink{le__doubly_linked_list_8h_a13dd41bc5ca2c0b787bca4f57486f600}{le\+\_\+dls\+\_\+\+Is\+In\+List()}} -\/ Checks if a specified link is in the list.
\item {\ttfamily \hyperlink{le__doubly_linked_list_8h_a207e3dc720d0121f2e62eb639aea8d24}{le\+\_\+dls\+\_\+\+Num\+Links()}} -\/ Checks the number of links currently in the list.
\item {\ttfamily \hyperlink{le__doubly_linked_list_8h_a38538339f5eeb2f0c7205fc45a2a3f55}{le\+\_\+dls\+\_\+\+Is\+List\+Corrupted()}} -\/ Checks if the list is corrupted.
\end{DoxyItemize}\hypertarget{c_doubly_linked_list_dls_fifo}{}\subsection{Queues and Stacks}\label{c_doubly_linked_list_dls_fifo}
This implementation of linked lists can be used for either queues or stacks.

To use the list as a queue, restrict additions to the list to {\ttfamily \hyperlink{le__doubly_linked_list_8h_a264df63b847a9c485df0bf9050ac5deb}{le\+\_\+dls\+\_\+\+Queue()}} and removals from the list to {\ttfamily \hyperlink{le__doubly_linked_list_8h_a4bd942822ffc97004f46f9d062f62270}{le\+\_\+dls\+\_\+\+Pop()}}.

To use the list as a stack, restrict additions to the list to {\ttfamily \hyperlink{le__doubly_linked_list_8h_a90f9072a55ef0cb573bbdad91e34d368}{le\+\_\+dls\+\_\+\+Stack()}} and removals from the list to {\ttfamily \hyperlink{le__doubly_linked_list_8h_a4bd942822ffc97004f46f9d062f62270}{le\+\_\+dls\+\_\+\+Pop()}}.\hypertarget{c_doubly_linked_list_dls_synch}{}\subsection{Thread Safety and Re-\/\+Entrancy}\label{c_doubly_linked_list_dls_synch}
All linked list function calls are re-\/entrant and thread safe. If the nodes and/or list object is shared by multiple threads, explicit steps must be taken to maintain mutual exclusion of access.





Copyright (C) Sierra Wireless Inc. Use of this work is subject to license. \hypertarget{c_memory}{}\section{Dynamic Memory Allocation A\+P\+I}\label{c_memory}
\hyperlink{le__mem_8h}{A\+P\+I Reference}





Dynamic memory allocation (especially deallocation) using the C runtime heap, through malloc, free, strdup, calloc, realloc, etc. can result in performance degradation and out-\/of-\/memory conditions.

This is due to fragmentation of the heap. The degraded performance and exhausted memory result from indirect interactions within the heap between unrelated application code. These issues are non-\/deterministic, and can be very difficult to rectify.

Memory Pools offer a powerful solution. They trade-\/off a deterministic amount of memory for
\begin{DoxyItemize}
\item deterministic behaviour,
\item O(1) allocation and release performance, and
\item built-\/in memory allocation tracking.
\end{DoxyItemize}

And it brings the power of {\bfseries destructors} to C!\hypertarget{c_memory_mem_overview}{}\subsection{Overview}\label{c_memory_mem_overview}
The most basic usage involves\+:
\begin{DoxyItemize}
\item Creating a pool (usually done once at process start-\/up)
\item Allocating objects (memory blocks) from a pool
\item Releasing objects back to their pool.
\end{DoxyItemize}

Pools generally can\textquotesingle{}t be deleted. You create them when your process starts-\/up, and use them until your process terminates. It\textquotesingle{}s up to the O\+S to clean-\/up the memory pools, along with everything else your process is using, when your process terminates. (Although, if you find yourself really needing to delete pools, \hyperlink{c_memory_mem_sub_pools}{Sub-\/\+Pools} could offer you a solution.)

Pools also support the following advanced features\+:
\begin{DoxyItemize}
\item reference counting
\item destructors
\item statistics
\item multi-\/threading
\item sub-\/pools (pools that can be deleted).
\end{DoxyItemize}

The following sections describe these, beginning with the most basic usage and working up to more advanced topics.\hypertarget{c_memory_mem_creating}{}\subsection{Creating a Pool}\label{c_memory_mem_creating}
Before allocating memory from a pool, the pool must be created using \hyperlink{le__mem_8h_ab91efaa2978c9c1c7b2427d25b33241c}{le\+\_\+mem\+\_\+\+Create\+Pool()}, passing it the name of the pool and the size of the objects to be allocated from that pool. This returns a reference to the new pool, which has zero free objects in it.

To populate your new pool with free objects, you call {\ttfamily \hyperlink{le__mem_8h_a79a4321ffa0345f267eaf3b7d3d3528a}{le\+\_\+mem\+\_\+\+Expand\+Pool()}}. This is separated into two functions (rather than having one function with three parameters) to make it virtually impossible to accidentally get the parameters in the wrong order (which would result in nasty bugs that couldn\textquotesingle{}t be caught by the compiler). The ability to expand pools comes in handy (see \hyperlink{c_memory_mem_pool_sizes}{Managing Pool Sizes}).

This code sample defines a class \char`\"{}\+Point\char`\"{} and a pool \char`\"{}\+Point\+Pool\char`\"{} used to allocate memory for objects of that class\+: 
\begin{DoxyCode}
\textcolor{preprocessor}{#define MAX\_POINTS 12  // Maximum number of points that can be handled.}

\textcolor{keyword}{typedef} \textcolor{keyword}{struct}
\{
    \textcolor{keywordtype}{int} x;  \textcolor{comment}{// pixel position along x-axis}
    \textcolor{keywordtype}{int} y;  \textcolor{comment}{// pixel position along y-axis}
\}
Point\_t;

le\_mem\_PoolRef\_t PointPool;

\textcolor{keywordtype}{int} xx\_pt\_ProcessStart(\textcolor{keywordtype}{void})
\{
    PointPool = \hyperlink{le__mem_8h_ab91efaa2978c9c1c7b2427d25b33241c}{le\_mem\_CreatePool}(\textcolor{stringliteral}{"xx.pt.Points"}, \textcolor{keyword}{sizeof}(Point\_t));
    \hyperlink{le__mem_8h_a79a4321ffa0345f267eaf3b7d3d3528a}{le\_mem\_ExpandPool}(PointPool, MAX\_POINTS);

    \textcolor{keywordflow}{return} SUCCESS;
\}
\end{DoxyCode}


To make things easier for power-\/users, {\ttfamily \hyperlink{le__mem_8h_a79a4321ffa0345f267eaf3b7d3d3528a}{le\+\_\+mem\+\_\+\+Expand\+Pool()}} returns the same pool reference that it was given. This allows the xx\+\_\+pt\+\_\+\+Process\+Start() function to be re-\/implemented as follows\+: 
\begin{DoxyCode}
\textcolor{keywordtype}{int} xx\_pt\_ProcessStart(\textcolor{keywordtype}{void})
\{
    PointPool = \hyperlink{le__mem_8h_a79a4321ffa0345f267eaf3b7d3d3528a}{le\_mem\_ExpandPool}(\hyperlink{le__mem_8h_ab91efaa2978c9c1c7b2427d25b33241c}{le\_mem\_CreatePool}(\textcolor{keyword}{sizeof}(Point\_t)), 
      MAX\_POINTS);

    \textcolor{keywordflow}{return} SUCCESS;
\}
\end{DoxyCode}


Although this requires a dozen or so fewer keystrokes of typing and occupies one less line of code, it\textquotesingle{}s arguably less readable than the previous example.

For a discussion on how to pick the number of objects to have in your pools, see \hyperlink{c_memory_mem_pool_sizes}{Managing Pool Sizes}.\hypertarget{c_memory_mem_allocating}{}\subsection{Allocating From a Pool}\label{c_memory_mem_allocating}
Allocating from a pool has multiple options\+:
\begin{DoxyItemize}
\item {\ttfamily \hyperlink{le__mem_8h_a742e4f9d621ca27493733ca781bbe187}{le\+\_\+mem\+\_\+\+Try\+Alloc()}} -\/ Quietly return N\+U\+L\+L if there are no free blocks in the pool.
\item {\ttfamily \hyperlink{le__mem_8h_a2993bf7a9705d119c96cf80cd64a56bb}{le\+\_\+mem\+\_\+\+Assert\+Alloc()}} -\/ Log an error and take down the process if there are no free blocks in the pool.
\item {\ttfamily \hyperlink{le__mem_8h_af7c289c73d4182835a26a9099f3db359}{le\+\_\+mem\+\_\+\+Force\+Alloc()}} -\/ If there are no free blocks in the pool, log a warning and automatically expand the pool (or log an error and terminate the calling process there\textquotesingle{}s not enough free memory to expand the pool).
\end{DoxyItemize}

All of these functions take a pool reference and return a pointer to the object allocated from the pool.

The first option, using {\ttfamily \hyperlink{le__mem_8h_a742e4f9d621ca27493733ca781bbe187}{le\+\_\+mem\+\_\+\+Try\+Alloc()}}, is the closest to the way good old malloc() works. It requires the caller check the return code to see if it\textquotesingle{}s N\+U\+L\+L. This can be annoying enough that a lot of people get lazy and don\textquotesingle{}t check the return code (Bad programmer! Bad!). It turns out that this option isn\textquotesingle{}t really what people usually want (but occasionally they do)

The second option, using {\ttfamily \hyperlink{le__mem_8h_a2993bf7a9705d119c96cf80cd64a56bb}{le\+\_\+mem\+\_\+\+Assert\+Alloc()}}, is only used when the allocation should never fail, by design; a failure to allocate a block is a fatal error. This isn\textquotesingle{}t often used, but can save a lot of boilerplate error checking code.

The third option, {\ttfamily using} \hyperlink{le__mem_8h_af7c289c73d4182835a26a9099f3db359}{le\+\_\+mem\+\_\+\+Force\+Alloc()}, is the one that gets used most often. It allows developers to avoid writing error checking code, because the allocation will essentially never fail because it\textquotesingle{}s handled inside the memory allocator. It also allows developers to defer fine tuning their pool sizes until after they get things working. Later, they check the logs for pool size usage, and then modify their pool sizes accordingly. If a particular pool is continually growing, it\textquotesingle{}s a good indication there\textquotesingle{}s a memory leak. This permits seeing exactly what objects are being leaked. If certain debug options are turned on, they can even find out which line in which file allocated the blocks being leaked.\hypertarget{c_memory_mem_releasing}{}\subsection{Releasing Back Into a Pool}\label{c_memory_mem_releasing}
Releasing memory back to a pool never fails, so there\textquotesingle{}s no need to check a return code. Also, each object knows which pool it came from, so the code that releases the object doesn\textquotesingle{}t have to care. All it has to do is call {\ttfamily \hyperlink{le__mem_8h_a6d8e3fe430bcb81efe97b57ce30ef2de}{le\+\_\+mem\+\_\+\+Release()}} and pass a pointer to the object to be released.

The critical thing to remember is that once an object has been released, it {\bfseries  must never be accessed again }. Here is a {\bfseries  very bad code example}\+: 
\begin{DoxyCode}
Point\_t* pointPtr = \hyperlink{le__mem_8h_af7c289c73d4182835a26a9099f3db359}{le\_mem\_ForceAlloc}(PointPool);
pointPtr->x = 5;
pointPtr->y = 10;
\hyperlink{le__mem_8h_a6d8e3fe430bcb81efe97b57ce30ef2de}{le\_mem\_Release}(pointPtr);
printf(\textcolor{stringliteral}{"Point is at position (%d, %d).\(\backslash\)n"}, pointPtr->x, pointPtr->y);
\end{DoxyCode}
\hypertarget{c_memory_mem_ref_counting}{}\subsection{Reference Counting}\label{c_memory_mem_ref_counting}
Reference counting is a powerful feature of our memory pools. Here\textquotesingle{}s how it works\+:
\begin{DoxyItemize}
\item Every object allocated from a pool starts with a reference count of 1.
\item Whenever someone calls \hyperlink{le__mem_8h_a92e869f92a344d61fb44922f99fe679b}{le\+\_\+mem\+\_\+\+Add\+Ref()} on an object, its reference count is incremented by 1.
\item When it\textquotesingle{}s released, its reference count is decremented by 1.
\item When its reference count reaches zero, it\textquotesingle{}s destroyed (i.\+e., its memory is released back into the pool.)
\end{DoxyItemize}

This allows one function to\+:
\begin{DoxyItemize}
\item create an object.
\item work with it.
\item increment its reference count and pass a pointer to the object to another function (or thread, data structure, etc.).
\item work with it some more.
\item release the object without having to worry about when the other function is finished with it.
\end{DoxyItemize}

The other function also releases the object when it\textquotesingle{}s done with it. So, the object will exist until both functions are done.

If there are multiple threads involved, be careful to protect the shared object from race conditions(see the \hyperlink{c_memory_mem_threading}{Multi-\/\+Threading}).

Another great advantage of reference counting is it enables \hyperlink{c_memory_mem_destructors}{Destructors}.\hypertarget{c_memory_mem_destructors}{}\subsection{Destructors}\label{c_memory_mem_destructors}
Destructors are a powerful feature of C++. Anyone who has any non-\/trivial experience with C++ has used them. Because C was created before object-\/oriented programming was around, there\textquotesingle{}s no native language support for destructors in C. Object-\/oriented design is still possible and highly desireable even when the programming is done in C.

In Legato, it\textquotesingle{}s possible to call {\ttfamily \hyperlink{le__mem_8h_a055007b38ce04bcb823e08034fd11b85}{le\+\_\+mem\+\_\+\+Set\+Destructor()}} to attach a function to a memory pool to be used as a destructor for objects allocated from that pool. If a pool has a destructor, whenever the reference count reaches zero for an object allocated from that pool, the pool\textquotesingle{}s destructor function will pass a pointer to that object. After the destructor returns, the object will be fully destroyed, and its memory will be released back into the pool for later reuse by another object.

Here\textquotesingle{}s a destructor code sample\+: 
\begin{DoxyCode}
\textcolor{keyword}{static} \textcolor{keywordtype}{void} PointDestructor(\textcolor{keywordtype}{void}* objPtr)
\{
    Point\_t* pointPtr = objPtr;

    printf(\textcolor{stringliteral}{"Destroying point (%d, %d)\(\backslash\)n"}, pointPtr->x, pointPtr->y);

    @todo Add more to sample.
\}

\textcolor{keywordtype}{int} xx\_pt\_ProcessStart(\textcolor{keywordtype}{void})
\{
    PointPool = \hyperlink{le__mem_8h_ab91efaa2978c9c1c7b2427d25b33241c}{le\_mem\_CreatePool}(\textcolor{keyword}{sizeof}(Point\_t));
    \hyperlink{le__mem_8h_a79a4321ffa0345f267eaf3b7d3d3528a}{le\_mem\_ExpandPool}(PointPool, MAX\_POINTS);
    \hyperlink{le__mem_8h_a055007b38ce04bcb823e08034fd11b85}{le\_mem\_SetDestructor}(PointPool, PointDestructor);
    \textcolor{keywordflow}{return} SUCCESS;
\}

\textcolor{keyword}{static} \textcolor{keywordtype}{void} DeletePointList(Point\_t** pointList, \textcolor{keywordtype}{size\_t} numPoints)
\{
    \textcolor{keywordtype}{size\_t} i;
    \textcolor{keywordflow}{for} (i = 0; i < numPoints; i++)
    \{
        \hyperlink{le__mem_8h_a6d8e3fe430bcb81efe97b57ce30ef2de}{le\_mem\_Release}(pointList[i]);
    \}
\}
\end{DoxyCode}


In this sample, when Delete\+Point\+List() is called (with a pointer to an array of pointers to Point\+\_\+t objects with reference counts of 1), each of the objects in the point\+List is released. This causes their reference counts to hit 0, which triggers executing Point\+Destructor() for each object in the point\+List, and the \char`\"{}\+Destroying point...\char`\"{} message will be printed for each.\hypertarget{c_memory_mem_stats}{}\subsection{Statistics}\label{c_memory_mem_stats}
Some statistics are gathered for each memory pool\+:
\begin{DoxyItemize}
\item Number of allocations.
\item Number of currently free objects.
\item Number of overflows (times that \hyperlink{le__mem_8h_af7c289c73d4182835a26a9099f3db359}{le\+\_\+mem\+\_\+\+Force\+Alloc()} had to expand the pool).
\end{DoxyItemize}

Statistics (and other pool properties) can be checked using functions\+:
\begin{DoxyItemize}
\item {\ttfamily \hyperlink{le__mem_8h_ab7b41431c57c8c7b5c4ff1501fd5b772}{le\+\_\+mem\+\_\+\+Get\+Stats()}} 
\item {\ttfamily \hyperlink{le__mem_8h_a76725588ed757ca95cdf36e5ab3aeebf}{le\+\_\+mem\+\_\+\+Get\+Object\+Count()}} 
\item {\ttfamily \hyperlink{le__mem_8h_a0f6fbc0c886486a1e19fc43143991c66}{le\+\_\+mem\+\_\+\+Get\+Object\+Size()}} 
\end{DoxyItemize}

Statistics are fetched together atomically using a single function call. This prevents inconsistencies between them if in a multi-\/threaded program.

If you don\textquotesingle{}t have a reference to a specified pool, but you have the name of the pool, you can get a reference to the pool using {\ttfamily \hyperlink{le__mem_8h_a67e004702344963aea788b1c0ca70862}{le\+\_\+mem\+\_\+\+Find\+Pool()}}.

In addition to programmatically fetching these, they\textquotesingle{}re also available through the \char`\"{}poolstat\char`\"{} console command (unless your process\textquotesingle{}s main thread is blocked).

To reset the pool statistics, use {\ttfamily \hyperlink{le__mem_8h_a35b7e757356764c39f0a7ede2aa242ae}{le\+\_\+mem\+\_\+\+Reset\+Stats()}}.\hypertarget{c_memory_mem_diagnostics}{}\subsection{Diagnostics}\label{c_memory_mem_diagnostics}
The memory system also supports two different forms of diagnostics. Both are enabled by defining special preprocessor macros when building the framework.

The first of which is {\ttfamily L\+E\+\_\+\+M\+E\+M\+\_\+\+T\+R\+A\+C\+E}. When you define {\ttfamily L\+E\+\_\+\+M\+E\+M\+\_\+\+T\+R\+A\+C\+E} every pool is given a tracepoint with the name of the pool on creation.

For instance, the config\+Tree node pool is called, \char`\"{}config\+Tree.\+node\+Pool\char`\"{}. So to enable a trace of all config tree node creation and deletion one would use the log tool as follows\+:


\begin{DoxyCode}
$ log trace configTree.nodePool
\end{DoxyCode}


The second diagnostic build flag is {\ttfamily L\+E\+\_\+\+M\+E\+M\+\_\+\+V\+A\+L\+G\+R\+I\+N\+D}. When {\ttfamily L\+E\+\_\+\+M\+E\+M\+\_\+\+V\+A\+L\+G\+R\+I\+N\+D} is enabled, the pools are disabled and instead malloc and free are directly used. Thus enabling the use of tools like Valgrind.\hypertarget{c_memory_mem_threading}{}\subsection{Multi-\/\+Threading}\label{c_memory_mem_threading}
All functions in this A\+P\+I are {\bfseries  thread-\/safe, but not async-\/safe }. The objects allocated from pools are not inherently protected from races between threads.

Allocating and releasing objects, checking stats, incrementing reference counts, etc. can all be done from multiple threads (excluding signal handlers) without having to worry about corrupting the memory pools\textquotesingle{} hidden internal data structures.

There\textquotesingle{}s no magical way to prevent different threads from interferring with each other if they both access the {\itshape contents} of the same object at the same time.

The best way to prevent multi-\/threaded race conditions is simply don\textquotesingle{}t share data between threads. If multiple threads must access the same data structure, then mutexes, semaphores, or similar methods should be used to {\itshape synchronize} threads and avoid data structure corruption or thread misbehaviour.

Although memory pools are {\itshape thread-\/safe}, they are not {\itshape async-\/safe}. This means that memory pools {\itshape can} be corrupted if they are accessed by a signal handler while they are being accessed by a normal thread. To be safe, {\bfseries  don\textquotesingle{}t call any memory pool functions from within a signal handler. }

One problem using destructor functions in a multi-\/threaded environment is that the destructor function modifies a data structure shared between threads, so it\textquotesingle{}s easy to forget to synchronize calls to {\ttfamily \hyperlink{le__mem_8h_a6d8e3fe430bcb81efe97b57ce30ef2de}{le\+\_\+mem\+\_\+\+Release()}} with other code accessing the data structure. If a mutex is used to coordinate access to the data structure, then the mutex must be held by the thread that calls \hyperlink{le__mem_8h_a6d8e3fe430bcb81efe97b57ce30ef2de}{le\+\_\+mem\+\_\+\+Release()} to ensure there\textquotesingle{}s no other thread accessing the data structure when the destructor runs.\hypertarget{c_memory_mem_pool_sizes}{}\subsection{Managing Pool Sizes}\label{c_memory_mem_pool_sizes}
We know it\textquotesingle{}s possible to have pools automatically expand when they are exhausted, but we don\textquotesingle{}t really want that to happen normally. Ideally, the pools should be fully allocated to their maximum sizes at start-\/up so there aren\textquotesingle{}t any surprises later when certain feature combinations cause the system to run out of memory in the field. If we allocate everything we think is needed up-\/front, then we are much more likely to uncover any memory shortages during testing, before it\textquotesingle{}s in the field.

Choosing the right size for your pools correctly at start-\/up is easy to do if there is a maximum number of fixed, external {\itshape things} that are being represented by the objects being allocated from the pool. If the pool holds \char`\"{}call objects\char`\"{} representing phone calls over a T1 carrier that will never carry more than 24 calls at a time, then it\textquotesingle{}s obvious that you need to size your call object pool at 24.

Other times, it\textquotesingle{}s not so easy to choose the pool size like code to be reused in different products or different configurations that have different needs. In those cases, you still have a few options\+:


\begin{DoxyItemize}
\item At start-\/up, query the operating environment and base the pool sizes.
\item Read a configuration setting from a file or other configuration data source.
\item Use a build-\/time configuration setting.
\end{DoxyItemize}

The build-\/time configuration setting is the easiest, and generally requires less interaction between components at start-\/up simplifying A\+P\+Is and reducing boot times.

If the pool size must be determined at start-\/up, use {\ttfamily \hyperlink{le__mem_8h_a79a4321ffa0345f267eaf3b7d3d3528a}{le\+\_\+mem\+\_\+\+Expand\+Pool()}}. Perhaps there\textquotesingle{}s a service-\/provider module designed to allocate objects on behalf of client. It can have multiple clients at the same time, but it doesn\textquotesingle{}t know how many clients or what their resource needs will be until the clients register with it at start-\/up. We\textquotesingle{}d want those clients to be as decoupled from each other as possible (i.\+e., we want the clients know as little as possible about each other); we don\textquotesingle{}t want the clients to get together and add up all their needs before telling the service-\/provider. We\textquotesingle{}d rather have the clients independently report their own needs to the service-\/provider. Also, we don\textquotesingle{}t want each client to have to wait for all the other clients to report their needs before starting to use the services offered by the service-\/provider. That would add more complexity to the interactions between the clients and the service-\/provider.

This is what should happen when the service-\/provider can\textquotesingle{}t wait for all clients to report their needs before creating the pool\+:
\begin{DoxyItemize}
\item When the service-\/provider starts up, it creates an empty pool.
\item Whenever a client registers itself with the service-\/provider, the client can tell the service-\/provider what its specific needs are, and the service-\/provider can expand its object pool accordingly.
\item Since registrations happen at start-\/up, pool expansion occurs at start-\/up, and testing will likely find any pool sizing before going into the field.
\end{DoxyItemize}

Where clients dynamically start and stop during runtime in response to external events (e.\+g., when someone is using the device\textquotesingle{}s Web U\+I), we still have a problem because we can\textquotesingle{}t {\itshape shrink} pools or delete pools when clients go away. This is where \hyperlink{c_memory_mem_sub_pools}{Sub-\/\+Pools} is useful.\hypertarget{c_memory_mem_sub_pools}{}\subsection{Sub-\/\+Pools}\label{c_memory_mem_sub_pools}
Essentially, a Sub-\/\+Pool is a memory pool that gets its blocks from another pool (the super-\/pool). Sub Pools {\itshape can} be deleted, causing its blocks to be released back into the super-\/pool.

This is useful when a service-\/provider module needs to handle clients that dynamically pop into existence and later disappear again. When a client attaches to the service and says it will probably need a maximum of X of the service-\/provider\textquotesingle{}s resources, the service provider can set aside that many of those resources in a sub-\/pool for that client. If that client goes over its limit, the sub-\/pool will log a warning message.

The problem of sizing the super-\/pool correctly at start-\/up still exists, so what\textquotesingle{}s the point of having a sub-\/pool, when all of the resources could just be allocated from the super-\/pool?

The benefit is really gained in troubleshooting. If client A, B, C, D and E are all behaving nicely, but client F is leaking resources, the sub-\/pool created on behalf of client F will start warning about the memory leak; time won\textquotesingle{}t have to be wasted looking at clients A through E to rule them out.

To create a sub-\/pool, call {\ttfamily \hyperlink{le__mem_8h_a8b043fcb013deb4c58c90ca2e0ab9d16}{le\+\_\+mem\+\_\+\+Create\+Sub\+Pool()}}. It takes a reference to the super-\/pool and the objects specified to the sub-\/pool, and it returns a reference to the new sub-\/pool.

To delete a sub-\/pool, call {\ttfamily \hyperlink{le__mem_8h_aa1d51a37f572c2d891cdfb625ea19f27}{le\+\_\+mem\+\_\+\+Delete\+Sub\+Pool()}}. Do not try to use it to delete a pool that was created using \hyperlink{le__mem_8h_ab91efaa2978c9c1c7b2427d25b33241c}{le\+\_\+mem\+\_\+\+Create\+Pool()}. It\textquotesingle{}s only for sub-\/pools created using \hyperlink{le__mem_8h_a8b043fcb013deb4c58c90ca2e0ab9d16}{le\+\_\+mem\+\_\+\+Create\+Sub\+Pool()}. Also, it\textquotesingle{}s {\bfseries not} okay to delete a sub-\/pool while there are still blocks allocated from it. You\textquotesingle{}ll see errors in your logs if you do that.

Sub-\/\+Pools automatically inherit their parent\textquotesingle{}s destructor function.

\begin{DoxyNote}{Note}
You can\textquotesingle{}t create sub-\/pools of sub-\/pools (i.\+e., sub-\/pools that get their blocks from another sub-\/pool).
\end{DoxyNote}




Copyright (C) Sierra Wireless Inc. Use of this work is subject to license. \hypertarget{c_eventLoop}{}\section{Event Loop A\+P\+I}\label{c_eventLoop}
\hyperlink{le__event_loop_8h}{A\+P\+I Reference}





The Event Loop A\+P\+I supports the event-\/driven programming model, which is favoured in Legato (but not forced). Each thread that uses this system has a central {\bfseries event loop} which calls {\bfseries event handler} functions in response to {\bfseries event reports}.

Software components register their event handler functions with the event system (either directly through the Event Loop A\+P\+I or indirectly through other A\+P\+Is that use the Event Loop A\+P\+I) so the central event loop knows the functions to call in response to defined events.

Every event loop has an {\bfseries event queue}, which is a queue of events waiting to be handled by that event loop.

\begin{DoxyNote}{Note}
When the process dies, all events, event loops, queues, reports, and handlers will be automatically cleared.
\end{DoxyNote}
The following different usage patterns are supported by the Event Loop A\+P\+I\+:

\hyperlink{c_event_loop_c_event_deferredFunctionCalls}{Deferred Function Calls} ~\newline
 \hyperlink{c_event_loop_c_event_dispatchingToOtherThreads}{Dispatching Function Execution to Other Threads} ~\newline
 \hyperlink{c_event_loop_c_event_publishSubscribe}{Publish-\/\+Subscribe Events} ~\newline
 \hyperlink{c_event_loop_c_event_layeredPublishSubscribe}{Layered Publish-\/\+Subscribe Handlers} ~\newline


Other Legato C Runtime Library A\+P\+Is using the event loop include\+:

\hyperlink{c_fdMonitor}{File Descriptor Monitor A\+P\+I} ~\newline
 \hyperlink{c_timer}{Timer A\+P\+I} ~\newline
 \hyperlink{c_args}{Command Line Arguments A\+P\+I} ~\newline
 \hyperlink{c_signals}{Signals A\+P\+I} ~\newline
 \hyperlink{c_messaging}{Low-\/\+Level Messaging A\+P\+I} ~\newline
\hypertarget{c_event_loop_c_event_deferredFunctionCalls}{}\subsection{Deferred Function Calls}\label{c_event_loop_c_event_deferredFunctionCalls}
A basic Event Queue usage is to queue a function for the Event Loop to call later (when that function gets to the head of the Event Queue) by calling l{\ttfamily e\+\_\+event\+\_\+\+Queue\+Function()}.

This code sample has a component initialization function queueing another function to be call later, by the process\textquotesingle{}s main thread when the Event Loop is running. Two parameters are needed by the deferred function. The third is just filled with N\+U\+L\+L and ignored by the deferred function.


\begin{DoxyCode}
\textcolor{keyword}{static} \textcolor{keywordtype}{void} MyDeferredFunction
(
    \textcolor{keywordtype}{void}* param1Ptr,
    \textcolor{keywordtype}{void}* param2Ptr
)
\{
    \textcolor{comment}{// Type cast the parameters to what they really are and do whatever it is that}
    \textcolor{comment}{// I need to do with them.}
\}

...

COMPONENT\_INIT
\{
    \hyperlink{le__event_loop_8h_a6dcc88f96060c5bc107a81a978132f38}{le\_event\_QueueFunction}(MyDeferredFunction, firstParamPtr, secondParamPtr);
\}
\end{DoxyCode}


Deferred function calls are useful when implementing A\+P\+Is with asynchronous result call-\/backs. If an error is detected before the A\+P\+I function returns, it can\textquotesingle{}t just call the call-\/back directly, because it could cause re-\/entrancy problems in the client code or cause recursive loops. Instead of forcing the A\+P\+I function to return an error code in special cases (which will increase the client\textquotesingle{}s code complexity and may leak A\+P\+I implementation details to the client), the A\+P\+I function can defers executing the call-\/back until later by queuing an error handling function onto the Event Queue.\hypertarget{c_event_loop_c_event_dispatchingToOtherThreads}{}\subsection{Dispatching Function Execution to Other Threads}\label{c_event_loop_c_event_dispatchingToOtherThreads}
In multi-\/threaded programs, sometimes the implementor needs to ask another thread to run a function because\+:
\begin{DoxyItemize}
\item The function to be executed takes a long time, but doesn\textquotesingle{}t have to be done at a high priority.
\item A call needs to be made into a non-\/thread-\/safe A\+P\+I function.
\item A blocking function needs to be called, but the current thread can\textquotesingle{}t afford to block.
\end{DoxyItemize}

To assist with this, the Event Loop A\+P\+I provides {\ttfamily \hyperlink{le__event_loop_8h_a228da2d1f53ffa74517f108b0dcfa4d9}{le\+\_\+event\+\_\+\+Queue\+Function\+To\+Thread()}}. It works the same as \hyperlink{le__event_loop_8h_a6dcc88f96060c5bc107a81a978132f38}{le\+\_\+event\+\_\+\+Queue\+Function()}, except that it queues the function onto a specific thread\textquotesingle{}s Event Queue.

If the other thread isn\textquotesingle{}t running the Event Loop, then the queued function will never be executed.

This code sample shows two arguments started by the process\textquotesingle{}s main thread, and executed in the background by a low-\/priority thread. The result is reported back to the client through a completion callback running in the same thread that requested that the computation be performed.


\begin{DoxyCode}
\textcolor{keyword}{static} le\_mem\_PoolRef\_t ComputeRequestPool;
\textcolor{keyword}{static} \hyperlink{le__thread_8h_a32121104c6b4ca39008eb79a4d6862f2}{le\_thread\_Ref\_t} LowPriorityThreadRef;

\textcolor{keyword}{typedef} \textcolor{keyword}{struct}
\{
    \textcolor{keywordtype}{size\_t}           arg1;                                   \textcolor{comment}{// First argument}
    \textcolor{keywordtype}{size\_t}           arg2;                                   \textcolor{comment}{// Second argument}
    ssize\_t          result;                                 \textcolor{comment}{// The result}
    void           (*completionCallback)(ssize\_t result);    \textcolor{comment}{// The client's completion callback}
    \hyperlink{le__thread_8h_a32121104c6b4ca39008eb79a4d6862f2}{le\_thread\_Ref\_t}  requestingThreadRef;                    \textcolor{comment}{// The client's thread.}
\}
ComputeRequest\_t;

\textcolor{comment}{// Main function of low-priority background thread.}
\textcolor{keyword}{static} \textcolor{keywordtype}{void}* LowPriorityThreadMain
(
    \textcolor{keywordtype}{void}* contextPtr \textcolor{comment}{// not used.}
)
\{
    \hyperlink{le__event_loop_8h_ae313b457994371c658be9fe0494a01ff}{le\_event\_RunLoop}();
\}

\hyperlink{le__event_loop_8h_abdb9187a56836a93d19cc793cbd4b7ec}{COMPONENT\_INIT}
\{
    ComputeRequestPool = \hyperlink{le__mem_8h_ab91efaa2978c9c1c7b2427d25b33241c}{le\_mem\_CreatePool}(\textcolor{stringliteral}{"Compute Request"}, \textcolor{keyword}{sizeof}(ComputeRequest\_t));

    LowPriorityThreadRef = \hyperlink{le__thread_8h_a87e02a46f92e9e3e11ed28a2b265872f}{le\_thread\_Create}(\textcolor{stringliteral}{"Background Computation Thread"},
                                            LowPriorityThreadMain,
                                            NULL);
    \hyperlink{le__thread_8h_a95257a2f60cacdadc787647453b77356}{le\_thread\_SetPriority}(LowPriorityThreadRef, 
      \hyperlink{le__thread_8h_a653b0f17cd4d4567c86a25e23d004f07a8237422b4c3d3df0ffcdbc9981c45d98}{LE\_THREAD\_PRIORITY\_IDLE});
    \hyperlink{le__thread_8h_a38df3877ee5ab9fac17b2fc0be46c27e}{le\_thread\_Start}(LowPriorityThreadRef);
\}

\textcolor{comment}{// This function gets run by a low-priority, background thread.}
\textcolor{keyword}{static} \textcolor{keywordtype}{void} ComputeResult
(
    \textcolor{keywordtype}{void}* param1Ptr, \textcolor{comment}{// request object pointer}
    \textcolor{keywordtype}{void}* param2Ptr  \textcolor{comment}{// not used}
)
\{
    ComputeRequest\_t* requestPtr = param1Ptr;

    requestPtr->result = DoSomeReallySlowComputation(requestPtr->arg1, requestPtr->arg2);

    \hyperlink{le__event_loop_8h_a228da2d1f53ffa74517f108b0dcfa4d9}{le\_event\_QueueFunctionToThread}(requestPtr->requestingThreadRef,
                                   ProcessResult,
                                   requestPtr,
                                   NULL);
\}

\textcolor{comment}{// This function gets called by a component running in the main thread.}
\textcolor{keyword}{static} \textcolor{keywordtype}{void} ComputeResultInBackground
(
     \textcolor{keywordtype}{size\_t} arg1,
     \textcolor{keywordtype}{size\_t} arg2,
     \textcolor{keywordtype}{void} (*completionCallback)(ssize\_t result)
)
\{
    ComputeRequest\_t* requestPtr = \hyperlink{le__mem_8h_af7c289c73d4182835a26a9099f3db359}{le\_mem\_ForceAlloc}(ComputeRequestPool);
    requestPtr->arg1 = arg1;
    requestPtr->arg2 = arg2;
    requestPtr->requestingThreadRef = \hyperlink{le__thread_8h_a90a9d67db26f816fd1e1032d74a24fcd}{le\_thread\_GetCurrent}();
    requestPtr->completionCallback = completionCallback;
    \hyperlink{le__event_loop_8h_a228da2d1f53ffa74517f108b0dcfa4d9}{le\_event\_QueueFunctionToThread}(LowPriorityThreadRef,
                                   ComputeResult,
                                   requestPtr,
                                   NULL);
\}

\textcolor{comment}{// This function gets run by the main thread.}
\textcolor{keyword}{static} \textcolor{keywordtype}{void} ProcessResult
(
    \textcolor{keywordtype}{void}* param1Ptr, \textcolor{comment}{// request object pointer}
    \textcolor{keywordtype}{void}* param2Ptr  \textcolor{comment}{// not used}
)
\{
    ComputeRequest\_t* requestPtr = param1Ptr;
    completionCallback(requestPtr->result);
    \hyperlink{le__mem_8h_a6d8e3fe430bcb81efe97b57ce30ef2de}{le\_mem\_Release}(requestPtr);
\}
\end{DoxyCode}
\hypertarget{c_event_loop_c_event_publishSubscribe}{}\subsection{Publish-\/\+Subscribe Events}\label{c_event_loop_c_event_publishSubscribe}
In the publish-\/subscribe pattern, someone publishes information and if anyone cares about that information, they subscribe to receive it. The publisher doesn\textquotesingle{}t have to know whether anything is listening, or how many subscribers might be listening. Likewise, the subscribers don\textquotesingle{}t have to know whether anything is publishing or how many publishers there might be. This decouples publishers and subscribers.

Subscribers {\bfseries add} handlers for events and wait for those handlers to be executed.

Publishers {\bfseries report} events.

When an event report reaches the front of an Event Queue, the Event Loop will pop it from the queue and call any handlers that have been registered for that event.

Events are identified using an {\bfseries  Event I\+D } created by calling {\ttfamily \hyperlink{le__event_loop_8h_a41a96eb3affb07184b519164cf54e213}{le\+\_\+event\+\_\+\+Create\+Id()}} before registering an handler for that event or report. Any thread within the process with an Event I\+D can register a handler or report events.

\begin{DoxyNote}{Note}
These Event I\+Ds are only valid within the process where they were created. The Event Loop A\+P\+I can\textquotesingle{}t be used for inter-\/process communication (I\+P\+C).
\end{DoxyNote}

\begin{DoxyCode}
\hyperlink{le__event_loop_8h_ae6e351b38bc95954f159d16d19d2d55c}{le\_event\_Id\_t} eventId = \hyperlink{le__event_loop_8h_a41a96eb3affb07184b519164cf54e213}{le\_event\_CreateId}(\textcolor{stringliteral}{"MyEvent"}, \textcolor{keyword}{sizeof}(MyEventReport\_t))
      ;
\end{DoxyCode}


Event reports can carry a payload. The size and format of the payload depends on the type of event. For example, reports of temperature changes may need to carry the new temperature. To support this, {\ttfamily \hyperlink{le__event_loop_8h_a41a96eb3affb07184b519164cf54e213}{le\+\_\+event\+\_\+\+Create\+Id()}} takes the payload size as a parameter.

To report an event, the publisher builds their report payload in their own buffer and passes a pointer to that buffer (and its size) to {\ttfamily \hyperlink{le__event_loop_8h_ae3ffe6990b70fb572b4eef06739b4f54}{le\+\_\+event\+\_\+\+Report()}}\+:


\begin{DoxyCode}
MyEventReport\_t report;
...     \textcolor{comment}{// Fill in the event report.}
\hyperlink{le__event_loop_8h_ae3ffe6990b70fb572b4eef06739b4f54}{le\_event\_Report}(EventId, &report, \textcolor{keyword}{sizeof}(report));
\end{DoxyCode}


This results in the report getting queued to the Event Queues of all threads with handlers registered for that event I\+D.

To register a handler, the subscriber calls {\ttfamily \hyperlink{le__event_loop_8h_ae65a65b4111618f47d7e6d57a48289e5}{le\+\_\+event\+\_\+\+Add\+Handler()}}.

\begin{DoxyNote}{Note}
It\textquotesingle{}s okay to have a payload size of zero, in which case N\+U\+L\+L can be passed into \hyperlink{le__event_loop_8h_ae3ffe6990b70fb572b4eef06739b4f54}{le\+\_\+event\+\_\+\+Report()}.
\end{DoxyNote}

\begin{DoxyCode}
\hyperlink{le__event_loop_8h_ae7ab96b8e3441b3d484fcf52aa7a9dad}{le\_event\_HandlerRef\_t} handlerRef = \hyperlink{le__event_loop_8h_ae65a65b4111618f47d7e6d57a48289e5}{le\_event\_AddHandler}(\textcolor{stringliteral}{"MyHandler"},
       eventId, MyHandlerFunc);
\end{DoxyCode}


When an event report reaches the front of a thread\textquotesingle{}s Event Queue, that thread\textquotesingle{}s Event Loop reads the report and then\+:
\begin{DoxyItemize}
\item Calls the handler functions registered by that thread.
\item Points to the report payload passed to the handler as a parameter.
\item Reports the payload was deleted on return, so the handler function must copy any contents to keep.
\end{DoxyItemize}


\begin{DoxyCode}
\textcolor{keyword}{static} \textcolor{keywordtype}{void} MyHandlerFunc
(
    \textcolor{keywordtype}{void}* reportPayloadPtr
)
\{
    MyEventReport\_t* reportPtr = reportPayloadPtr;
    \textcolor{comment}{// Process the report.}
    ...
\}
\end{DoxyCode}


Another opaque pointer, called the {\bfseries  context pointer } can be set for the handler using {\ttfamily \hyperlink{le__event_loop_8h_ae0c4307a9715794c720e525032aa0bfd}{le\+\_\+event\+\_\+\+Set\+Context\+Ptr()}}. When the handler function is called, it can call \hyperlink{le__event_loop_8h_a1c73916295cc9e17af07e02756aa86c9}{le\+\_\+event\+\_\+\+Get\+Context\+Ptr()} to fetch the context pointer.


\begin{DoxyCode}
\textcolor{keyword}{static} \textcolor{keywordtype}{void} MyHandlerFunc
(
    \textcolor{keywordtype}{void}* reportPayloadPtr
)
\{
    MyEventReport\_t* reportPtr = reportPayloadPtr;
    MyContext\_t* contextPtr = \hyperlink{le__event_loop_8h_a1c73916295cc9e17af07e02756aa86c9}{le\_event\_GetContextPtr}();

    \textcolor{comment}{// Process the report.}
    ...
\}

\hyperlink{le__event_loop_8h_abdb9187a56836a93d19cc793cbd4b7ec}{COMPONENT\_INIT}
\{
    MyEventId = \hyperlink{le__event_loop_8h_a41a96eb3affb07184b519164cf54e213}{le\_event\_CreateId}(\textcolor{stringliteral}{"MyEvent"}, \textcolor{keyword}{sizeof}(MyEventReport\_t));

    MyHandlerRef = \hyperlink{le__event_loop_8h_ae65a65b4111618f47d7e6d57a48289e5}{le\_event\_AddHandler}(\textcolor{stringliteral}{"MyHandler"}, MyEventId, MyHandlerFunc);
    \hyperlink{le__event_loop_8h_ae0c4307a9715794c720e525032aa0bfd}{le\_event\_SetContextPtr}(MyHandlerRef, \textcolor{keyword}{sizeof}(\textcolor{keywordtype}{float}));
\}
\end{DoxyCode}


Finally, \hyperlink{le__event_loop_8h_ae31a85d4acbef72451b5411a613eea58}{le\+\_\+event\+\_\+\+Remove\+Handler()} can be used to remove an event handler registration, if necessary.


\begin{DoxyCode}
\hyperlink{le__event_loop_8h_ae31a85d4acbef72451b5411a613eea58}{le\_event\_RemoveHandler}(MyHandlerRef);
\end{DoxyCode}


If a handler is removed after the report for that event has been added to the event queue, but before the report reaches the head of the queue, then the handler will not be called.

\begin{DoxyNote}{Note}
To prevent race conditions, it\textquotesingle{}s not permitted for one thread to remove another thread\textquotesingle{}s handlers.
\end{DoxyNote}
\hypertarget{c_event_loop_c_event_layeredPublishSubscribe}{}\subsection{Layered Publish-\/\+Subscribe Handlers}\label{c_event_loop_c_event_layeredPublishSubscribe}
If you need to implement an A\+P\+I that allows clients to register \char`\"{}handler\char`\"{} functions to be called-\/back after a specific event occurs, the Event Loop A\+P\+I provides some special help.

You can have the Event Loop call your handler function (the first-\/layer handler), to unpack specified items from the Event Report and call the client\textquotesingle{}s handler function (the second-\/layer handler).

For example, you could create a \char`\"{}\+Temperature Sensor A\+P\+I\char`\"{} that allows its clients to register handler functions to be called to handle changes in the temperature, like this\+:


\begin{DoxyCode}
\textcolor{comment}{// Temperature change handler functions must look like this.}
\textcolor{keyword}{typedef} void (*tempSensor\_ChangeHandlerFunc\_t)(int32\_t newTemperature, \textcolor{keywordtype}{void}* contextPtr);

\textcolor{comment}{// Opaque type used to refer to a registered temperature change handler.}
\textcolor{keyword}{typedef} \textcolor{keyword}{struct }tempSensor\_ChangeHandler* tempSensor\_ChangeHandlerRef\_t;

\textcolor{comment}{// Register a handler function to be called when the temperature changes.}
tempSensor\_ChangeHandlerRef\_t tempSensor\_AddChangeHandler
(
    tempSensor\_ChangeHandlerFunc\_t  handlerFunc,  \textcolor{comment}{// The handler function.}
    \textcolor{keywordtype}{void}*                           contextPtr    \textcolor{comment}{// Opaque pointer to pass to handler function.}
);

\textcolor{comment}{// De-register a handler function that was previously registered using}
\textcolor{comment}{// tempSensor\_AddChangeHandler().}
\textcolor{keywordtype}{void} tempSensor\_RemoveChangeHandler
(
    tempSensor\_ChangeHandlerRef\_t  handlerRef
);
\end{DoxyCode}


The implementation could look like this\+:


\begin{DoxyCode}
\hyperlink{le__event_loop_8h_abdb9187a56836a93d19cc793cbd4b7ec}{COMPONENT\_INIT}
\{
    TempChangeEventId = \hyperlink{le__event_loop_8h_a41a96eb3affb07184b519164cf54e213}{le\_event\_CreateId}(\textcolor{stringliteral}{"TempChange"}, \textcolor{keyword}{sizeof}(int32\_t));
\}

\textcolor{keyword}{static} \textcolor{keywordtype}{void} TempChangeHandler
(
    \textcolor{keywordtype}{void}* reportPtr,
    \textcolor{keywordtype}{void}* secondLayerHandlerFunc
)
\{
    int32\_t* temperaturePtr = reportPtr;
    tempSensor\_ChangeHandlerRef\_t clientHandlerFunc = secondLayerHandlerFunc;

    clientHandlerFunc(*temperaturePtr, \hyperlink{le__event_loop_8h_a1c73916295cc9e17af07e02756aa86c9}{le\_event\_GetContextPtr}());
\}

tempSensor\_ChangeHandlerRef\_t tempSensor\_AddChangeHandler
(
    tempSensor\_ChangeHandlerFunc\_t  handlerFunc,
    \textcolor{keywordtype}{void}*                           contextPtr
)
\{
    \hyperlink{le__event_loop_8h_ae7ab96b8e3441b3d484fcf52aa7a9dad}{le\_event\_HandlerRef\_t} handlerRef;

    handlerRef = \hyperlink{le__event_loop_8h_a8b906d38935f64953482f42c745e1c18}{le\_event\_AddLayeredHandler}(\textcolor{stringliteral}{"TempChange"},
                                            TempChangeEventId,
                                            TempChangeHandler,
                                            handlerFunc);
    \hyperlink{le__event_loop_8h_ae0c4307a9715794c720e525032aa0bfd}{le\_event\_SetContextPtr}(handlerRef, contextPtr);

    \textcolor{keywordflow}{return} (tempSensor\_ChangeHandlerRef\_t)handlerRef;
\}

\textcolor{keywordtype}{void} tempSensor\_RemoveChangeHandler
(
    tempSensor\_ChangeHandlerRef\_t    handlerRef
)
\{
    \hyperlink{le__event_loop_8h_ae31a85d4acbef72451b5411a613eea58}{le\_event\_RemoveHandler}((\hyperlink{le__event_loop_8h_ae7ab96b8e3441b3d484fcf52aa7a9dad}{le\_event\_HandlerRef\_t})handlerRef);
\}
\end{DoxyCode}


This approach gives strong type checking of both handler references and handler function pointers in code that uses this Temperature Sensor A\+P\+I.\hypertarget{c_event_loop_c_event_reportingRefCountedObjects}{}\subsection{Event Reports Containing Reference-\/\+Counted Objects}\label{c_event_loop_c_event_reportingRefCountedObjects}
Sometimes you need to report an event where the report payload is pointing to a reference-\/counted object allocated from a memory pool (see \hyperlink{c_memory}{Dynamic Memory Allocation A\+P\+I}). Memory leaks and/or crashes can result if its is sent through the Event Loop A\+P\+I without telling the Event Loop A\+P\+I it\textquotesingle{}s pointing to a reference counted object. If there are no subscribers, the Event Loop A\+P\+I iscards the reference without releasing it, and the object is never be deleted. If multiple handlers are registered, the reference could be released by the handlers too many times. Also, there are other, subtle issues that are nearly impossible to solve if threads terminate while reports containing pointers to reference-\/counted objects are on their Event Queues.

To help with this, the functions {\ttfamily \hyperlink{le__event_loop_8h_a31bef8276ad0e911fd84fb710d58ca2b}{le\+\_\+event\+\_\+\+Create\+Id\+With\+Ref\+Counting()}} and {\ttfamily \hyperlink{le__event_loop_8h_af0277165493b512216fabb6086ec7d9c}{le\+\_\+event\+\_\+\+Report\+With\+Ref\+Counting()}} have been provided. These allow a pointer to a reference-\/counted memory pool object to be sent as the payload of an Event Report.

{\ttfamily \hyperlink{le__event_loop_8h_af0277165493b512216fabb6086ec7d9c}{le\+\_\+event\+\_\+\+Report\+With\+Ref\+Counting()}} passes ownership of one reference to the Event Loop A\+P\+I, and when the handler is called, it receives ownership for one reference. It then becomes the handler\textquotesingle{}s responsibility to release its reference (using \hyperlink{le__mem_8h_a6d8e3fe430bcb81efe97b57ce30ef2de}{le\+\_\+mem\+\_\+\+Release()}) when it\textquotesingle{}s done.

{\ttfamily \hyperlink{le__event_loop_8h_a31bef8276ad0e911fd84fb710d58ca2b}{le\+\_\+event\+\_\+\+Create\+Id\+With\+Ref\+Counting()}} is used the same way as \hyperlink{le__event_loop_8h_a41a96eb3affb07184b519164cf54e213}{le\+\_\+event\+\_\+\+Create\+Id()}, except that it doesn\textquotesingle{}t require a payload size as the payload is always known from the pointer to a reference-\/counted memory pool object. Only Event I\+Ds created using \hyperlink{le__event_loop_8h_a31bef8276ad0e911fd84fb710d58ca2b}{le\+\_\+event\+\_\+\+Create\+Id\+With\+Ref\+Counting()} can be used with \hyperlink{le__event_loop_8h_af0277165493b512216fabb6086ec7d9c}{le\+\_\+event\+\_\+\+Report\+With\+Ref\+Counting()}.


\begin{DoxyCode}
\textcolor{keyword}{static} \hyperlink{le__event_loop_8h_ae6e351b38bc95954f159d16d19d2d55c}{le\_event\_Id\_t} EventId;
le\_mem\_PoolRef\_t MyObjectPoolRef;

\textcolor{keyword}{static} \textcolor{keywordtype}{void} MyHandler
(
    \textcolor{keywordtype}{void}* reportPtr  \textcolor{comment}{// Pointer to my reference-counted object.}
)
\{
    MyObj\_t* objPtr = reportPtr;

    \textcolor{comment}{// Do something with the object.}
    ...

    \textcolor{comment}{// Okay, I'm done with the object now.}
    \hyperlink{le__mem_8h_a6d8e3fe430bcb81efe97b57ce30ef2de}{le\_mem\_Release}(objPtr);
\}

\hyperlink{le__event_loop_8h_abdb9187a56836a93d19cc793cbd4b7ec}{COMPONENT\_INIT}
\{
    EventId = \hyperlink{le__event_loop_8h_a31bef8276ad0e911fd84fb710d58ca2b}{le\_event\_CreateIdWithRefCounting}(\textcolor{stringliteral}{"SomethingHappened"});
    \hyperlink{le__event_loop_8h_ae65a65b4111618f47d7e6d57a48289e5}{le\_event\_AddHandler}(\textcolor{stringliteral}{"SomethingHandler"}, EventId, MyHandler);
    MyObjectPoolRef = \hyperlink{le__mem_8h_ab91efaa2978c9c1c7b2427d25b33241c}{le\_mem\_CreatePool}(\textcolor{stringliteral}{"MyObjects"}, \textcolor{keyword}{sizeof}(MyObj\_t));
\}

\textcolor{keyword}{static} \textcolor{keywordtype}{void} ReportSomethingDetected
(
    ...
)
\{
    MyObj\_t* objPtr = \hyperlink{le__mem_8h_af7c289c73d4182835a26a9099f3db359}{le\_mem\_ForceAlloc}(MyObjectPool);

    \textcolor{comment}{// Fill in the object.}
    ...

    \hyperlink{le__event_loop_8h_af0277165493b512216fabb6086ec7d9c}{le\_event\_ReportWithRefCounting}(EventId, objPtr);
\}
\end{DoxyCode}
\hypertarget{c_event_loop_c_event_miscThreadingTopics}{}\subsection{Miscellaneous Multithreading Topics}\label{c_event_loop_c_event_miscThreadingTopics}
All functions in this A\+P\+I are thread safe.

Each thread can have only one Event Loop. The main thread in every Legato process will always run an Event Loop after it\textquotesingle{}s run the component initialization functions. As soon as all component initialization functions have returned, the main thread will start processing its event queue.

When a function is called to \char`\"{}\+Add\char`\"{} an event handler, that handler is associated with the calling thread\textquotesingle{}s Event Loop. If the calling thread doesn\textquotesingle{}t run its Event Loop, the event reports will pile up in the queue, never getting serviced and never releasing their memory. This will appear in the logs as event queue growth warnings.

If a client starts its own thread (e.\+g., by calling \hyperlink{le__thread_8h_a87e02a46f92e9e3e11ed28a2b265872f}{le\+\_\+thread\+\_\+\+Create()} ), then that thread will {\bfseries not} automatically run an Event Loop. To make it run an Event Loop, it must call {\ttfamily \hyperlink{le__event_loop_8h_ae313b457994371c658be9fe0494a01ff}{le\+\_\+event\+\_\+\+Run\+Loop()}} (which will never return).

If a thread running an Event Loop terminates, the Legato framework automatically deregisters any handlers and deletes the thread\textquotesingle{}s Event Loop, its Event Queue, and any event reports still in that Event Queue.\hypertarget{c_event_loop_c_event_integratingLegacyPosix}{}\subsection{Integrating with Legacy P\+O\+S\+I\+X Code}\label{c_event_loop_c_event_integratingLegacyPosix}
Many legacy programs written on top of P\+O\+S\+I\+X A\+P\+Is will have previously built their own event loop using poll(), select(), or some other blocking functions. It may be difficult to refactor this type of event loop to use the Legato event loop instead.

Two functions are provided to assist integrating legacy code with the Legato Event Loop\+:
\begin{DoxyItemize}
\item {\ttfamily \hyperlink{le__event_loop_8h_a12ce7f92f4bc6f5167d5a6ef86d7d0b1}{le\+\_\+event\+\_\+\+Get\+Fd()}} -\/ Fetches a file descriptor that can be monitored using some variant of poll() or select() (including epoll). It will appear readable when the Event Loop needs servicing.
\item {\ttfamily \hyperlink{le__event_loop_8h_a096222e98f6a0d92a79722018a752b58}{le\+\_\+event\+\_\+\+Service\+Loop()}} -\/ Services the event loop. This should be called if the file descriptor returned by \hyperlink{le__event_loop_8h_a12ce7f92f4bc6f5167d5a6ef86d7d0b1}{le\+\_\+event\+\_\+\+Get\+Fd()} appears readable to poll() or select().
\end{DoxyItemize}

In an attempt to avoid starving the caller when there are a lot of things that need servicing on the Event Loop, {\ttfamily \hyperlink{le__event_loop_8h_a096222e98f6a0d92a79722018a752b58}{le\+\_\+event\+\_\+\+Service\+Loop()}} will only perform one servicing step (i.\+e., call one event handler function) before returning, regardless of how much work there is to do. It\textquotesingle{}s the caller\textquotesingle{}s responsibility to check the return code from \hyperlink{le__event_loop_8h_a096222e98f6a0d92a79722018a752b58}{le\+\_\+event\+\_\+\+Service\+Loop()} and keep calling until it indicates that there is no more work to be done.\hypertarget{c_event_loop_c_event_troubleshooting}{}\subsection{Troubleshooting}\label{c_event_loop_c_event_troubleshooting}
A logging keyword can be enabled to view a given thread\textquotesingle{}s event handling activity. The keyword name depends on the thread and process name where the thread is located. For example, the keyword \char`\"{}\+P/\+T/events\char`\"{} controls logging for a thread named \char`\"{}\+T\char`\"{} running inside a process named \char`\"{}\+P\char`\"{}.





Copyright (C) Sierra Wireless Inc. Use of this work is subject to license. \hypertarget{c_fdMonitor}{}\section{File Descriptor Monitor A\+P\+I}\label{c_fdMonitor}
\hyperlink{le__fd_monitor_8h}{A\+P\+I Reference}





In a P\+O\+S\+I\+X environment, like Linux, file descriptors (fds) are used for most process I/\+O. Many components need to be notified when one or more fds are ready to read from or write to, or if there\textquotesingle{}s an error or hang-\/up.

Although it\textquotesingle{}s common to block a thread on a call to {\ttfamily read()}, {\ttfamily write()}, {\ttfamily accept()}, {\ttfamily select()}, {\ttfamily poll()} (or some variantion of these), if that\textquotesingle{}s done in a thread shared with other components, the other components won\textquotesingle{}t run when needed. To avoid this, Legato has methods to monitor fds reporting related events so they won\textquotesingle{}t interfere with other software sharing the same thread.\hypertarget{c_fd_monitor_c_fdMonitorStartStop}{}\subsection{Start/\+Stop Monitoring}\label{c_fd_monitor_c_fdMonitorStartStop}
\hyperlink{le__fd_monitor_8h_a52902d634d810f9b7a23c53c9c5164f0}{le\+\_\+fd\+Monitor\+\_\+\+Create()} creates a {\bfseries  File Descriptor Monitor } and starts monitoring an fd. A handler function and set of events is also provided to \hyperlink{le__fd_monitor_8h_a52902d634d810f9b7a23c53c9c5164f0}{le\+\_\+fd\+Monitor\+\_\+\+Create()}.


\begin{DoxyCode}
\textcolor{comment}{// Monitor for data available to read.}
\hyperlink{le__fd_monitor_8h_a85048556f0b95147af81e76907895d42}{le\_fdMonitor\_Ref\_t} fdMonitor = \hyperlink{le__fd_monitor_8h_a52902d634d810f9b7a23c53c9c5164f0}{le\_fdMonitor\_Create}(\textcolor{stringliteral}{"Serial Port"},     
       \textcolor{comment}{// Name for diagnostics}
                                                   fd,                 \textcolor{comment}{// fd to monitor}
                                                   SerialPortHandler,  \textcolor{comment}{// Handler function}
                                                   POLLIN);            \textcolor{comment}{// Monitor readability}
\end{DoxyCode}


When an fd no longer needs to be monitored, the File Descriptor Monitor object is deleted by calling \hyperlink{le__fd_monitor_8h_ad7f0f1a0cd2f99b081403784f048aef0}{le\+\_\+fd\+Monitor\+\_\+\+Delete()}.


\begin{DoxyCode}
\hyperlink{le__fd_monitor_8h_ad7f0f1a0cd2f99b081403784f048aef0}{le\_fdMonitor\_Delete}(fdMonitor);
\end{DoxyCode}


\begin{DoxyWarning}{Warning}
Always delete the Monitor object for an fd {\bfseries  before closing the fd }. After an fd is closed, it could get reused for something completely different. If monitoring of the new fd incarnation is started before the old Monitor object is deleted, deleting the old Monitor will cause monitoring of the new incarnation to fail.
\end{DoxyWarning}
\hypertarget{c_fd_monitor_c_fdMonitorEvents}{}\subsection{Event Types}\label{c_fd_monitor_c_fdMonitorEvents}
Events that can be handled\+:


\begin{DoxyItemize}
\item {\ttfamily P\+O\+L\+L\+I\+N} = Data available to read.
\item {\ttfamily P\+O\+L\+L\+P\+R\+I} = Urgent data available to read (e.\+g., out-\/of-\/band data on a socket).
\item {\ttfamily P\+O\+L\+L\+O\+U\+T} = Writing to the fd should accept some data now.
\item {\ttfamily P\+O\+L\+L\+R\+D\+H\+U\+P} = Other end of stream socket closed or shutdown.
\item {\ttfamily P\+O\+L\+L\+E\+R\+R} = Error occurred.
\item {\ttfamily P\+O\+L\+L\+H\+U\+P} = Hang up.
\end{DoxyItemize}

These are bitmask values and can be combined using the bit-\/wise O\+R operator (\textquotesingle{}$\vert$\textquotesingle{}) and tested for using the bit-\/wise {\itshape and} (\textquotesingle{}\&\textquotesingle{}) operator.

\begin{DoxyNote}{Note}
{\ttfamily P\+O\+L\+L\+R\+D\+H\+U\+P}, {\ttfamily P\+O\+L\+L\+E\+R\+R} and {\ttfamily P\+O\+L\+L\+H\+U\+P} can\textquotesingle{}t be disabled. Monitoring these events is always enabled as soon as the File Descriptor Monitor is created regardless of the set of events given to \hyperlink{le__fd_monitor_8h_a52902d634d810f9b7a23c53c9c5164f0}{le\+\_\+fd\+Monitor\+\_\+\+Create()}.
\end{DoxyNote}
\hypertarget{c_fd_monitor_c_fdTypes}{}\subsection{F\+D Types}\label{c_fd_monitor_c_fdTypes}
The fd type affects how events are monitored\+:


\begin{DoxyItemize}
\item \hyperlink{c_fd_monitor_c_fdTypes_files}{Files}
\item \hyperlink{c_fd_monitor_c_fdTypes_pipes}{Pipes}
\item \hyperlink{c_fd_monitor_c_fdTypes_sockets}{Sockets}
\item \hyperlink{c_fd_monitor_c_fdTypes_terminals}{Terminals and Pseud-\/terminals}
\end{DoxyItemize}\hypertarget{c_fd_monitor_c_fdTypes_files}{}\subsubsection{Files}\label{c_fd_monitor_c_fdTypes_files}

\begin{DoxyItemize}
\item P\+O\+L\+L\+I\+N and P\+O\+L\+L\+O\+U\+T are always S\+E\+T
\item N\+O\+N\+E of the other E\+V\+E\+N\+T\+S are ever set
\end{DoxyItemize}\hypertarget{c_fd_monitor_c_fdTypes_pipes}{}\subsubsection{Pipes}\label{c_fd_monitor_c_fdTypes_pipes}
Pipe fd events indicate two conditions for reading from a pipe and two conditions for writing to a pipe.

\begin{TabularC}{3}
\hline
\rowcolor{lightgray}{\bf }&{\bf Event }&{\bf Condition  }\\\cline{1-3}
R\+E\+A\+D\+I\+N\+G from a pipe &P\+O\+L\+L\+H\+U\+P &No D\+A\+T\+A in the pipe and the W\+R\+I\+T\+E-\/\+E\+N\+D is closed \\\cline{1-3}
&P\+O\+L\+L\+I\+N &D\+A\+T\+A in the pipe and the W\+R\+I\+T\+E\+\_\+\+E\+N\+D is open \\\cline{1-3}
&P\+O\+L\+L\+I\+N or P\+O\+L\+L\+H\+U\+P&D\+A\+T\+A in the pipe B\+U\+T the W\+R\+I\+T\+E-\/\+E\+N\+D is closed \\\cline{1-3}
W\+R\+I\+T\+I\+N\+G to the pipe &P\+O\+L\+L\+E\+R\+R &No S\+P\+A\+C\+E in the pipe and the R\+E\+A\+D-\/\+E\+N\+D is closed \\\cline{1-3}
&P\+O\+L\+L\+O\+U\+T &S\+P\+A\+C\+E in the pipe and the R\+E\+A\+D-\/\+E\+N\+D is open \\\cline{1-3}
&P\+O\+L\+L\+O\+U\+T or P\+O\+L\+L\+E\+R\+R &S\+P\+A\+C\+E in the pipe B\+U\+T the R\+E\+A\+D-\/\+E\+N\+D is closed \\\cline{1-3}
\end{TabularC}
\hypertarget{c_fd_monitor_c_fdTypes_sockets}{}\subsubsection{Sockets}\label{c_fd_monitor_c_fdTypes_sockets}
Socket activity (establishing/closing) is monitored for connection-\/orientated sockets including S\+O\+C\+K\+\_\+\+S\+T\+R\+E\+A\+M and S\+O\+C\+K\+\_\+\+S\+E\+Q\+P\+A\+C\+K\+E\+T. Input and output data availability for all socket types is monitored. \begin{TabularC}{2}
\hline
\rowcolor{lightgray}{\bf Event }&{\bf Condition  }\\\cline{1-2}
P\+O\+L\+L\+I\+N &Input is available from the socket \\\cline{1-2}
P\+O\+L\+L\+O\+U\+T &Possible to send data on the socket \\\cline{1-2}
P\+O\+L\+L\+I\+N &Incoming connection being established on the listen port \\\cline{1-2}
P\+O\+L\+L\+P\+R\+I &Out of band data received only on T\+C\+P \\\cline{1-2}
P\+O\+L\+L\+I\+N or P\+O\+L\+L\+O\+U\+T or P\+O\+L\+L\+R\+D\+H\+U\+P &Peer closed the connection in a connection-\/orientated socket \\\cline{1-2}
\end{TabularC}
\hypertarget{c_fd_monitor_c_fdTypes_terminals}{}\subsubsection{Terminals and Pseud-\/terminals}\label{c_fd_monitor_c_fdTypes_terminals}
Terminals and pseudo-\/terminals operate in pairs. When one terminal pair closes, an event is generated to indicate the closure. P\+O\+L\+L\+I\+N, P\+O\+L\+L\+O\+U\+T and P\+O\+L\+L\+P\+R\+I are the event indicators related to terminal status.

\begin{TabularC}{2}
\hline
\rowcolor{lightgray}{\bf Event }&{\bf Condition  }\\\cline{1-2}
P\+O\+L\+L\+I\+N &Ready to receive data \\\cline{1-2}
P\+O\+L\+L\+O\+U\+T &Ready to send data \\\cline{1-2}
P\+O\+L\+L\+P\+R\+I &Master/pseudo terminal detects slave state has changed (in packet mode only). \\\cline{1-2}
P\+O\+L\+L\+H\+U\+P &Either half of the terminal pair has closed. \\\cline{1-2}
\end{TabularC}
\hypertarget{c_fd_monitor_c_fdMonitorHandlers}{}\subsection{Handler Functions}\label{c_fd_monitor_c_fdMonitorHandlers}
Parameters to the fd event handler functions are the fd and the events active for the fd. The events are passed as a bit mask; the bit-\/wise A\+N\+D operator (\textquotesingle{}\&\textquotesingle{}) must be used to check for specific events.


\begin{DoxyCode}
\hyperlink{le__event_loop_8h_abdb9187a56836a93d19cc793cbd4b7ec}{COMPONENT\_INIT}
\{
    \textcolor{comment}{// Open the serial port.}
    \textcolor{keywordtype}{int} fd = open(\textcolor{stringliteral}{"/dev/ttyS0"}, O\_RDWR|O\_NONBLOCK);
    \hyperlink{le__log_8h_a7a3e66a87026cc9e57bcb748840ab41b}{LE\_FATAL\_IF}(fd == -1, \textcolor{stringliteral}{"open failed with errno %d (%m)"}, errno);

    \textcolor{comment}{// Create a File Descriptor Monitor object for the serial port's file descriptor.}
    \textcolor{comment}{// Monitor for data available to read.}
    \hyperlink{le__fd_monitor_8h_a85048556f0b95147af81e76907895d42}{le\_fdMonitor\_Ref\_t} fdMonitor = \hyperlink{le__fd_monitor_8h_a52902d634d810f9b7a23c53c9c5164f0}{le\_fdMonitor\_Create}(\textcolor{stringliteral}{"Serial Port"}, 
           \textcolor{comment}{// Name for diagnostics}
                                                       fd,                 \textcolor{comment}{// fd to monitor}
                                                       SerialPortHandler,  \textcolor{comment}{// Handler function}
                                                       POLLIN);            \textcolor{comment}{// Monitor readability}
\}

\textcolor{keyword}{static} \textcolor{keywordtype}{void} SerialPortHandler(\textcolor{keywordtype}{int} fd, \textcolor{keywordtype}{short} events)
\{
    \textcolor{keywordflow}{if} (events & POLLIN)    \textcolor{comment}{// Data available to read?}
    \{
        \textcolor{keywordtype}{char} buff[MY\_BUFF\_SIZE];

        ssize\_t bytesRead = read(fd, buff, \textcolor{keyword}{sizeof}(buff));

        ...
    \}

    \textcolor{keywordflow}{if} ((events & POLLERR) || (events & POLLHUP) || (events & POLLRDHUP))   \textcolor{comment}{// Error or hang-up?}
    \{
        ...
    \}
\}
\end{DoxyCode}
\hypertarget{c_fd_monitor_c_fdMonitorEnableDisable}{}\subsection{Enable/\+Disable Event Monitoring}\label{c_fd_monitor_c_fdMonitorEnableDisable}
The set of fd events being monitored can be adjusted using \hyperlink{le__fd_monitor_8h_a497aee19dbedadf884f404958713b414}{le\+\_\+fd\+Monitor\+\_\+\+Enable()} and \hyperlink{le__fd_monitor_8h_ada2b1023507b99e9247175dd3ffe5d48}{le\+\_\+fd\+Monitor\+\_\+\+Disable()}. However, {\ttfamily P\+O\+L\+L\+R\+D\+H\+U\+P}, {\ttfamily P\+O\+L\+L\+E\+R\+R} and {\ttfamily P\+O\+L\+L\+H\+U\+P} can\textquotesingle{}t be disabled.

C\+P\+U cycles (and power) can be saved by disabling monitoring when not needed. For example, {\ttfamily P\+O\+L\+L\+O\+U\+T} monitoring should be disabled while nothing needs to be written to the fd, so that the event handler doesn\textquotesingle{}t keep getting called with a {\ttfamily P\+O\+L\+L\+O\+U\+T} event because the fd is writeable.


\begin{DoxyCode}
\textcolor{keyword}{static} \textcolor{keywordtype}{void} StartWriting()
\{
    \textcolor{comment}{// Enable monitoring for POLLOUT.  When connection is ready, handler will be called.}
    \hyperlink{le__fd_monitor_8h_a497aee19dbedadf884f404958713b414}{le\_fdMonitor\_Enable}(FdMonitorRef, POLLOUT);
\}

\textcolor{keyword}{static} \textcolor{keywordtype}{void} ConnectionEventHandler(\textcolor{keywordtype}{int} fd, \textcolor{keywordtype}{int} event)
\{
    \textcolor{keywordflow}{if} (event & POLLOUT)
    \{
        \textcolor{comment}{// Connection is ready for us to send some data.}
        \hyperlink{le__basics_8h_a1cca095ed6ebab24b57a636382a6c86c}{le\_result\_t} result = SendWaitingData();
        \textcolor{keywordflow}{if} (result == \hyperlink{le__basics_8h_a1cca095ed6ebab24b57a636382a6c86ca77a7505b0443df2fa1bab375c7267637}{LE\_NOT\_FOUND})
        \{
            \textcolor{comment}{// Buffer empty, stop monitoring POLLOUT so handler doesn't keep getting called.}
            \hyperlink{le__fd_monitor_8h_ada2b1023507b99e9247175dd3ffe5d48}{le\_fdMonitor\_Disable}(\hyperlink{le__fd_monitor_8h_a688da8b3627d20b01795dfa1ae46bb78}{le\_fdMonitor\_GetMonitor}(), 
      POLLOUT);
        \}
        ...
    \}
    ...
\}
\end{DoxyCode}


If an event occurs on an fd while monitoring of that event is disabled, the event will be ignored. If that event is later enabled, and that event\textquotesingle{}s trigger condition is still true (e.\+g., the fd still has data available to be read), then the event will be reported to the handler at that time. If the event trigger condition is gone (e.\+g., the fd no longer has data available to read), then the event will not be reported until its trigger condition becomes true again.

If events occur on different fds at the same time, the order in which the handlers are called is implementation-\/dependent.\hypertarget{c_fd_monitor_c_fdMonitorHandlerContext}{}\subsection{Handler Function Context}\label{c_fd_monitor_c_fdMonitorHandlerContext}
Calling \hyperlink{le__fd_monitor_8h_a688da8b3627d20b01795dfa1ae46bb78}{le\+\_\+fd\+Monitor\+\_\+\+Get\+Monitor()} inside the handler function fetches a reference to the File Descriptor Monitor object for the event being handled. This is handy to enable and disable event monitoring from inside the handler.

If additional data needs to be passed to the handler function, the context pointer can be set to use \hyperlink{le__fd_monitor_8h_af88e40be018bcd6cd47a64bde49a9f98}{le\+\_\+fd\+Monitor\+\_\+\+Set\+Context\+Ptr()} and retrieved inside the handler function with \hyperlink{le__fd_monitor_8h_a3073205dc7ee7a054857119bc11c9dfc}{le\+\_\+fd\+Monitor\+\_\+\+Get\+Context\+Ptr()}. \hyperlink{le__event_loop_8h_a1c73916295cc9e17af07e02756aa86c9}{le\+\_\+event\+\_\+\+Get\+Context\+Ptr()} can also be used, but \hyperlink{le__fd_monitor_8h_a3073205dc7ee7a054857119bc11c9dfc}{le\+\_\+fd\+Monitor\+\_\+\+Get\+Context\+Ptr()} is preferred as it double checks it\textquotesingle{}s being called inside a File Descriptor Monitor\textquotesingle{}s handler function.


\begin{DoxyCode}
\textcolor{keyword}{static} \textcolor{keywordtype}{void} SerialPortHandler(\textcolor{keywordtype}{int} fd, \textcolor{keywordtype}{short} events)
\{
    MyContext\_t* contextPtr = \hyperlink{le__fd_monitor_8h_a3073205dc7ee7a054857119bc11c9dfc}{le\_fdMonitor\_GetContextPtr}();

    \textcolor{comment}{// Process the fd event(s).}
    ...
\}

\textcolor{keyword}{static} \textcolor{keywordtype}{void} StartDataTransmission(\textcolor{keyword}{const} \textcolor{keywordtype}{char}* port, uint8\_t* txBuffPtr, \textcolor{keywordtype}{size\_t} txBytes)
\{
    \textcolor{comment}{// Open the serial port.}
    \textcolor{keywordtype}{int} fd = open(port, O\_RDWR|O\_NONBLOCK);
    \hyperlink{le__log_8h_a7a3e66a87026cc9e57bcb748840ab41b}{LE\_FATAL\_IF}(fd == -1, \textcolor{stringliteral}{"open failed with errno %d (%m)"}, errno);

    \textcolor{comment}{// Create a File Descriptor Monitor object for the serial port's file descriptor.}
    \textcolor{comment}{// Monitor for write buffer space availability.}
    \hyperlink{le__fd_monitor_8h_a85048556f0b95147af81e76907895d42}{le\_fdMonitor\_Ref\_t} fdMonitor = \hyperlink{le__fd_monitor_8h_a52902d634d810f9b7a23c53c9c5164f0}{le\_fdMonitor\_Create}(\textcolor{stringliteral}{"Port"}, fd, 
      SerialPortHandler, POLLOUT);

    \textcolor{comment}{// Allocate a data block and populate with stuff we need in SerialPortHandler().}
    MyContext\_t* contextPtr = \hyperlink{le__mem_8h_af7c289c73d4182835a26a9099f3db359}{le\_mem\_ForceAlloc}(ContextMemPool);
    contextPtr->txBuffPtr = txBuffPtr;
    contextPtr->bytesRemaining = txBytes;

    \textcolor{comment}{// Make this available to SerialPortHandler() via le\_fdMonitor\_GetContextPtr().}
    \hyperlink{le__fd_monitor_8h_af88e40be018bcd6cd47a64bde49a9f98}{le\_fdMonitor\_SetContextPtr}(fdMonitor, contextPtr);
\}
\end{DoxyCode}
\hypertarget{c_fd_monitor_c_fdMonitorPowerManagement}{}\subsection{Power Management}\label{c_fd_monitor_c_fdMonitorPowerManagement}
If your process has the privilege of being able to block the system from going to sleep, whenever the fd that is being monitored has a pending event, the system will be kept awake. To allow the system to go to sleep while this fd has a pending event, you can call \hyperlink{le__fd_monitor_8h_a66a93ae01f1e6faf1d0c7645752d4442}{le\+\_\+fd\+Monitor\+\_\+\+Set\+Deferrable()} with {\ttfamily is\+Deferrable} flag set to \textquotesingle{}true\textquotesingle{}.\hypertarget{c_fd_monitor_c_fdMonitorThreading}{}\subsection{Threading}\label{c_fd_monitor_c_fdMonitorThreading}
fd monitoring is performed by the Event Loop of the thread that created the Monitor object for that fd. If that the is blocked, events won\textquotesingle{}t be detected for that fd until the thread is unblocked and returns to its Event Loop. Similarly, if the thread that creates a File Descriptor Monitor object doesn\textquotesingle{}t run an Event Loop at all, no events will be detected for that fd.

It\textquotesingle{}s not recommended to monitor the same fd in two threads at the same time, because the threads will race to handle any events on that fd.\hypertarget{c_fd_monitor_c_fdMonitorTroubleshooting}{}\subsection{Troubleshooting}\label{c_fd_monitor_c_fdMonitorTroubleshooting}
The \char`\"{}fd\+Monitor\char`\"{} logging keyword can be enabled to view fd monitoring activity.





Copyright (C) Sierra Wireless Inc. Use of this work is subject to license. \hypertarget{c_flock}{}\section{File Locking A\+P\+I}\label{c_flock}
\hyperlink{le__file_lock_8h}{A\+P\+I Reference}





File locking is a form of I\+P\+C used to synchronize multiple processes\textquotesingle{} access to common files.

This A\+P\+I provides a co-\/operative file locking mechanism that can be used by multiple processes and/or threads to sychronize reads and writes to common files.

This A\+P\+I only supports regular files. Attempts to use this A\+P\+I on sockets, devices, etc. results in undefined behaviour.\hypertarget{c_flock_c_flock_cooperative}{}\subsection{Co-\/operative File Locking}\label{c_flock_c_flock_cooperative}
Co-\/operative file locks (also known as advisory file locks) means that the processes and threads must co-\/operate to synchronize their access to the file. If a process or thread simply ignores the lock and accesses the file then access synchronization errors may occur.\hypertarget{c_flock_c_flock_locks}{}\subsection{Locking Files}\label{c_flock_c_flock_locks}
There are two types of locks that can be applied\+: read lock and write lock. A file can have multiple simultaneous read locks, but can only have one write lock. Also, a file can only have one type of lock on it at one time. A file may be locked for reading if the file is unlocked or if there are read locks on the file, but to lock a file for writing the file must be unlocked.

Use {\ttfamily \hyperlink{le__file_lock_8h_aac3e11a6f7f363d29b8dbb1eb6c2c287}{le\+\_\+flock\+\_\+\+Open()}} to lock a file and open it for access. When attempting to lock a file that already has an incompatible lock on it, {\ttfamily \hyperlink{le__file_lock_8h_aac3e11a6f7f363d29b8dbb1eb6c2c287}{le\+\_\+flock\+\_\+\+Open()}} will block until it can obtain the lock. Call {\ttfamily \hyperlink{le__file_lock_8h_a457a07dbf8967757322f531d5beb10b6}{le\+\_\+flock\+\_\+\+Close()}} to close the file and remove the lock on the file.

This code sample shows four processes attempting to access the same file. Assume that all the calls to \hyperlink{le__file_lock_8h_aac3e11a6f7f363d29b8dbb1eb6c2c287}{le\+\_\+flock\+\_\+\+Open()} in the example occur in chronological order as they appear\+:


\begin{DoxyCode}
     \textcolor{comment}{// Code in Process 1.}

     \textcolor{comment}{// Lock the file for reading.}
     \textcolor{keywordtype}{int} fd = \hyperlink{le__file_lock_8h_aac3e11a6f7f363d29b8dbb1eb6c2c287}{le\_flock\_Open}(\textcolor{stringliteral}{"foo"}, \hyperlink{le__file_lock_8h_a5e5400e33a5e10b7c624748a9ce11280a887421ec0def966e3ffc65e6bde1f1fc}{LE\_FLOCK\_READ});  \textcolor{comment}{// This call will not block.}

     \textcolor{comment}{// Read from the file.}
     ...

     \textcolor{comment}{// Close the file and release the lock.}
     \hyperlink{le__file_lock_8h_a457a07dbf8967757322f531d5beb10b6}{le\_flock\_Close}(fd);
-------------------------------------------------------------------------------------------------

     \textcolor{comment}{// Code in Process 2.}

     \textcolor{comment}{// Lock the file for reading.}
     \textcolor{keywordtype}{int} fd = \hyperlink{le__file_lock_8h_aac3e11a6f7f363d29b8dbb1eb6c2c287}{le\_flock\_Open}(\textcolor{stringliteral}{"foo"}, \hyperlink{le__file_lock_8h_a5e5400e33a5e10b7c624748a9ce11280a887421ec0def966e3ffc65e6bde1f1fc}{LE\_FLOCK\_READ});  \textcolor{comment}{// This call will not block.}

     \textcolor{comment}{// Read from the file.}
     ...

     \textcolor{comment}{// Close the file and release the lock.}
     \hyperlink{le__file_lock_8h_a457a07dbf8967757322f531d5beb10b6}{le\_flock\_Close}(fd);
-------------------------------------------------------------------------------------------------

     \textcolor{comment}{// Code in Process 3.}

     \textcolor{comment}{// Lock the file for writing.}
     \textcolor{keywordtype}{int} fd = \hyperlink{le__file_lock_8h_aac3e11a6f7f363d29b8dbb1eb6c2c287}{le\_flock\_Open}(\textcolor{stringliteral}{"foo"}, \hyperlink{le__file_lock_8h_a5e5400e33a5e10b7c624748a9ce11280a058867728a1de4773023f009c2934188}{LE\_FLOCK\_WRITE});  \textcolor{comment}{// This call will block
       until both Process 1}
                                                     \textcolor{comment}{// and Process 2 removes their locks.}

     \textcolor{comment}{// Write to the file.}
     ...

     \textcolor{comment}{// Close the file and release the lock.}
     \hyperlink{le__file_lock_8h_a457a07dbf8967757322f531d5beb10b6}{le\_flock\_Close}(fd);
\end{DoxyCode}


This sample shows that Process 2 obtains the read lock even though Process 1 already has a read lock on the file. Process 3 is blocked because it\textquotesingle{}s attempting a write lock on the file. Process 3 is blocked until both Process 1 and 2 remove their locks.

When multiple processes are blocked waiting to obtain a lock on the file, it\textquotesingle{}s unspecified which process will obtain the lock when the file becomes available.

The \hyperlink{le__file_lock_8h_a8fdca3e28190ef85e4457ebf009410b5}{le\+\_\+flock\+\_\+\+Create()} function can be used to create, lock and open a file in one function call.\hypertarget{c_flock_c_flock_streams}{}\subsection{Streams}\label{c_flock_c_flock_streams}
The functions {\ttfamily \hyperlink{le__file_lock_8h_ae9a845ef8afe7cb7c4767573a974e5a0}{le\+\_\+flock\+\_\+\+Open\+Stream()}} and {\ttfamily \hyperlink{le__file_lock_8h_a6444d5e3d885a7c346cba6993534020b}{le\+\_\+flock\+\_\+\+Create\+Stream()}} can be used to obtain a file stream to a locked file. {\ttfamily \hyperlink{le__file_lock_8h_a8cd7aad1d732c6719097daf0359bf32f}{le\+\_\+flock\+\_\+\+Close\+Stream()}} is used to close the stream and remove the lock. These functions are analogous to \hyperlink{le__file_lock_8h_aac3e11a6f7f363d29b8dbb1eb6c2c287}{le\+\_\+flock\+\_\+\+Open()}, \hyperlink{le__file_lock_8h_a8fdca3e28190ef85e4457ebf009410b5}{le\+\_\+flock\+\_\+\+Create()} and \hyperlink{le__file_lock_8h_a457a07dbf8967757322f531d5beb10b6}{le\+\_\+flock\+\_\+\+Close()} except that they return file streams rather than file descriptors.\hypertarget{c_flock_c_flock_nonblock}{}\subsection{Non-\/blocking}\label{c_flock_c_flock_nonblock}
Functions \hyperlink{le__file_lock_8h_aac3e11a6f7f363d29b8dbb1eb6c2c287}{le\+\_\+flock\+\_\+\+Open()}, \hyperlink{le__file_lock_8h_a8fdca3e28190ef85e4457ebf009410b5}{le\+\_\+flock\+\_\+\+Create()}, \hyperlink{le__file_lock_8h_ae9a845ef8afe7cb7c4767573a974e5a0}{le\+\_\+flock\+\_\+\+Open\+Stream()} and \hyperlink{le__file_lock_8h_a6444d5e3d885a7c346cba6993534020b}{le\+\_\+flock\+\_\+\+Create\+Stream()} always block if there is an incompatible lock on the file. Functions \hyperlink{le__file_lock_8h_add7b73f75a8e7956a397081987458590}{le\+\_\+flock\+\_\+\+Try\+Open()}, \hyperlink{le__file_lock_8h_a4f7b134b467adb749401f2ef2ccd92d2}{le\+\_\+flock\+\_\+\+Try\+Create()}, \hyperlink{le__file_lock_8h_aa4712b501c620401a3f269c5cb34d91a}{le\+\_\+flock\+\_\+\+Try\+Open\+Stream()} and \hyperlink{le__file_lock_8h_aa1c3c10f1f72a5541f31855b5c2eed98}{le\+\_\+flock\+\_\+\+Try\+Create\+Stream()} are their non-\/blocking counterparts.\hypertarget{c_flock_c_flock_threads}{}\subsection{Multiple Threads}\label{c_flock_c_flock_threads}
All functions in this A\+P\+I are thread-\/safe; processes and threads can use this A\+P\+I to synchronize their access to files.\hypertarget{c_flock_c_flock_replicateFd}{}\subsection{Replicating File Descriptors}\label{c_flock_c_flock_replicateFd}
File locks are contained in the file descriptors that are returned by \hyperlink{le__file_lock_8h_aac3e11a6f7f363d29b8dbb1eb6c2c287}{le\+\_\+flock\+\_\+\+Open()} and \hyperlink{le__file_lock_8h_a8fdca3e28190ef85e4457ebf009410b5}{le\+\_\+flock\+\_\+\+Create()} and in the underlying file descriptors of the file streams returned by \hyperlink{le__file_lock_8h_ae9a845ef8afe7cb7c4767573a974e5a0}{le\+\_\+flock\+\_\+\+Open\+Stream()} and \hyperlink{le__file_lock_8h_a6444d5e3d885a7c346cba6993534020b}{le\+\_\+flock\+\_\+\+Create\+Stream()}.

File descriptors are closed the locks are automatically removed. Functions \hyperlink{le__file_lock_8h_a457a07dbf8967757322f531d5beb10b6}{le\+\_\+flock\+\_\+\+Close()} and \hyperlink{le__file_lock_8h_a8cd7aad1d732c6719097daf0359bf32f}{le\+\_\+flock\+\_\+\+Close\+Stream()} are provided as a convenience. When a process dies, all of its file descriptors are closed and any file locks they may contain are removed.

If a file descriptor is replicated either through dup() or fork(), the file lock will also be replicated in the new file descriptor\+:


\begin{DoxyCode}
\textcolor{keywordtype}{int} oldfd = \hyperlink{le__file_lock_8h_aac3e11a6f7f363d29b8dbb1eb6c2c287}{le\_flock\_Open}(\textcolor{stringliteral}{"foo"}, \hyperlink{le__file_lock_8h_a5e5400e33a5e10b7c624748a9ce11280a887421ec0def966e3ffc65e6bde1f1fc}{LE\_FLOCK\_READ});  \textcolor{comment}{// Place a read lock on the
       file "foo".}
\textcolor{keywordtype}{int} newfd = dup(oldfd);

\hyperlink{le__file_lock_8h_a457a07dbf8967757322f531d5beb10b6}{le\_flock\_Close}(oldfd); \textcolor{comment}{// Closes the fd and removes the lock.}
\end{DoxyCode}


There must still be a read lock on the file \char`\"{}foo\char`\"{} because newfd has not been closed.

This behaviour can be used to pass file locks from a parent to a child through a fork() call. The parent can obtain the file lock, fork() and close its file descriptor. Now the child has exclusive possession of the file lock.\hypertarget{c_flock_c_flock_limitations}{}\subsection{Limitations}\label{c_flock_c_flock_limitations}
Here are some limitations to the file locking mechanisms in this A\+P\+I\+:

The file locks in this A\+P\+I are advisory only, meaning that a process may simply ignore the lock and access the file anyways.

This A\+P\+I does not detect deadlocks and a process may deadlock itself. For example\+:


\begin{DoxyCode}
\textcolor{keywordtype}{int} fd1 = \hyperlink{le__file_lock_8h_aac3e11a6f7f363d29b8dbb1eb6c2c287}{le\_flock\_Open}(\textcolor{stringliteral}{"foo"}, \hyperlink{le__file_lock_8h_a5e5400e33a5e10b7c624748a9ce11280a887421ec0def966e3ffc65e6bde1f1fc}{LE\_FLOCK\_READ});   \textcolor{comment}{// Obtains a read lock on the
       file.}
\textcolor{keywordtype}{int} fd2 = \hyperlink{le__file_lock_8h_aac3e11a6f7f363d29b8dbb1eb6c2c287}{le\_flock\_Open}(\textcolor{stringliteral}{"foo"}, \hyperlink{le__file_lock_8h_a5e5400e33a5e10b7c624748a9ce11280a058867728a1de4773023f009c2934188}{LE\_FLOCK\_WRITE});  \textcolor{comment}{// This call will block
       forever.}
\end{DoxyCode}


This A\+P\+I only permits whole files to be locked, not portions of a file.

Many N\+F\+S implementations don\textquotesingle{}t recognize locks used by this A\+P\+I.





Copyright (C) Sierra Wireless Inc. Use of this work is subject to license. \hypertarget{c_hashmap}{}\section{Hash\+Map A\+P\+I}\label{c_hashmap}
\hyperlink{le__hashmap_8h}{A\+P\+I Reference}





This A\+P\+I provides a straightforward Hash\+Map implementation.\hypertarget{c_hashmap_c_hashmap_create}{}\subsection{Creating a Hash\+Map}\label{c_hashmap_c_hashmap_create}
Use {\ttfamily \hyperlink{le__hashmap_8h_ade79896a5b2ceec82c570fe21f7efe3a}{le\+\_\+hashmap\+\_\+\+Create()}} to create a hashmap. It\textquotesingle{}s the responsibility of the caller to maintain type integrity using this function\textquotesingle{}s parameters. It\textquotesingle{}s important to supply hash and equality functions that operate on the type of key that you intend to store. It\textquotesingle{}s unwise to mix types in a single table because implementation of the table has no way to detect this behaviour.

Choose the initial size should carefully as the index size remains fixed. The best choice for the initial size is a prime number slightly larger than the maximum expected capacity. If a too small size is chosen, there will be an increase in collisions that degrade performance over time.

All hashmaps have names for diagnostic purposes.\hypertarget{c_hashmap_c_hashmap_insert}{}\subsection{Adding key-\/value pairs}\label{c_hashmap_c_hashmap_insert}
Key-\/value pairs are added using \hyperlink{le__hashmap_8h_a68759fb8291c487a507eae6d92710fc7}{le\+\_\+hashmap\+\_\+\+Put()}. For example\+:


\begin{DoxyCode}
\textcolor{keyword}{static} \textcolor{keywordtype}{void} StoreStuff(\textcolor{keyword}{const} \textcolor{keywordtype}{char}* keyStr, \textcolor{keyword}{const} \textcolor{keywordtype}{char}* valueStr)
\{
    myTable = \hyperlink{le__hashmap_8h_ade79896a5b2ceec82c570fe21f7efe3a}{le\_hashmap\_Create}(
                        \textcolor{stringliteral}{"My Table"},
                        31,
                        \hyperlink{le__hashmap_8h_a3ff75de814b38d4c4283379acb406b65}{le\_hashmap\_HashString},
                        \hyperlink{le__hashmap_8h_a63d2b6c0689ece50ce979557029b8483}{le\_hashmap\_EqualsString}
                      );

    \hyperlink{le__hashmap_8h_a68759fb8291c487a507eae6d92710fc7}{le\_hashmap\_Put}(myTable, keyStr, valueStr);
    ....
\}
\end{DoxyCode}


The table does not take control of the keys or values. The map only stores the pointers to these values, not the values themselves. It\textquotesingle{}s the responsibility of the caller to manage the actual data storage.\hypertarget{c_hashmap_c_hashmap_tips}{}\subsubsection{Tip}\label{c_hashmap_c_hashmap_tips}
The code sample shows some pre-\/defined functions for certain key types. The key types supported are uint32\+\_\+t and strings. The strings must be N\+U\+L\+L terminated.

Tables can also have their own hash and equality functions, but ensure the functions work on the type of key you\textquotesingle{}re storing. The hash function should provide a good distribution of values. It is not required that they be unique.\hypertarget{c_hashmap_c_hashmap_iterating}{}\subsection{Iterating over a map}\label{c_hashmap_c_hashmap_iterating}
This A\+P\+I allows the user of the map to iterate over the entire map, acting on each key-\/value pair. You supply a callback function conforming to the prototype\+: 
\begin{DoxyCode}
bool (*callback)(\textcolor{keywordtype}{void}* key, \textcolor{keywordtype}{void}* value, \textcolor{keywordtype}{void}* context)
\end{DoxyCode}


This can then be used to process every value in the map. The return value from the callback function determines if iteration should continue or stop. If the function returns false then iteration will cease. For example\+:


\begin{DoxyCode}
\textcolor{keywordtype}{bool} ProcessTableData
(
    \textcolor{keywordtype}{void}* keyPtr,     \textcolor{comment}{// Pointer to the map entry's key}
    \textcolor{keywordtype}{void}* valuePtr,   \textcolor{comment}{// Pointer to the map entry's value}
    \textcolor{keywordtype}{void}* contextPtr  \textcolor{comment}{// Pointer to an abritrary context for the callback function}
)
\{
    \textcolor{keywordtype}{int} keyCheck = *((\textcolor{keywordtype}{int}*)context);
    \textcolor{keywordtype}{int} currentKey = *((\textcolor{keywordtype}{int}*)key);

    \textcolor{keywordflow}{if} (keyCheck == currentKey) \textcolor{keywordflow}{return} \textcolor{keyword}{false};

    \textcolor{comment}{// Do some stuff, maybe print out data or do a calculation}
    \textcolor{keywordflow}{return} \textcolor{keyword}{true};
\}

\{
    \textcolor{comment}{// ... somewhere else in the code ...}
    \textcolor{keywordtype}{int} lastKey = 10;
    \hyperlink{le__hashmap_8h_a2fc335fffcf59a677ac2ac4e5733cdda}{le\_hashmap\_ForEach} (myTable, processTableData, &lastKey);
\}
\end{DoxyCode}


This code sample shows the callback function must also be aware of the types stored in the table.

However, keep in mind that it is unsafe and undefined to modify the map during this style of iteration.

Alternatively, the calling function can control the iteration by first calling {\ttfamily \hyperlink{le__hashmap_8h_a8fb1d3a3d4c4b1b52a45205ac11a12c1}{le\+\_\+hashmap\+\_\+\+Get\+Iterator()}}. This returns an iterator that is ready to return each key/value pair in the map in the order in which they are stored. The iterator is controlled by calling {\ttfamily \hyperlink{le__hashmap_8h_a601b7d3e5d92e91e4090d726e5b190ca}{le\+\_\+hashmap\+\_\+\+Next\+Node()}}, and must be called before accessing any elements. You can then retrieve pointers to the key and value by using \hyperlink{le__hashmap_8h_a8c983aea3bfa393419b4ea26cfe35f42}{le\+\_\+hashmap\+\_\+\+Get\+Key()} and \hyperlink{le__hashmap_8h_aefd09b502200c3260a047cb12097e8ad}{le\+\_\+hashmap\+\_\+\+Get\+Value()}.

\begin{DoxyNote}{Note}
There is only one iterator per hashtable. Calling \hyperlink{le__hashmap_8h_a8fb1d3a3d4c4b1b52a45205ac11a12c1}{le\+\_\+hashmap\+\_\+\+Get\+Iterator()} will simply re-\/initialize the current iterator
\end{DoxyNote}
It is possible to add and remove items during this style of iteration. When adding items during an iteration it is not guaranteed that the newly added item will be iterated over. It\textquotesingle{}s very possible that the newly added item is added in an earlier location than the iterator is curently pointed at.

When removing items during an iteration you also have to keep in mind that the iterator\textquotesingle{}s current item may be the one removed. If this is the case, le\+\_\+hashmap\+\_\+\+Get\+Key, and le\+\_\+hashmap\+\_\+\+Get\+Value will return N\+U\+L\+L until either, le\+\_\+hashmap\+\_\+\+Next\+Node, or le\+\_\+hashmap\+\_\+\+Prev\+Node are called.

For example (assuming a table of string/string)\+:


\begin{DoxyCode}
\textcolor{keywordtype}{void} ProcessTable(\hyperlink{le__hashmap_8h_ae81c60860dbdb8c59beaf25985e5605a}{le\_hashmap\_Ref\_t} myTable)
\{
    \textcolor{keywordtype}{char}* nextKey;
    \textcolor{keywordtype}{char}* nextVal;

    \hyperlink{le__hashmap_8h_a8ab2021261a368add28c1be14f248459}{le\_hashmap\_It\_Ref\_t} myIter = \hyperlink{le__hashmap_8h_a8fb1d3a3d4c4b1b52a45205ac11a12c1}{le\_hashmap\_GetIterator}(myTable);

    \textcolor{keywordflow}{while} (\hyperlink{le__basics_8h_a1cca095ed6ebab24b57a636382a6c86ca5066a4bcec691c6b67843b8f79656422}{LE\_OK} == \hyperlink{le__hashmap_8h_a601b7d3e5d92e91e4090d726e5b190ca}{le\_hashmap\_NextNode}(myIter))
    \{
        \textcolor{comment}{// Do something with the strings}
        nextKey = \hyperlink{le__hashmap_8h_a8c983aea3bfa393419b4ea26cfe35f42}{le\_hashmap\_GetKey}(myIter);
        nextVal = \hyperlink{le__hashmap_8h_aefd09b502200c3260a047cb12097e8ad}{le\_hashmap\_GetValue}(myIter);
    \}
\}
\end{DoxyCode}


If you need to control access to the hashmap, then a mutex can be used.\hypertarget{c_hashmap_c_hashmap_tracing}{}\subsection{Tracing a map}\label{c_hashmap_c_hashmap_tracing}
Hashmaps can be traced using the logging system.

If {\ttfamily \hyperlink{le__hashmap_8h_a853082500b05e57d899606cfc0e34fab}{le\+\_\+hashmap\+\_\+\+Make\+Traceable()}} is called for a specified hashmap object, the name of that hashmap (the name passed into \hyperlink{le__hashmap_8h_ade79896a5b2ceec82c570fe21f7efe3a}{le\+\_\+hashmap\+\_\+\+Create()} ) becomes a trace keyword to enable and disable tracing of that particular hashmap.

If {\ttfamily \hyperlink{le__hashmap_8h_a10b30e794df1c866fe39c40c7949eb29}{le\+\_\+hashmap\+\_\+\+Enable\+Trace()}} is called for a hashmap object, tracing is immediately activated for that hashmap.

See \hyperlink{c_logging_c_log_controlling}{Log Controls} for more information on how to enable and disable tracing using configuration and/or the log control tool.





Copyright (C) Sierra Wireless Inc. Use of this work is subject to license. \hypertarget{c_hex}{}\section{Hex string A\+P\+I}\label{c_hex}
\hyperlink{le__hex_8h}{A\+P\+I Reference}





This A\+P\+I provides convertion tools to switch between\+:
\begin{DoxyItemize}
\item \hyperlink{le__hex_8h_a2ad2c35d567e8fc3fc962a58272b093d}{le\+\_\+hex\+\_\+\+String\+To\+Binary} Hex-\/\+String to binary
\item \hyperlink{le__hex_8h_a2482e5240d47b176e41369fc5a654551}{le\+\_\+hex\+\_\+\+Binary\+To\+String} Binary to Hex-\/\+String
\end{DoxyItemize}\hypertarget{c_hex_hex_conversion}{}\subsection{Conversion}\label{c_hex_hex_conversion}
Code sample\+:


\begin{DoxyCode}
\textcolor{keywordtype}{char}     HexString[] = \textcolor{stringliteral}{"136ABC"};
uint8\_t  binString[] = \{0x13,0x6A,0xBC\};
\end{DoxyCode}


So \hyperlink{le__hex_8h_a2ad2c35d567e8fc3fc962a58272b093d}{le\+\_\+hex\+\_\+\+String\+To\+Binary} will convert Hex\+String to bin\+String.

and \hyperlink{le__hex_8h_a2482e5240d47b176e41369fc5a654551}{le\+\_\+hex\+\_\+\+Binary\+To\+String} will convert bin\+String to Hex\+String.





Copyright (C) Sierra Wireless Inc. Use of this work is subject to license. \hypertarget{c_logging}{}\section{Logging A\+P\+I}\label{c_logging}
\hyperlink{le__log_8h}{A\+P\+I Reference} ~\newline
 \hyperlink{howToLogs}{Use Logs} 



The Legato Logging A\+P\+I provides a toolkit allowing code to be instrumented with error, warning, informational, and debugging messages. These messages can be turned on or off remotely and pushed or pulled from the device through a secure shell, cloud services interfaces, e-\/mail, S\+M\+S, etc.\hypertarget{c_logging_c_log_logging}{}\subsection{Logging Basics}\label{c_logging_c_log_logging}
Legato\textquotesingle{}s logging can be configured through this A\+P\+I, and there\textquotesingle{}s also a command-\/line target \hyperlink{toolsTarget_log}{log} tool available.\hypertarget{c_logging_c_log_levels}{}\subsubsection{Levels}\label{c_logging_c_log_levels}
Log messages are categorized according to the severity of the information being logged. A log message may be purely informational, describing something that is expected to occur from time-\/to-\/time during normal operation; or it may be a report of a fault that might have a significant negative impact on the operation of the system. To differentiate these, each log entry is associated with one of the following log levels\+:


\begin{DoxyItemize}
\item \hyperlink{le__log_8h_aa3de78c088c398afb23c0b582deabc0aa1a5c0e9f9bcf857faad0cc4187002479}{D\+E\+B\+U\+G}\+: Handy for troubleshooting.
\item \hyperlink{le__log_8h_aa3de78c088c398afb23c0b582deabc0aa83ea376539849ee701096fdb022e74b3}{I\+N\+F\+O\+R\+M\+A\+T\+I\+O\+N}\+: Expected to happen; can be interesting even when not troubleshooting.
\item \hyperlink{le__log_8h_aa3de78c088c398afb23c0b582deabc0aae0f565809442f7de555d36f76c36627c}{W\+A\+R\+N\+I\+N\+G}\+: Should not normally happen; may not have any real impact on system performance.
\item \hyperlink{le__log_8h_aa3de78c088c398afb23c0b582deabc0aaf1203f512370bfec7a05f8adae13c7d9}{E\+R\+R\+O\+R}\+: Fault that may result in noticeable short-\/term system misbehaviour. Needs attention.
\item \hyperlink{le__log_8h_aa3de78c088c398afb23c0b582deabc0aa298894fe77b90cb9b28f25817a620df8}{C\+R\+I\+T\+I\+C\+A\+L}\+: Fault needs urgent attention. Will likely result in system failure.
\item \hyperlink{le__log_8h_aa3de78c088c398afb23c0b582deabc0aa6843cc8d9fb10d02cad1834b236dd5cb}{E\+M\+E\+R\+G\+E\+N\+C\+Y}\+: Definite system failure.
\end{DoxyItemize}\hypertarget{c_logging_c_log_basic_defaultSyslog}{}\subsubsection{Standard Out and Standard Error in Syslog}\label{c_logging_c_log_basic_defaultSyslog}
By default, app processes will have their {\ttfamily stdout} and {\ttfamily stderr} redirected to the {\ttfamily syslog}. Each process’s stdout will be logged at I\+N\+F\+O severity level; it’s stderr will be logged at “\+E\+R\+R” severity level.

There are two limitations with this feature\+:
\begin{DoxyItemize}
\item the P\+I\+D reported in the logs generally refer to the P\+I\+D of the process that generates the stdout/stderr message. If a process forks, then both the parent and child processes’ stdout/stderr will share the same connection to the syslog, and the parent’s P\+I\+D will be reported in the logs for both processes.
\item stdout is line buffered when connected to a terminal, which means {\ttfamily printf(“hello~\newline
”)} will be printed to the terminal immediately. If stdout is connected to something like a pipe it\textquotesingle{}s bulk buffered, which means a flush doesn\textquotesingle{}t occur until the buffer is full.
\end{DoxyItemize}

To make your process line buffer stdout so that printf will show up in the logs as expected, the {\ttfamily setlinebuf(stdout)} system call can be used. Alternatively, {\ttfamily fflush(stdout)} can be called \textbackslash{} to force a flush of the stdout buffer.

This issue doesn\textquotesingle{}t exist with stderr as stderr is never buffered.\hypertarget{c_logging_c_log_basic_logging}{}\subsubsection{Basic Logging}\label{c_logging_c_log_basic_logging}
A series of macros are available to make logging easy.

None of them return anything.

All of them accept printf-\/style arguments, consisting of a format string followed by zero or more parameters to be printed (depending on the contents of the format string).

There is a logging macro for each of the log levels\+:


\begin{DoxyItemize}
\item \hyperlink{le__log_8h_a2a91ea8857cf190fde71d85ba930a498}{L\+E\+\_\+\+D\+E\+B\+U\+G}(format\+String, ...)
\item \hyperlink{le__log_8h_a23e6d206faa64f612045d688cdde5808}{L\+E\+\_\+\+I\+N\+F\+O}(format\+String, ...)
\item \hyperlink{le__log_8h_a0201b2f60ee0e945479f91e181bf04b6}{L\+E\+\_\+\+W\+A\+R\+N}(format\+String, ...)
\item \hyperlink{le__log_8h_a353590f91b3143a7ba3a416ae5a50c3d}{L\+E\+\_\+\+E\+R\+R\+O\+R}(format\+String, ...)
\item \hyperlink{le__log_8h_a5efa1e4b6292c820c8555b4066a5c10d}{L\+E\+\_\+\+C\+R\+I\+T}(format\+String, ...)
\item \hyperlink{le__log_8h_a651e75cb4ec9d59f5ddc7bae2fbdde88}{L\+E\+\_\+\+E\+M\+E\+R\+G}(format\+String, ...)
\end{DoxyItemize}

For example, 
\begin{DoxyCode}
\hyperlink{le__log_8h_a23e6d206faa64f612045d688cdde5808}{LE\_INFO}(\textcolor{stringliteral}{"Obtained new IP address %s."}, ipAddrStr);
\end{DoxyCode}
\hypertarget{c_logging_c_log_conditional_logging}{}\subsubsection{Conditional Logging}\label{c_logging_c_log_conditional_logging}
Similar to the basic macros, but these contain a conditional expression as their first parameter. If this expression equals true, then the macro will generate this log output\+:


\begin{DoxyItemize}
\item \hyperlink{le__log_8h_a4c689f24157ca91e6d39827c967de734}{L\+E\+\_\+\+D\+E\+B\+U\+G\+\_\+\+I\+F}(expression, format\+String, ...)
\item \hyperlink{le__log_8h_ad98222f9a0871cde1893eac589841f17}{L\+E\+\_\+\+I\+N\+F\+O\+\_\+\+I\+F}(expression, format\+String, ...)
\item \hyperlink{le__log_8h_a8d8f204806cd5fc0455fe3caacf1d251}{L\+E\+\_\+\+W\+A\+R\+N\+\_\+\+I\+F}(expression, format\+String, ...)
\item \hyperlink{le__log_8h_aceaf11a11691d6c676e36dd317b38dbd}{L\+E\+\_\+\+E\+R\+R\+O\+R\+\_\+\+I\+F}(expression, format\+String, ...)
\item \hyperlink{le__log_8h_ae507036675ece2d77e8e285cf864a6f4}{L\+E\+\_\+\+C\+R\+I\+T\+\_\+\+I\+F}(expression, format\+String, ...)
\item \hyperlink{le__log_8h_a5dcf8dc55407a8f41ada2e85ff8131fe}{L\+E\+\_\+\+E\+M\+E\+R\+G\+\_\+\+I\+F}(expression, format\+String, ...)
\end{DoxyItemize}

Instead of writing 
\begin{DoxyCode}
\textcolor{keywordflow}{if} (result == -1)
\{
    \hyperlink{le__log_8h_a0201b2f60ee0e945479f91e181bf04b6}{LE\_WARN}(\textcolor{stringliteral}{"Failed to send message to server.  Errno = %m."});
\}
\end{DoxyCode}


you could write this\+: 
\begin{DoxyCode}
\hyperlink{le__log_8h_a8d8f204806cd5fc0455fe3caacf1d251}{LE\_WARN\_IF}(result == -1, \textcolor{stringliteral}{"Failed to send message to server.  Errno = %m."});
\end{DoxyCode}
\hypertarget{c_logging_c_log_loging_fatals}{}\subsubsection{Fatal Errors}\label{c_logging_c_log_loging_fatals}
There are some special logging macros intended for fatal errors\+:


\begin{DoxyItemize}
\item \hyperlink{le__log_8h_a54b4b07f5396e19a8d9fca74238f4795}{L\+E\+\_\+\+F\+A\+T\+A\+L}(format\+String, ...) ~\newline
 Always kills the calling process after logging the message at E\+M\+E\+R\+G\+E\+N\+C\+Y level (never returns).
\item \hyperlink{le__log_8h_a7a3e66a87026cc9e57bcb748840ab41b}{L\+E\+\_\+\+F\+A\+T\+A\+L\+\_\+\+I\+F}(condition, format\+String, ...) ~\newline
 If the condition is true, kills the calling process after logging the message at E\+M\+E\+R\+G\+E\+N\+C\+Y level.
\item \hyperlink{le__log_8h_ac0dbbef91dc0fed449d0092ff0557b39}{L\+E\+\_\+\+A\+S\+S\+E\+R\+T(condition)} ~\newline
 If the condition is true, does nothing. If the condition is false, logs the source code text of the condition at E\+M\+E\+R\+G\+E\+N\+C\+Y level and kills the calling process.
\end{DoxyItemize}

For example, 
\begin{DoxyCode}
\textcolor{keywordflow}{if} (NULL == objPtr)
\{
    \hyperlink{le__log_8h_a54b4b07f5396e19a8d9fca74238f4795}{LE\_FATAL}(\textcolor{stringliteral}{"Object pointer is NULL!"});
\}

\textcolor{comment}{// Now I can go ahead and use objPtr, knowing that if it was NULL then LE\_FATAL() would have}
\textcolor{comment}{// been called and LE\_FATAL() never returns.}
\end{DoxyCode}


or,


\begin{DoxyCode}
\hyperlink{le__log_8h_a7a3e66a87026cc9e57bcb748840ab41b}{LE\_FATAL\_IF}(NULL == objPtr, \textcolor{stringliteral}{"Object pointer is NULL!"});

\textcolor{comment}{// Now I can go ahead and use objPtr, knowing that if it was NULL then LE\_FATAL\_IF() would not}
\textcolor{comment}{// have returned.}
\end{DoxyCode}


or,


\begin{DoxyCode}
\hyperlink{le__log_8h_ac0dbbef91dc0fed449d0092ff0557b39}{LE\_ASSERT}(NULL != objPtr);

\textcolor{comment}{// Now I can go ahead and use objPtr, knowing that if it was NULL then LE\_ASSERT() would not}
\textcolor{comment}{// have returned.}
\end{DoxyCode}
\hypertarget{c_logging_c_log_tracing}{}\subsubsection{Tracing}\label{c_logging_c_log_tracing}
Finally, a macro is provided for tracing\+:


\begin{DoxyItemize}
\item \hyperlink{le__log_8h_a331fb6c78ccddeafc455ad9c64e42008}{L\+E\+\_\+\+T\+R\+A\+C\+E}(trace\+Ref, string, ...)
\end{DoxyItemize}

This macro is special because it\textquotesingle{}s independent of log level. Instead, trace messages are associated with a trace keyword. Tracing can be enabled and disabled based on these keywords.

If a developer wanted to trace the creation of \char`\"{}shape\char`\"{} objects in their G\+U\+I package, they could add trace statements like the following\+:


\begin{DoxyCode}
\hyperlink{le__log_8h_a331fb6c78ccddeafc455ad9c64e42008}{LE\_TRACE}(NewShapeTraceRef, \textcolor{stringliteral}{"Created %p with position (%d,%d)."}, shapePtr, shapePtr->x, shapePtr->y)
      ;
\end{DoxyCode}


The reference to the trace is obtained at start-\/up as follows\+:


\begin{DoxyCode}
NewShapeTraceRef = \hyperlink{le__log_8h_a6d99d8147bcdcd1ed3848c9fdb72afe5}{le\_log\_GetTraceRef}(\textcolor{stringliteral}{"newShape"});
\end{DoxyCode}


This allows enabling and disabling these \hyperlink{le__log_8h_a331fb6c78ccddeafc455ad9c64e42008}{L\+E\+\_\+\+T\+R\+A\+C\+E()} calls using the \char`\"{}new\+Shape\char`\"{} keyword through configuration settings and runtime log control tools. See \hyperlink{c_logging_c_log_controlling}{Log Controls} below.

Applications can use \hyperlink{le__log_8h_a0746e2e37585f61ccdf7ce4976909936}{L\+E\+\_\+\+I\+S\+\_\+\+T\+R\+A\+C\+E\+\_\+\+E\+N\+A\+B\+L\+E\+D(\+New\+Shape\+Trace\+Ref)} to query whether a trace keyword is enabled.

These allow apps to hook into the trace management system to use it to implement sophisticated, app-\/specific tracing or profiling features.\hypertarget{c_logging_c_log_resultTxt}{}\subsubsection{Result Code Text}\label{c_logging_c_log_resultTxt}
The \hyperlink{le__basics_8h_a1cca095ed6ebab24b57a636382a6c86c}{le\+\_\+result\+\_\+t} macro supports printing an error condition in a human-\/readable text string.


\begin{DoxyCode}
result = le\_foo\_DoSomething();

\textcolor{keywordflow}{if} (result != \hyperlink{le__basics_8h_a1cca095ed6ebab24b57a636382a6c86ca5066a4bcec691c6b67843b8f79656422}{LE\_OK})
\{
    \hyperlink{le__log_8h_a353590f91b3143a7ba3a416ae5a50c3d}{LE\_ERROR}(\textcolor{stringliteral}{"Failed to do something. Result = %d (%s)."}, result, 
      \hyperlink{le__log_8h_a99402d6a983f318e5b8bfcdf5dfe9024}{LE\_RESULT\_TXT}(result));
\}
\end{DoxyCode}
\hypertarget{c_logging_c_log_controlling}{}\subsection{Log Controls}\label{c_logging_c_log_controlling}
Log level filtering and tracing can be controlled at runtime using\+:
\begin{DoxyItemize}
\item the command-\/line Log Control Tool (\hyperlink{toolsTarget_log}{log})
\item configuration settings
\item environment variables
\item function calls.
\end{DoxyItemize}\hypertarget{c_logging_c_log_control_tool}{}\subsubsection{Log Control Tool}\label{c_logging_c_log_control_tool}
The log control tool is used from the command-\/line to control the log level filtering, log output location (syslog/stderr), and tracing for different components within a running system.

Online documentation can be accessed from the log control tool by running \char`\"{}log help\char`\"{}.

Here are some code samples.

To set the log level to I\+N\+F\+O for a component \char`\"{}my\+Comp\char`\"{} running in all processes with the name \char`\"{}my\+Proc\char`\"{}\+: \begin{DoxyVerb}$ log level INFO myProc/myComp
\end{DoxyVerb}


To set the log level to D\+E\+B\+U\+G for a component \char`\"{}my\+Comp\char`\"{} running in a process with P\+I\+D 1234\+: \begin{DoxyVerb}$ log level DEBUG 1234/myComp
\end{DoxyVerb}


To enable all L\+E\+\_\+\+T\+R\+A\+C\+E statements tagged with the keyword \char`\"{}foo\char`\"{} in a component called \char`\"{}my\+Comp\char`\"{} running in all processes called \char`\"{}my\+Proc\char`\"{}\+: \begin{DoxyVerb}$ log trace foo myProc/myComp
\end{DoxyVerb}


To disable the trace statements tagged with \char`\"{}foo\char`\"{} in the component \char`\"{}my\+Comp\char`\"{} in processes called \char`\"{}my\+Proc\char`\"{}\+: \begin{DoxyVerb}$ log stoptrace foo myProc/myComp
\end{DoxyVerb}


With all of the above examples \char`\"{}$\ast$\char`\"{} can be used in place of the process name or a component name (or both) to mean \char`\"{}all processes\char`\"{} and/or \char`\"{}all components\char`\"{}.\hypertarget{c_logging_c_log_control_config}{}\subsubsection{Log Control Configuration Settings}\label{c_logging_c_log_control_config}
\begin{DoxyNote}{Note}
The configuration settings haven\textquotesingle{}t been implemented yet.
\end{DoxyNote}
\hypertarget{c_logging_c_log_control_environment_vars}{}\subsubsection{Environment Variables}\label{c_logging_c_log_control_environment_vars}
Environment variables can be used to control the default log settings, taking effect immediately at process start-\/up; even before the Log Control Daemon has been connected to.

Settings in the Log Control Daemon (applied through configuration and/or the log control tool) will override the environment variable settings when the process connects to the Log Control Daemon.\hypertarget{c_logging_c_log_control_env_level}{}\paragraph{L\+E\+\_\+\+L\+O\+G\+\_\+\+L\+E\+V\+E\+L}\label{c_logging_c_log_control_env_level}
{\ttfamily L\+E\+\_\+\+L\+O\+G\+\_\+\+L\+E\+V\+E\+L} can be used to set the default log filter level for all components in the process. Valid values are\+:


\begin{DoxyItemize}
\item {\ttfamily E\+M\+E\+R\+G\+E\+N\+C\+Y} 
\item {\ttfamily C\+R\+I\+T\+I\+C\+A\+L} 
\item {\ttfamily E\+R\+R\+O\+R} 
\item {\ttfamily W\+A\+R\+N\+I\+N\+G} 
\item {\ttfamily I\+N\+F\+O} 
\item {\ttfamily D\+E\+B\+U\+G} 
\end{DoxyItemize}

For example, \begin{DoxyVerb}$ export LE_LOG_LEVEL=DEBUG
\end{DoxyVerb}
\hypertarget{c_logging_c_log_control_env_trace}{}\paragraph{L\+E\+\_\+\+L\+O\+G\+\_\+\+T\+R\+A\+C\+E}\label{c_logging_c_log_control_env_trace}
{\ttfamily L\+E\+\_\+\+L\+O\+G\+\_\+\+T\+R\+A\+C\+E} allows trace keywords to be enabled by default. The contents of this variable is a colon-\/separated list of keywords that should be enabled. Each keyword must be prefixed with a component name followed by a slash (\textquotesingle{}/\textquotesingle{}).

For example, \begin{DoxyVerb}$ export LE_LOG_TRACE=framework/fdMonitor:framework/logControl
\end{DoxyVerb}
\hypertarget{c_logging_c_log_control_functions}{}\subsubsection{Programmatic Log Control}\label{c_logging_c_log_control_functions}
Normally, configuration settings and the log control tool should suffice for controlling logging functionality. In some situations, it can be convenient to control logging programmatically in C.

\hyperlink{le__log_8h_a06856e82ef8dc28a964d84db1aeee517}{le\+\_\+log\+\_\+\+Set\+Filter\+Level()} sets the log filter level.

\hyperlink{le__log_8h_abb2a75e476fdc54dbd64c46c4853ab9d}{le\+\_\+log\+\_\+\+Get\+Filter\+Level()} gets the log filter level.

Trace keywords can be enabled and disabled programmatically by calling \hyperlink{le__log_8h_a7dfe7c6d02a68b473d94d83ae71ff2b3}{le\+\_\+log\+\_\+\+Enable\+Trace()} and \hyperlink{le__log_8h_a15495177d8e9bf1fe4603cbc6bade8fc}{le\+\_\+log\+\_\+\+Disable\+Trace()}.\hypertarget{c_logging_c_log_format}{}\subsection{Log Formats}\label{c_logging_c_log_format}
Log entries can also contain any of these\+:
\begin{DoxyItemize}
\item timestamp (century, year, month, day, hours, minutes, seconds, milliseconds, microseconds)
\item level (debug, info, warning, etc.) {\bfseries or} trace keyword
\item process I\+D
\item component name
\item thread name
\item source code file name
\item function name
\item source code line number
\end{DoxyItemize}

Log messages have the following format\+:

\begin{DoxyVerb}Jan  3 02:37:56  INFO  | processName[pid]/componentName T=threadName | fileName.c funcName() lineNum | Message
\end{DoxyVerb}
\hypertarget{c_logging_c_log_debugFiles}{}\subsection{App Crash Logs}\label{c_logging_c_log_debugFiles}
When a process within an app faults or exits in error, a copy of the current syslog buffer is captured along with a core file of the process crash (if generated).

The core file maximum size is determined by the process settings {\ttfamily max\+Core\+Dump\+File\+Bytes} and {\ttfamily max\+File\+Bytes} found in the processes section of your app\textquotesingle{}s {\ttfamily }.adef file. By default, the {\ttfamily max\+Core\+Dump\+File\+Bytes} is set to 0, do not create a core file.

To help save the target from flash burnout, the syslog and core files are stored in the R\+A\+M F\+S under /tmp. When a crash occurs, this directory is created\+:

\begin{DoxyVerb} /tmp/legato_logs/<app-name>/<exe-name>/\end{DoxyVerb}


The path for my\+App and my\+Exe would be\+:

\begin{DoxyVerb} /tmp/legato_logs/myApp/myExe/\end{DoxyVerb}


The files in that directory look like this\+:

\begin{DoxyVerb} core-myExe-1418694851
 syslog-1418694851\end{DoxyVerb}


If the fault action for that app\textquotesingle{}s process is to reboot the target, the output location is changed to this (and is preserved across reboots)\+:

\begin{DoxyVerb} /mnt/flash/legato_logs/<app-name>/<exe-name>/\end{DoxyVerb}


To save on R\+A\+M and flash space, only the most recent 4 copies of each file are preserved.





Copyright (C) Sierra Wireless Inc. Use of this work is subject to license. \hypertarget{c_messaging}{}\section{Low-\/\+Level Messaging A\+P\+I}\label{c_messaging}
\hyperlink{le__messaging_8h}{A\+P\+I Reference}





Message-\/based interfaces in Legato are implemented in layers. This low-\/level messaging A\+P\+I is at the bottom layer. It\textquotesingle{}s designed to support higher layers of the messaging system. But it\textquotesingle{}s also intended to be easy to hand-\/code low-\/level messaging in C, when necessary.

This low-\/level messaging A\+P\+I supports\+:
\begin{DoxyItemize}
\item remote party identification (addressing)
\item very late (runtime) discovery and binding of parties
\item in-\/process and inter-\/process message delivery
\item location transparency
\item sessions
\item access control
\item request/reply transactions
\item message buffer memory management
\item support for single-\/threaded and multi-\/threaded programs
\item some level of protection from protocol mismatches between parties in a session.
\end{DoxyItemize}

This A\+P\+I is integrated with the Legato Event Loop A\+P\+I so components can interact with each other using messaging without having to create threads or file descriptor sets that block other software from handling other events. Support for integration with legacy P\+O\+S\+I\+X-\/based programs is also provided.\hypertarget{c_messaging_c_messagingInteractionModel}{}\subsection{Interaction Model}\label{c_messaging_c_messagingInteractionModel}
The Legato low-\/level messaging system follows a service-\/oriented pattern\+:
\begin{DoxyItemize}
\item Service providers advertise their service.
\item Clients open sessions with those services
\item Both sides can send and receive messages through the session.
\end{DoxyItemize}

Clients and servers can both send one-\/way messages within a session. Clients can start a request-\/response transaction by sending a request to the server, and the server can send a response. Request-\/response transactions can be blocking or non-\/blocking (with a completion callback). If the server dies or terminates the session before sending the response, Legato will automatically terminate the transaction.

Servers are prevented from sending blocking requests to clients as a safety measure. If a server were to block waiting for one of its clients, it would open up the server to being blocked indefinitely by one of its clients, which would allow one client to cause a server to deny service to other clients. Also, if a client started a blocking request-\/response transaction at the same time that the server started a blocking request-\/response transaction in the other direction, a deadlock would occur.\hypertarget{c_messaging_c_messagingAddressing}{}\subsection{Addressing}\label{c_messaging_c_messagingAddressing}
Servers and clients have interfaces that can been connected to each other via bindings. Both client-\/side and server-\/side interfaces are identified by name, but the names don\textquotesingle{}t have to match for them to be bound to each other. The binding determines which server-\/side interface will be connected to when a client opens a session.

Server-\/side interfaces are also known as \char`\"{}services\char`\"{}.

When a session is opened by a client, a session reference is provided to both the client and the server. Messages are then sent within the session using the session reference. This session reference becomes invalid when the session is closed.\hypertarget{c_messaging_c_messagingProtocols}{}\subsection{Protocols}\label{c_messaging_c_messagingProtocols}
Communication between client and server is done using a message-\/based protocol. This protocol is defined at a higher layer than this A\+P\+I, so this A\+P\+I doesn\textquotesingle{}t know the structure of the message payloads or the correct message sequences. That means this A\+P\+I can\textquotesingle{}t check for errors in the traffic it carries. However, it does provide a basic mechanism for detecting protocol mismatches by forcing both client and server to provide the protocol identifier of the protocol to be used. The client and server must also provide the maximum message size, as an extra sanity check.

To make this possible, the client and server must independently call {\ttfamily \hyperlink{le__messaging_8h_adcd1ff1a6906433aaa6d7038125c4473}{le\+\_\+msg\+\_\+\+Get\+Protocol\+Ref()}}, to get a reference to a \char`\"{}\+Protocol\char`\"{} object that encapsulates these protocol details\+:


\begin{DoxyCode}
protocolRef = \hyperlink{le__messaging_8h_adcd1ff1a6906433aaa6d7038125c4473}{le\_msg\_GetProtocolRef}(MY\_PROTOCOL\_ID, \textcolor{keyword}{sizeof}(myproto\_Msg\_t));
\end{DoxyCode}


\begin{DoxyNote}{Note}
In this example, the protocol identifier (which is a string uniquely identifying a specific version of a specific protocol) and the message structure would be defined elsewhere, in a header file shared between the client and the server. The structure {\ttfamily myproto\+\_\+\+Msg\+\_\+t} contains a {\ttfamily union} of all of the different messages included in the protocol, thereby making {\ttfamily myproto\+\_\+\+Msg\+\_\+t} as big as the biggest message in the protocol.
\end{DoxyNote}
When a server creates a service (by calling \hyperlink{le__messaging_8h_adbbb2737069b636028128c74ae407742}{le\+\_\+msg\+\_\+\+Create\+Service()}) and when a client creates a session (by calling \hyperlink{le__messaging_8h_a696d7c2d4e3725d3ddb5dd2d79d2d732}{le\+\_\+msg\+\_\+\+Create\+Session()}), they are required to provide a reference to a Protocol object that they obtained from \hyperlink{le__messaging_8h_adcd1ff1a6906433aaa6d7038125c4473}{le\+\_\+msg\+\_\+\+Get\+Protocol\+Ref()}.\hypertarget{c_messaging_c_messagingClientUsage}{}\subsection{Client Usage Model}\label{c_messaging_c_messagingClientUsage}
\hyperlink{c_messaging_c_messagingClientSending}{Sending a Message} ~\newline
 \hyperlink{c_messaging_c_messagingClientReceiving}{Receiving a Non-\/\+Response Message} ~\newline
 \hyperlink{c_messaging_c_messagingClientClosing}{Closing Sessions} ~\newline
 \hyperlink{c_messaging_c_messagingClientMultithreading}{Multithreading} ~\newline
 \hyperlink{c_messaging_c_messagingClientExample}{Sample Code}

Clients that want to use a service do the following\+:
\begin{DoxyEnumerate}
\item Get a reference to the protocol they want to use by calling {\ttfamily \hyperlink{le__messaging_8h_adcd1ff1a6906433aaa6d7038125c4473}{le\+\_\+msg\+\_\+\+Get\+Protocol\+Ref()}}.
\item Create a session using {\ttfamily \hyperlink{le__messaging_8h_a696d7c2d4e3725d3ddb5dd2d79d2d732}{le\+\_\+msg\+\_\+\+Create\+Session()}}, passing in the protocol reference and the client\textquotesingle{}s interface name.
\item Optionally register a message receive callback using {\ttfamily \hyperlink{le__messaging_8h_ac726cc93219d326e1b10a7d13a0f4f65}{le\+\_\+msg\+\_\+\+Set\+Session\+Recv\+Handler()}}.
\item Open the session using \hyperlink{le__messaging_8h_a574d37960a07c4fc2bde310408619cff}{le\+\_\+msg\+\_\+\+Open\+Session()}, \hyperlink{le__messaging_8h_a8c6480e3708e20e1f9da032a93c80bc0}{le\+\_\+msg\+\_\+\+Open\+Session\+Sync()}, or \hyperlink{le__messaging_8h_a2ee1410da1dc345c86d6958b6cdda5e1}{le\+\_\+msg\+\_\+\+Try\+Open\+Session\+Sync()}.
\end{DoxyEnumerate}


\begin{DoxyCode}
protocolRef = \hyperlink{le__messaging_8h_adcd1ff1a6906433aaa6d7038125c4473}{le\_msg\_GetProtocolRef}(PROTOCOL\_ID, \textcolor{keyword}{sizeof}(myproto\_Msg\_t));
sessionRef = \hyperlink{le__messaging_8h_a696d7c2d4e3725d3ddb5dd2d79d2d732}{le\_msg\_CreateSession}(protocolRef, MY\_INTERFACE\_NAME);
\hyperlink{le__messaging_8h_ac726cc93219d326e1b10a7d13a0f4f65}{le\_msg\_SetSessionRecvHandler}(sessionRef, NotifyMsgHandlerFunc, NULL);
\hyperlink{le__messaging_8h_a574d37960a07c4fc2bde310408619cff}{le\_msg\_OpenSession}(sessionRef, SessionOpenHandlerFunc, NULL);
\end{DoxyCode}


The Legato framework takes care of setting up any I\+P\+C connections, as needed (or not, if the client and server happen to be in the same process).

When the session opens, the Event Loop will call the \char`\"{}session open handler\char`\"{} call-\/back function that was passed into \hyperlink{le__messaging_8h_a574d37960a07c4fc2bde310408619cff}{le\+\_\+msg\+\_\+\+Open\+Session()}.

\hyperlink{le__messaging_8h_a8c6480e3708e20e1f9da032a93c80bc0}{le\+\_\+msg\+\_\+\+Open\+Session\+Sync()} is a synchronous alternative to \hyperlink{le__messaging_8h_a574d37960a07c4fc2bde310408619cff}{le\+\_\+msg\+\_\+\+Open\+Session()}. The difference is that \hyperlink{le__messaging_8h_a8c6480e3708e20e1f9da032a93c80bc0}{le\+\_\+msg\+\_\+\+Open\+Session\+Sync()} will not return until the session has opened or failed to open (most likely due to permissions settings).

\hyperlink{le__messaging_8h_a2ee1410da1dc345c86d6958b6cdda5e1}{le\+\_\+msg\+\_\+\+Try\+Open\+Session\+Sync()} is like \hyperlink{le__messaging_8h_a8c6480e3708e20e1f9da032a93c80bc0}{le\+\_\+msg\+\_\+\+Open\+Session\+Sync()} except that it will not wait for a server session to become available if it is not already available at the time of the call. That is, if the client\textquotesingle{}s interface is not bound to any service, or if the service that it is bound to is not currently advertised by the server, then \hyperlink{le__messaging_8h_a2ee1410da1dc345c86d6958b6cdda5e1}{le\+\_\+msg\+\_\+\+Try\+Open\+Session\+Sync()} will return an error code.\hypertarget{c_messaging_c_messagingClientSending}{}\subsubsection{Sending a Message}\label{c_messaging_c_messagingClientSending}
Before sending a message, the client must first allocate the message from the session\textquotesingle{}s message pool using \hyperlink{le__messaging_8h_a8293a69f256b98cbce5b9990ea3520f3}{le\+\_\+msg\+\_\+\+Create\+Msg()}. It can then get a pointer to the payload part of the message using \hyperlink{le__messaging_8h_a32d1c7ffd913db8546f6f1bd5cce58c4}{le\+\_\+msg\+\_\+\+Get\+Payload\+Ptr()}. Once the message payload is populated, the client sends the message.


\begin{DoxyCode}
msgRef = \hyperlink{le__messaging_8h_a8293a69f256b98cbce5b9990ea3520f3}{le\_msg\_CreateMsg}(sessionRef);
msgPayloadPtr = \hyperlink{le__messaging_8h_a32d1c7ffd913db8546f6f1bd5cce58c4}{le\_msg\_GetPayloadPtr}(msgRef);
msgPayloadPtr->... = ...; \textcolor{comment}{// <-- Populate message payload...}
\end{DoxyCode}


If no response is required from the server, the client sends the message using \hyperlink{le__messaging_8h_a073de097d281475c44a445b927fbb929}{le\+\_\+msg\+\_\+\+Send()}. At this point, the client has handed off the message to the messaging system, and the messaging system will delete the message automatically once it has finished sending it.


\begin{DoxyCode}
\hyperlink{le__messaging_8h_a073de097d281475c44a445b927fbb929}{le\_msg\_Send}(msgRef);
\end{DoxyCode}


If the client expects a response from the server, the client can use \hyperlink{le__messaging_8h_a5440ae06a89b60ed04e9de5601496608}{le\+\_\+msg\+\_\+\+Request\+Response()} to send their message and specify a callback function to be called when the response arrives. This callback will be called by the event loop of the thread that created the session (i.\+e., the thread that called \hyperlink{le__messaging_8h_a696d7c2d4e3725d3ddb5dd2d79d2d732}{le\+\_\+msg\+\_\+\+Create\+Session()}).


\begin{DoxyCode}
\hyperlink{le__messaging_8h_a5440ae06a89b60ed04e9de5601496608}{le\_msg\_RequestResponse}(msgRef, ResponseHandlerFunc, NULL);
\end{DoxyCode}


If the client expects an immediate response from the server, and the client wants to block until that response is received, it can use \hyperlink{le__messaging_8h_aa3cf113b26b154697ccef270dafe8798}{le\+\_\+msg\+\_\+\+Request\+Sync\+Response()} instead of \hyperlink{le__messaging_8h_a5440ae06a89b60ed04e9de5601496608}{le\+\_\+msg\+\_\+\+Request\+Response()}. However, keep in mind that blocking the client thread will block all event handlers that share that thread. That\textquotesingle{}s why \hyperlink{le__messaging_8h_aa3cf113b26b154697ccef270dafe8798}{le\+\_\+msg\+\_\+\+Request\+Sync\+Response()} should only be used when the server is expected to respond immediately, or when the client thread is not shared by other event handlers.


\begin{DoxyCode}
responseMsgRef = \hyperlink{le__messaging_8h_aa3cf113b26b154697ccef270dafe8798}{le\_msg\_RequestSyncResponse}(msgRef);
\end{DoxyCode}


\begin{DoxyWarning}{Warning}
If the client and server are running in the same thread, and the client calls \hyperlink{le__messaging_8h_aa3cf113b26b154697ccef270dafe8798}{le\+\_\+msg\+\_\+\+Request\+Sync\+Response()}, it will return an error immediately, instead of blocking the thread. If the thread were blocked in this scenario, the server would also be blocked and would therefore be unable to receive the request and respond to it, resulting in a deadlock.
\end{DoxyWarning}
When the client is finished with it, the {\bfseries  client must release its reference to the response message } by calling \hyperlink{le__messaging_8h_afc508c24d0b6933e8fbc4e0410d50271}{le\+\_\+msg\+\_\+\+Release\+Msg()}.


\begin{DoxyCode}
\hyperlink{le__messaging_8h_afc508c24d0b6933e8fbc4e0410d50271}{le\_msg\_ReleaseMsg}(responseMsgRef);
\end{DoxyCode}
\hypertarget{c_messaging_c_messagingClientReceiving}{}\subsubsection{Receiving a Non-\/\+Response Message}\label{c_messaging_c_messagingClientReceiving}
When a server sends a message to the client that is not a response to a request from the client, that non-\/response message will be passed to the receive handler that the client registered using \hyperlink{le__messaging_8h_ac726cc93219d326e1b10a7d13a0f4f65}{le\+\_\+msg\+\_\+\+Set\+Session\+Recv\+Handler()}. In fact, this is the only kind of message that will result in that receive handler being called.

\begin{DoxyNote}{Note}
Some protocols don\textquotesingle{}t include any messages that are not responses to client requests, which is why it is optional to register a receive handler on the client side.
\end{DoxyNote}
The payload of a received message can be accessed using \hyperlink{le__messaging_8h_a32d1c7ffd913db8546f6f1bd5cce58c4}{le\+\_\+msg\+\_\+\+Get\+Payload\+Ptr()}, and the client can check what session the message arrived through by calling \hyperlink{le__messaging_8h_a253088f1b852575b60d7732ca7afc79b}{le\+\_\+msg\+\_\+\+Get\+Session()}.

When the client is finished with the message, the {\bfseries  client must release its reference to the message } by calling \hyperlink{le__messaging_8h_afc508c24d0b6933e8fbc4e0410d50271}{le\+\_\+msg\+\_\+\+Release\+Msg()}.


\begin{DoxyCode}
\textcolor{comment}{// Function will be called whenever the server sends us a notification message.}
\textcolor{keyword}{static} \textcolor{keywordtype}{void} NotifyHandlerFunc
(
    \hyperlink{le__messaging_8h_a1e5c37fdd50a4d6d24cad82cb166f770}{le\_msg\_MessageRef\_t}  msgRef,    \textcolor{comment}{// Reference to the received message.}
    \textcolor{keywordtype}{void}*                contextPtr \textcolor{comment}{// contextPtr passed into le\_msg\_SetSessionRecvHandler().}
)
\{
    \textcolor{comment}{// Process notification message from the server.}
    myproto\_Msg\_t* msgPayloadPtr = \hyperlink{le__messaging_8h_a32d1c7ffd913db8546f6f1bd5cce58c4}{le\_msg\_GetPayloadPtr}(msgRef);
    ...

    \textcolor{comment}{// Release the message, now that we are finished with it.}
    \hyperlink{le__messaging_8h_afc508c24d0b6933e8fbc4e0410d50271}{le\_msg\_ReleaseMsg}(msgRef);
\}

\hyperlink{le__event_loop_8h_abdb9187a56836a93d19cc793cbd4b7ec}{COMPONENT\_INIT}
\{
    \hyperlink{le__messaging_8h_ac05e9b3268f8fb5776adab6fe11410e5}{le\_msg\_ProtocolRef\_t} protocolRef;
    \hyperlink{le__messaging_8h_aebfc01e15b430a5b4f3038a5bd518904}{le\_msg\_SessionRef\_t} sessionRef;

    protocolRef = \hyperlink{le__messaging_8h_adcd1ff1a6906433aaa6d7038125c4473}{le\_msg\_GetProtocolRef}(PROTOCOL\_ID, \textcolor{keyword}{sizeof}(myproto\_Msg\_t));
    sessionRef = \hyperlink{le__messaging_8h_a696d7c2d4e3725d3ddb5dd2d79d2d732}{le\_msg\_CreateSession}(protocolRef, MY\_INTERFACE\_NAME);
    \hyperlink{le__messaging_8h_ac726cc93219d326e1b10a7d13a0f4f65}{le\_msg\_SetSessionRecvHandler}(sessionRef, NotifyHandlerFunc, NULL);
    \hyperlink{le__messaging_8h_a574d37960a07c4fc2bde310408619cff}{le\_msg\_OpenSession}(sessionRef, SessionOpenHandlerFunc, NULL);
\}
\end{DoxyCode}
\hypertarget{c_messaging_c_messagingClientClosing}{}\subsubsection{Closing Sessions}\label{c_messaging_c_messagingClientClosing}
When the client is done using a service, it can close the session using \hyperlink{le__messaging_8h_a1af0671de74160d99be5bfe212c39369}{le\+\_\+msg\+\_\+\+Close\+Session()}. This will leave the session object in existence, though, so that it can be opened again using \hyperlink{le__messaging_8h_a574d37960a07c4fc2bde310408619cff}{le\+\_\+msg\+\_\+\+Open\+Session()}.


\begin{DoxyCode}
\hyperlink{le__messaging_8h_a1af0671de74160d99be5bfe212c39369}{le\_msg\_CloseSession}(sessionRef);
\end{DoxyCode}


To delete a session object, call \hyperlink{le__messaging_8h_a8d950c4b07741177d2c0927c31e3e29f}{le\+\_\+msg\+\_\+\+Delete\+Session()}. This will automatically close the session, if it is still open (but won\textquotesingle{}t automatically delete any messages).


\begin{DoxyCode}
\hyperlink{le__messaging_8h_a8d950c4b07741177d2c0927c31e3e29f}{le\_msg\_DeleteSession}(sessionRef);
\end{DoxyCode}


\begin{DoxyNote}{Note}
If a client process dies while it has a session open, that session will be automatically closed and deleted by the Legato framework, so there\textquotesingle{}s no need to register process clean-\/up handlers or anything like that for this purpose.
\end{DoxyNote}
Additionally, clients can choose to call \hyperlink{le__messaging_8h_a981b1b0714abba85efc19293ac6d2744}{le\+\_\+msg\+\_\+\+Set\+Session\+Close\+Handler()} to register to be notified when a session gets closed by the server. Servers often keep state on behalf of their clients, and if the server closes the session (or if the system closes the session because the server died), the client most likely will still be operating under the assumption (now false) that the server is maintaining state on its behalf. If a client is designed to recover from the server losing its state, the client can register a close handler and handle the close.


\begin{DoxyCode}
\hyperlink{le__messaging_8h_a981b1b0714abba85efc19293ac6d2744}{le\_msg\_SetSessionCloseHandler}(sessionRef, SessionCloseHandler, NULL);
\end{DoxyCode}


However, most clients are not designed to recover from their session being closed by someone else, so if a close handler is not registered by a client and the session closes for some reason other than the client calling \hyperlink{le__messaging_8h_a1af0671de74160d99be5bfe212c39369}{le\+\_\+msg\+\_\+\+Close\+Session()}, then the client process will be terminated.

\begin{DoxyNote}{Note}
If the client closes the session, the client-\/side session close handler will not be called, even if one is registered.
\end{DoxyNote}
\hypertarget{c_messaging_c_messagingClientMultithreading}{}\subsubsection{Multithreading}\label{c_messaging_c_messagingClientMultithreading}
The Low-\/\+Level Messaging A\+P\+I is thread safe, but not async safe.

When a client creates a session, that session gets \char`\"{}attached\char`\"{} to the thread that created it (i.\+e., the thread that called \hyperlink{le__messaging_8h_a696d7c2d4e3725d3ddb5dd2d79d2d732}{le\+\_\+msg\+\_\+\+Create\+Session()}). That thread will then call any callbacks registered for that session.

Note that this implies that if the client thread that creates the session does not run the Legato event loop then no callbacks will ever be called for that session. To work around this, move the session creation to another thread that that uses the Legato event loop.

Furthermore, to prevent race conditions, only the thread that is attached to a given session is allowed to call \hyperlink{le__messaging_8h_aa3cf113b26b154697ccef270dafe8798}{le\+\_\+msg\+\_\+\+Request\+Sync\+Response()} for that session.\hypertarget{c_messaging_c_messagingClientExample}{}\subsubsection{Sample Code}\label{c_messaging_c_messagingClientExample}

\begin{DoxyCode}
\textcolor{comment}{// Function will be called whenever the server sends us a notification message.}
\textcolor{keyword}{static} \textcolor{keywordtype}{void} NotifyHandlerFunc
(
    \hyperlink{le__messaging_8h_a1e5c37fdd50a4d6d24cad82cb166f770}{le\_msg\_MessageRef\_t}  msgRef,    \textcolor{comment}{// Reference to the received message.}
    \textcolor{keywordtype}{void}*                contextPtr \textcolor{comment}{// contextPtr passed into le\_msg\_SetSessionRecvHandler().}
)
\{
    \textcolor{comment}{// Process notification message from the server.}
    myproto\_Msg\_t* msgPayloadPtr = \hyperlink{le__messaging_8h_a32d1c7ffd913db8546f6f1bd5cce58c4}{le\_msg\_GetPayloadPtr}(msgRef);
    ...

    \textcolor{comment}{// Release the message, now that we are finished with it.}
    \hyperlink{le__messaging_8h_afc508c24d0b6933e8fbc4e0410d50271}{le\_msg\_ReleaseMsg}(msgRef);
\}

\textcolor{comment}{// Function will be called whenever the server sends us a response message or our}
\textcolor{comment}{// request-response transaction fails.}
\textcolor{keyword}{static} \textcolor{keywordtype}{void} ResponseHandlerFunc
(
    \hyperlink{le__messaging_8h_a1e5c37fdd50a4d6d24cad82cb166f770}{le\_msg\_MessageRef\_t}  msgRef,    \textcolor{comment}{// Reference to response message (NULL if
       transaction failed).}
    \textcolor{keywordtype}{void}*                contextPtr \textcolor{comment}{// contextPtr passed into le\_msg\_RequestResponse().}
)
\{
    \textcolor{comment}{// Check if we got a response.}
    \textcolor{keywordflow}{if} (msgRef == NULL)
    \{
        \textcolor{comment}{// Transaction failed.  No response received.}
        \textcolor{comment}{// This might happen if the server deleted the request without sending a response,}
        \textcolor{comment}{// or if we had registered a "Session End Handler" and the session terminated before}
        \textcolor{comment}{// the response was sent.}
        \hyperlink{le__log_8h_a353590f91b3143a7ba3a416ae5a50c3d}{LE\_ERROR}(\textcolor{stringliteral}{"Transaction failed!"});
    \}
    \textcolor{keywordflow}{else}
    \{
        \textcolor{comment}{// Process response message from the server.}
        myproto\_Msg\_t* msgPayloadPtr = \hyperlink{le__messaging_8h_a32d1c7ffd913db8546f6f1bd5cce58c4}{le\_msg\_GetPayloadPtr}(msgRef);
        ...

        \textcolor{comment}{// Release the response message, now that we are finished with it.}
        \hyperlink{le__messaging_8h_afc508c24d0b6933e8fbc4e0410d50271}{le\_msg\_ReleaseMsg}(msgRef);
    \}
\}

\textcolor{comment}{// Function will be called when the client-server session opens.}
\textcolor{keyword}{static} \textcolor{keywordtype}{void} SessionOpenHandlerFunc
(
    \hyperlink{le__messaging_8h_aebfc01e15b430a5b4f3038a5bd518904}{le\_msg\_SessionRef\_t}  sessionRef, \textcolor{comment}{// Reference tp the session that opened.}
    \textcolor{keywordtype}{void}*                contextPtr  \textcolor{comment}{// contextPtr passed into le\_msg\_OpenSession().}
)
\{
    \hyperlink{le__messaging_8h_a1e5c37fdd50a4d6d24cad82cb166f770}{le\_msg\_MessageRef\_t} msgRef;
    myproto\_Msg\_t* msgPayloadPtr;

    \textcolor{comment}{// Send a request to the server.}
    msgRef = \hyperlink{le__messaging_8h_a8293a69f256b98cbce5b9990ea3520f3}{le\_msg\_CreateMsg}(sessionRef);
    msgPayloadPtr = \hyperlink{le__messaging_8h_a32d1c7ffd913db8546f6f1bd5cce58c4}{le\_msg\_GetPayloadPtr}(msgRef);
    msgPayloadPtr->... = ...; \textcolor{comment}{// <-- Populate message payload...}
    \hyperlink{le__messaging_8h_a5440ae06a89b60ed04e9de5601496608}{le\_msg\_RequestResponse}(msgRef, ResponseHandlerFunc, NULL);
\}

\hyperlink{le__event_loop_8h_abdb9187a56836a93d19cc793cbd4b7ec}{COMPONENT\_INIT}
\{
    \hyperlink{le__messaging_8h_ac05e9b3268f8fb5776adab6fe11410e5}{le\_msg\_ProtocolRef\_t} protocolRef;
    \hyperlink{le__messaging_8h_aebfc01e15b430a5b4f3038a5bd518904}{le\_msg\_SessionRef\_t} sessionRef;

    \textcolor{comment}{// Open a session.}
    protocolRef = \hyperlink{le__messaging_8h_adcd1ff1a6906433aaa6d7038125c4473}{le\_msg\_GetProtocolRef}(PROTOCOL\_ID, \textcolor{keyword}{sizeof}(myproto\_Msg\_t));
    sessionRef = \hyperlink{le__messaging_8h_a696d7c2d4e3725d3ddb5dd2d79d2d732}{le\_msg\_CreateSession}(protocolRef, MY\_INTERFACE\_NAME);
    \hyperlink{le__messaging_8h_ac726cc93219d326e1b10a7d13a0f4f65}{le\_msg\_SetSessionRecvHandler}(sessionRef, NotifyHandlerFunc, NULL);
    \hyperlink{le__messaging_8h_a574d37960a07c4fc2bde310408619cff}{le\_msg\_OpenSession}(sessionRef, SessionOpenHandlerFunc, NULL);
\}
\end{DoxyCode}
\hypertarget{c_messaging_c_messagingServerUsage}{}\subsection{Server Usage Model}\label{c_messaging_c_messagingServerUsage}
\hyperlink{c_messaging_c_messagingServerProcessingMessages}{Processing Messages from Clients} ~\newline
 \hyperlink{c_messaging_c_messagingServerSendingNonResponse}{Sending Non-\/\+Response Messages to Clients} ~\newline
 \hyperlink{c_messaging_c_messagingServerCleanUp}{Cleaning up when Sessions Close} ~\newline
 \hyperlink{c_messaging_c_messagingRemovingService}{Removing Service} ~\newline
 \hyperlink{c_messaging_c_messagingServerMultithreading}{Multithreading} ~\newline
 \hyperlink{c_messaging_c_messagingServerExample}{Sample Code}

Servers that wish to offer a service do the following\+:
\begin{DoxyEnumerate}
\item Get a reference to the protocol they want to use by calling \hyperlink{le__messaging_8h_adcd1ff1a6906433aaa6d7038125c4473}{le\+\_\+msg\+\_\+\+Get\+Protocol\+Ref()}.
\item Create a Service object using \hyperlink{le__messaging_8h_adbbb2737069b636028128c74ae407742}{le\+\_\+msg\+\_\+\+Create\+Service()}, passing in the protocol reference and the service name.
\item Call \hyperlink{le__messaging_8h_a21c2e57ad1ffbbdfedd3987c468e3130}{le\+\_\+msg\+\_\+\+Set\+Service\+Recv\+Handler()} to register a function to handle messages received from clients.
\item Advertise the service using \hyperlink{le__messaging_8h_ad3ff11d1962840f879d9c8fe7054de0c}{le\+\_\+msg\+\_\+\+Advertise\+Service()}.
\end{DoxyEnumerate}


\begin{DoxyCode}
protocolRef = \hyperlink{le__messaging_8h_adcd1ff1a6906433aaa6d7038125c4473}{le\_msg\_GetProtocolRef}(PROTOCOL\_ID, \textcolor{keyword}{sizeof}(myproto\_Msg\_t));
serviceRef = \hyperlink{le__messaging_8h_adbbb2737069b636028128c74ae407742}{le\_msg\_CreateService}(protocolRef, SERVER\_INTERFACE\_NAME);
\hyperlink{le__messaging_8h_a21c2e57ad1ffbbdfedd3987c468e3130}{le\_msg\_SetServiceRecvHandler}(serviceRef, RequestMsgHandlerFunc, NULL);
\hyperlink{le__messaging_8h_ad3ff11d1962840f879d9c8fe7054de0c}{le\_msg\_AdvertiseService}(serviceRef);
\end{DoxyCode}


Once the service is advertised, clients can open it and start sending it messages. The server will receive messages via callbacks to the function it registered using \hyperlink{le__messaging_8h_a21c2e57ad1ffbbdfedd3987c468e3130}{le\+\_\+msg\+\_\+\+Set\+Service\+Recv\+Handler()}.

Servers also have the option of being notified when sessions are opened by clients. They get this notification by registering a handler function using \hyperlink{le__messaging_8h_a829d6450d487166e0b2994b4bf44ee5d}{le\+\_\+msg\+\_\+\+Add\+Service\+Open\+Handler()}.


\begin{DoxyCode}
\textcolor{comment}{// Function will be called whenever a client opens a session with our service.}
\textcolor{keyword}{static} \textcolor{keywordtype}{void} SessionOpenHandlerFunc
(
    \hyperlink{le__messaging_8h_aebfc01e15b430a5b4f3038a5bd518904}{le\_msg\_SessionRef\_t}  sessionRef, 
    \textcolor{keywordtype}{void}*                contextPtr  
)
\{
    \textcolor{comment}{// Handle new session opening...}
    ...
\}

\hyperlink{le__event_loop_8h_abdb9187a56836a93d19cc793cbd4b7ec}{COMPONENT\_INIT}
\{
    \hyperlink{le__messaging_8h_ac05e9b3268f8fb5776adab6fe11410e5}{le\_msg\_ProtocolRef\_t} protocolRef;
    \hyperlink{le__messaging_8h_ad9f0b13cde1d8c1eab5318dbcf0d9e28}{le\_msg\_ServiceRef\_t} serviceRef;

    \textcolor{comment}{// Create my service and advertise it.}
    protocolRef = \hyperlink{le__messaging_8h_adcd1ff1a6906433aaa6d7038125c4473}{le\_msg\_GetProtocolRef}(PROTOCOL\_ID, \textcolor{keyword}{sizeof}(myproto\_Msg\_t));
    serviceRef = \hyperlink{le__messaging_8h_adbbb2737069b636028128c74ae407742}{le\_msg\_CreateService}(protocolRef, SERVER\_INTERFACE\_NAME);
    \hyperlink{le__messaging_8h_a829d6450d487166e0b2994b4bf44ee5d}{le\_msg\_AddServiceOpenHandler}(serviceRef, SessionOpenHandlerFunc, NULL);
    \hyperlink{le__messaging_8h_ad3ff11d1962840f879d9c8fe7054de0c}{le\_msg\_AdvertiseService}(serviceRef);
\}
\end{DoxyCode}


Both the \char`\"{}\+Open Handler\char`\"{} and the \char`\"{}\+Receive Handler\char`\"{} will be called by the Legato event loop in the thread that registered those handlers (which must also be the same thread that created the service).\hypertarget{c_messaging_c_messagingServerProcessingMessages}{}\subsubsection{Processing Messages from Clients}\label{c_messaging_c_messagingServerProcessingMessages}
The payload of any received message can be accessed using \hyperlink{le__messaging_8h_a32d1c7ffd913db8546f6f1bd5cce58c4}{le\+\_\+msg\+\_\+\+Get\+Payload\+Ptr()}.

If a received message does not require a response (i.\+e., if the client sent it using \hyperlink{le__messaging_8h_a073de097d281475c44a445b927fbb929}{le\+\_\+msg\+\_\+\+Send()}), then when the server is finished with the message, the server must release the message by calling \hyperlink{le__messaging_8h_afc508c24d0b6933e8fbc4e0410d50271}{le\+\_\+msg\+\_\+\+Release\+Msg()}.


\begin{DoxyCode}
\textcolor{keywordtype}{void} RequestMsgHandlerFunc
(
    \hyperlink{le__messaging_8h_a1e5c37fdd50a4d6d24cad82cb166f770}{le\_msg\_MessageRef\_t} msgRef  \textcolor{comment}{// Reference to the received message.}
)
\{
    myproto\_Msg\_t* msgPtr = \hyperlink{le__messaging_8h_a32d1c7ffd913db8546f6f1bd5cce58c4}{le\_msg\_GetPayloadPtr}(msgRef);
    \hyperlink{le__log_8h_a23e6d206faa64f612045d688cdde5808}{LE\_INFO}(\textcolor{stringliteral}{"Received request '%s'"}, msgPtr->request.string);

    \textcolor{comment}{// No response required and I'm done with this message, so release it.}
    \hyperlink{le__messaging_8h_afc508c24d0b6933e8fbc4e0410d50271}{le\_msg\_ReleaseMsg}(msgRef);
\}
\end{DoxyCode}


If a received message requires a response (i.\+e., if the client sent it using \hyperlink{le__messaging_8h_a5440ae06a89b60ed04e9de5601496608}{le\+\_\+msg\+\_\+\+Request\+Response()} or \hyperlink{le__messaging_8h_aa3cf113b26b154697ccef270dafe8798}{le\+\_\+msg\+\_\+\+Request\+Sync\+Response()}), the server must eventually respond to that message by calling \hyperlink{le__messaging_8h_ac4545833ad2da9bb4cc5d18125f7d9f2}{le\+\_\+msg\+\_\+\+Respond()} on that message. \hyperlink{le__messaging_8h_ac4545833ad2da9bb4cc5d18125f7d9f2}{le\+\_\+msg\+\_\+\+Respond()} sends the message back to the client that sent the request. The response payload is stored inside the same payload buffer that contained the request payload.

To do this, the request payload pointer can be cast to a pointer to the response payload structure, and then the response payload can be written into it.


\begin{DoxyCode}
\textcolor{keywordtype}{void} RequestMsgHandlerFunc
(
    \hyperlink{le__messaging_8h_a1e5c37fdd50a4d6d24cad82cb166f770}{le\_msg\_MessageRef\_t} msgRef  \textcolor{comment}{// Reference to the received message.}
)
\{
    myproto\_RequestMsg\_t* requestPtr = \hyperlink{le__messaging_8h_a32d1c7ffd913db8546f6f1bd5cce58c4}{le\_msg\_GetPayloadPtr}(msgRef);
    myproto\_ResponseMsg\_t* responsePtr;

    \hyperlink{le__log_8h_a23e6d206faa64f612045d688cdde5808}{LE\_INFO}(\textcolor{stringliteral}{"Received request '%s'"}, requestPtr->string);

    responsePtr = (myproto\_ResponseMsg\_t*)requestPtr;
    responsePtr->value = Value;
    \hyperlink{le__messaging_8h_ac4545833ad2da9bb4cc5d18125f7d9f2}{le\_msg\_Respond}(msgRef);
\}
\end{DoxyCode}


Alternatively, the request payload structure and the response payload structure could be placed into a union together.


\begin{DoxyCode}
\textcolor{keyword}{typedef} \textcolor{keyword}{union}
\{
    myproto\_Request\_t request;
    myproto\_Response\_t response;
\}
myproto\_Msg\_t;

 ...

void RequestMsgHandlerFunc
(
    \hyperlink{le__messaging_8h_a1e5c37fdd50a4d6d24cad82cb166f770}{le\_msg\_MessageRef\_t} msgRef  \textcolor{comment}{// Reference to the received message.}
)
\{
    myproto\_Msg\_t* msgPtr = \hyperlink{le__messaging_8h_a32d1c7ffd913db8546f6f1bd5cce58c4}{le\_msg\_GetPayloadPtr}(msgRef);
    \hyperlink{le__log_8h_a23e6d206faa64f612045d688cdde5808}{LE\_INFO}(\textcolor{stringliteral}{"Received request '%s'"}, msgPtr->request.string);
    msgPtr->response.value = Value;
    \hyperlink{le__messaging_8h_ac4545833ad2da9bb4cc5d18125f7d9f2}{le\_msg\_Respond}(msgRef);
\}
\end{DoxyCode}


\begin{DoxyWarning}{Warning}
Of course, once you\textquotesingle{}ve started writing the response payload into the buffer, the request payload is no longer available, so if you still need it, copy it somewhere else first.
\end{DoxyWarning}
\begin{DoxyNote}{Note}
The server doesn\textquotesingle{}t have to send the response back to the client right away. It could hold onto the request for an indefinite amount of time, for whatever reason.
\end{DoxyNote}
Whenever any message is received from a client, the message is associated with the session through which the client sent it. A reference to the session can be retrieved from the message, if needed, by calling \hyperlink{le__messaging_8h_a253088f1b852575b60d7732ca7afc79b}{le\+\_\+msg\+\_\+\+Get\+Session()}. This can be handy for tagging things in the server\textquotesingle{}s internal data structures that need to be cleaned up when the client closes the session (see \hyperlink{c_messaging_c_messagingServerCleanUp}{Cleaning up when Sessions Close} for more on this).

The function \hyperlink{le__messaging_8h_a8b0824781c030b9ed6f8be48e4d20419}{le\+\_\+msg\+\_\+\+Needs\+Response()} can be used to check if a received message requires a response or not.\hypertarget{c_messaging_c_messagingServerSendingNonResponse}{}\subsubsection{Sending Non-\/\+Response Messages to Clients}\label{c_messaging_c_messagingServerSendingNonResponse}
If a server wants to send a non-\/response message to a client, it first needs a reference to the session that client opened. It could have got the session reference from a previous message received from the client (by calling \hyperlink{le__messaging_8h_a253088f1b852575b60d7732ca7afc79b}{le\+\_\+msg\+\_\+\+Get\+Session()} on that message). Or, it could have got the session reference from a Session Open Handler callback (see \hyperlink{le__messaging_8h_a829d6450d487166e0b2994b4bf44ee5d}{le\+\_\+msg\+\_\+\+Add\+Service\+Open\+Handler()}). Either way, once it has the session reference, it can call \hyperlink{le__messaging_8h_a8293a69f256b98cbce5b9990ea3520f3}{le\+\_\+msg\+\_\+\+Create\+Msg()} to create a message from that session\textquotesingle{}s server-\/side message pool. The message can then be populated and sent in the same way that a client would send a message to the server using \hyperlink{le__messaging_8h_a32d1c7ffd913db8546f6f1bd5cce58c4}{le\+\_\+msg\+\_\+\+Get\+Payload\+Ptr()} and \hyperlink{le__messaging_8h_a073de097d281475c44a445b927fbb929}{le\+\_\+msg\+\_\+\+Send()}.


\begin{DoxyCode}
\textcolor{comment}{// Function will be called whenever a client opens a session with our service.}
\textcolor{keyword}{static} \textcolor{keywordtype}{void} SessionOpenHandlerFunc
(
    \hyperlink{le__messaging_8h_aebfc01e15b430a5b4f3038a5bd518904}{le\_msg\_SessionRef\_t}  sessionRef, 
    \textcolor{keywordtype}{void}*                contextPtr  
)
\{
    \hyperlink{le__messaging_8h_a1e5c37fdd50a4d6d24cad82cb166f770}{le\_msg\_MessageRef\_t} msgRef;
    myproto\_Msg\_t* msgPayloadPtr;

    \textcolor{comment}{// Send a "Welcome" message to the client.}
    msgRef = \hyperlink{le__messaging_8h_a8293a69f256b98cbce5b9990ea3520f3}{le\_msg\_CreateMsg}(sessionRef);
    msgPayloadPtr = \hyperlink{le__messaging_8h_a32d1c7ffd913db8546f6f1bd5cce58c4}{le\_msg\_GetPayloadPtr}(msgRef);
    msgPayloadPtr->... = ...; \textcolor{comment}{// <-- Populate message payload...}
    \hyperlink{le__messaging_8h_a073de097d281475c44a445b927fbb929}{le\_msg\_Send}(msgRef);
\}

\hyperlink{le__event_loop_8h_abdb9187a56836a93d19cc793cbd4b7ec}{COMPONENT\_INIT}
\{
    \hyperlink{le__messaging_8h_ac05e9b3268f8fb5776adab6fe11410e5}{le\_msg\_ProtocolRef\_t} protocolRef;
    \hyperlink{le__messaging_8h_ad9f0b13cde1d8c1eab5318dbcf0d9e28}{le\_msg\_ServiceRef\_t} serviceRef;

    \textcolor{comment}{// Create my service and advertise it.}
    protocolRef = \hyperlink{le__messaging_8h_adcd1ff1a6906433aaa6d7038125c4473}{le\_msg\_GetProtocolRef}(PROTOCOL\_ID, \textcolor{keyword}{sizeof}(myproto\_Msg\_t));
    serviceRef = \hyperlink{le__messaging_8h_adbbb2737069b636028128c74ae407742}{le\_msg\_CreateService}(protocolRef, SERVER\_INTERFACE\_NAME);
    \hyperlink{le__messaging_8h_a829d6450d487166e0b2994b4bf44ee5d}{le\_msg\_AddServiceOpenHandler}(serviceRef, SessionOpenHandlerFunc, NULL);
    \hyperlink{le__messaging_8h_ad3ff11d1962840f879d9c8fe7054de0c}{le\_msg\_AdvertiseService}(serviceRef);
\}
\end{DoxyCode}
\hypertarget{c_messaging_c_messagingServerCleanUp}{}\subsubsection{Cleaning up when Sessions Close}\label{c_messaging_c_messagingServerCleanUp}
If a server keeps state on behalf of its clients, it can call \hyperlink{le__messaging_8h_a426dfbae396599d80e52902165368907}{le\+\_\+msg\+\_\+\+Add\+Service\+Close\+Handler()} to ask to be notified when clients close sessions with a given service. This allows the server to clean up any state associated with a given session when the client closes that session (or when the system closes the session because the client died). The close handler is passed a session reference, so the server can check its internal data structures and clean up anything that it has previously tagged with that same session reference.

\begin{DoxyNote}{Note}
Servers don\textquotesingle{}t delete sessions. On the server side, sessions are automatically deleted when they close.
\end{DoxyNote}
\hypertarget{c_messaging_c_messagingRemovingService}{}\subsubsection{Removing Service}\label{c_messaging_c_messagingRemovingService}
If a server wants to stop offering a service, it can hide the service by calling \hyperlink{le__messaging_8h_a38cc9dec16a758c5262d8bb5c7a2e57f}{le\+\_\+msg\+\_\+\+Hide\+Service()}. This will not terminate any sessions that are already open, but it will prevent clients from opening new sessions until it is advertised again.

\begin{DoxyWarning}{Warning}
Watch out for race conditions here. It\textquotesingle{}s possible that a client is in the process of opening a session when you decide to hide your service. In this case, a new session may open after you hid the service. Be prepared to handle that gracefully.
\end{DoxyWarning}
The server also has the option to delete the service. This hides the service and closes all open sessions.

If a server process dies, the Legato framework will automatically delete all of its services.\hypertarget{c_messaging_c_messagingServerMultithreading}{}\subsubsection{Multithreading}\label{c_messaging_c_messagingServerMultithreading}
The Low-\/\+Level Messaging A\+P\+I is thread safe, but not async safe.

When a server creates a service, that service gets attached to the thread that created it (i.\+e., the thread that called \hyperlink{le__messaging_8h_adbbb2737069b636028128c74ae407742}{le\+\_\+msg\+\_\+\+Create\+Service()}). That thread will call any handler functions registered for that service.

Note that this implies that if the thread that creates the service does not run the Legato event loop, then no callbacks will ever be called for that service. To work around this, you could move the service to another thread that that runs the Legato event loop.\hypertarget{c_messaging_c_messagingServerExample}{}\subsubsection{Sample Code}\label{c_messaging_c_messagingServerExample}

\begin{DoxyCode}
\textcolor{keywordtype}{void} RequestMsgHandlerFunc
(
    \hyperlink{le__messaging_8h_a1e5c37fdd50a4d6d24cad82cb166f770}{le\_msg\_MessageRef\_t} msgRef,     \textcolor{comment}{// Reference to the received message.}
    \textcolor{keywordtype}{void}*               contextPtr  
)
\{
    \textcolor{comment}{// Check the message type to decide what to do.}
    myproto\_Msg\_t* msgPtr = \hyperlink{le__messaging_8h_a32d1c7ffd913db8546f6f1bd5cce58c4}{le\_msg\_GetPayloadPtr}(msgRef);
    \textcolor{keywordflow}{switch} (msgPtr->type)
    \{
        \textcolor{keywordflow}{case} MYPROTO\_MSG\_TYPE\_SET\_VALUE:
             \textcolor{comment}{// Message does not require a response.}
             Value = msgPtr->...;
             \hyperlink{le__messaging_8h_afc508c24d0b6933e8fbc4e0410d50271}{le\_msg\_ReleaseMsg}(msgRef);
             \textcolor{keywordflow}{break};

        \textcolor{keywordflow}{case} MYPROTO\_MSG\_TYPE\_GET\_VALUE:
             \textcolor{comment}{// Message is a request that requires a response.}
             \textcolor{comment}{// Notice that we just reuse the request message buffer for the response.}
             msgPtr->... = Value;
             \hyperlink{le__messaging_8h_ac4545833ad2da9bb4cc5d18125f7d9f2}{le\_msg\_Respond}(msgRef);
             \textcolor{keywordflow}{break};

        \textcolor{keywordflow}{default}:
             \textcolor{comment}{// Unexpected message type!}
             \hyperlink{le__log_8h_a353590f91b3143a7ba3a416ae5a50c3d}{LE\_ERROR}(\textcolor{stringliteral}{"Received unexpected message type %d from session %s."},
                      msgPtr->type,
                      \hyperlink{le__messaging_8h_a32e952eda728d1a1b4236d879bfc05f9}{le\_msg\_GetInterfaceName}(
      \hyperlink{le__messaging_8h_a708715b598d1ecbe3f62f52f5955a61c}{le\_msg\_GetSessionInterface}(\hyperlink{le__messaging_8h_a253088f1b852575b60d7732ca7afc79b}{le\_msg\_GetSession}(msgRef))));
             \hyperlink{le__messaging_8h_afc508c24d0b6933e8fbc4e0410d50271}{le\_msg\_ReleaseMsg}(msgRef);
    \}
\}

\hyperlink{le__event_loop_8h_abdb9187a56836a93d19cc793cbd4b7ec}{COMPONENT\_INIT}
\{
    \hyperlink{le__messaging_8h_ac05e9b3268f8fb5776adab6fe11410e5}{le\_msg\_ProtocolRef\_t} protocolRef;
    \hyperlink{le__messaging_8h_ad9f0b13cde1d8c1eab5318dbcf0d9e28}{le\_msg\_ServiceRef\_t} serviceRef;

    \textcolor{comment}{// Create my service and advertise it.}
    protocolRef = \hyperlink{le__messaging_8h_adcd1ff1a6906433aaa6d7038125c4473}{le\_msg\_GetProtocolRef}(PROTOCOL\_ID, \textcolor{keyword}{sizeof}(myproto\_Msg\_t));
    serviceRef = \hyperlink{le__messaging_8h_adbbb2737069b636028128c74ae407742}{le\_msg\_CreateService}(protocolRef, SERVER\_INTERFACE\_NAME);
    \hyperlink{le__messaging_8h_a21c2e57ad1ffbbdfedd3987c468e3130}{le\_msg\_SetServiceRecvHandler}(serviceRef, RequestMsgHandlerFunc, NULL);
    \hyperlink{le__messaging_8h_ad3ff11d1962840f879d9c8fe7054de0c}{le\_msg\_AdvertiseService}(serviceRef);
\}
\end{DoxyCode}
\hypertarget{c_messaging_c_messagingStartUp}{}\subsection{Start Up Sequencing}\label{c_messaging_c_messagingStartUp}
Worthy of special mention is the fact that the low-\/level messaging system can be used to solve the age-\/old problem of coordinating the start-\/up sequence of processes that interact with each other. Far too often, the start-\/up sequence of multiple interacting processes is addressed using hacks like polling or sleeping for arbitrary lengths of time. These solutions can waste a lot of C\+P\+U cycles and battery power, slow down start-\/up, and (in the case of arbitrary sleeps) introduce race conditions that can cause failures in the field.

In Legato, a messaging client can attempt to open a session before the server process has even started. The client will be notified asynchronously (via callback) when the server advertises its service.

In this way, clients are guaranteed to wait for the servers they use, without the inefficiency of polling, and without having to add code elsewhere to coordinate the start-\/up sequence. Furthermore, if there is work that needs to be done by the client at start-\/up before it opens a session with the server, the client is allowed to do that work in parallel with the start-\/up of the server, so the C\+P\+U can be more fully utilized to shorten the overall duration of the start-\/up sequence.\hypertarget{c_messaging_c_messagingMemoryManagement}{}\subsection{Memory Management}\label{c_messaging_c_messagingMemoryManagement}
Message buffer memory is allocated and controlled behind the scenes, inside the Messaging A\+P\+I. This allows the Messaging A\+P\+I to
\begin{DoxyItemize}
\item take some steps to remove programmer pitfalls,
\item provide some built-\/in remote troubleshooting features
\item encapsulate the I\+P\+C implementation, allowing for future optimization and porting.
\end{DoxyItemize}

Each message object is allocated from a session. The sessions\textquotesingle{} message pool sizes can be tuned through component and application configuration files and device configuration settings.

Generally speaking, message payload sizes are determined by the protocol that is being used. Application protocols and the packing of messages into message buffers are the domain of higher-\/layers of the software stack. But, at this low layer, servers and clients just declare the name and version of the protocol, and the size of the largest message in the protocol. From this, they obtain a protocol reference that they provide to sessions when they create them.\hypertarget{c_messaging_c_messagingSecurity}{}\subsection{Security}\label{c_messaging_c_messagingSecurity}
Security is provided in the form of authentication and access control.

Clients cannot open sessions with servers until their client-\/side interface is \char`\"{}bound\char`\"{} to a server-\/side interface (service). The binding thereby provides configuration of both routing and access control.

Neither the client-\/side nor the server-\/side I\+P\+C sockets are named. Therefore, no process other than the Service Directory has access to these sockets. The Service Directory passes client connections to the appropriate server based on the binding configuration of the client\textquotesingle{}s interface.

The binding configuration is kept in the \char`\"{}system\char`\"{} configuration tree, so clients that do not have write access to the \char`\"{}system\char`\"{} configuration tree have no control over their own binding configuration. By default, sandboxed apps do not have any access (read or write) to the \char`\"{}system\char`\"{} configuration tree.\hypertarget{c_messaging_c_messagingClientUserIdChecking}{}\subsection{Client User I\+D Checking}\label{c_messaging_c_messagingClientUserIdChecking}
In rare cases, a server may wish to check the user I\+D of the remote client. Generally, this is not necessary because the I\+P\+C system enforces user-\/based access control restrictions automatically before allowing an I\+P\+C connection to be established. However, sometimes it may be useful when the service wishes to change the way it behaves, based on what user is connected to it.

\hyperlink{le__messaging_8h_a8c04f9cad0a768b4922a9987df84b65f}{le\+\_\+msg\+\_\+\+Get\+Client\+User\+Id()} can be used to fetch the user I\+D of the client at the far end of a given I\+P\+C session.


\begin{DoxyCode}
uid\_t clientUserId;
\textcolor{keywordflow}{if} (\hyperlink{le__messaging_8h_a8c04f9cad0a768b4922a9987df84b65f}{le\_msg\_GetClientUserId}(sessionRef, &clientUserId) != 
      \hyperlink{le__basics_8h_a1cca095ed6ebab24b57a636382a6c86ca5066a4bcec691c6b67843b8f79656422}{LE\_OK})
\{
    \textcolor{comment}{// The session must have closed.}
    ...
\}
\textcolor{keywordflow}{else}
\{
    \hyperlink{le__log_8h_a23e6d206faa64f612045d688cdde5808}{LE\_INFO}(\textcolor{stringliteral}{"My client has user ID %ud."}, clientUserId);
\}
\end{DoxyCode}
\hypertarget{c_messaging_c_messagingSendingFileDescriptors}{}\subsection{Sending File Descriptors}\label{c_messaging_c_messagingSendingFileDescriptors}
It is possible to send an open file descriptor through an I\+P\+C session by adding an fd to a message before sending it. On the sender\textquotesingle{}s side, \hyperlink{le__messaging_8h_a43459773ce8a9febf0f2e66681e40e91}{le\+\_\+msg\+\_\+\+Set\+Fd()} is used to set the file descriptor to be sent. On the receiver\textquotesingle{}s side, \hyperlink{le__messaging_8h_a1eff9dd7f93de58f5678ad3d6e0b734d}{le\+\_\+msg\+\_\+\+Get\+Fd()} is used to get the fd from the message.

The I\+P\+C A\+P\+I will close the original fd in the sender\textquotesingle{}s address space once it has been sent, so if the sender still needs the fd open on its side, it should duplicate the fd (e.\+g., using dup() ) before sending it.

On the receiving side, if the fd is not extracted from the message, it will be closed when the message is released. The fd can only be extracted from the message once. Subsequent calls to \hyperlink{le__messaging_8h_a1eff9dd7f93de58f5678ad3d6e0b734d}{le\+\_\+msg\+\_\+\+Get\+Fd()} will return -\/1.

As a denial-\/of-\/service prevention measure, receiving of file descriptors is disabled by default on servers. To enable receiving of file descriptors, the server must call le\+\_\+msg\+\_\+\+Enable\+Fd\+Reception() on their service.

\begin{DoxyWarning}{Warning}
D\+O N\+O\+T S\+E\+N\+D D\+I\+R\+E\+C\+T\+O\+R\+Y F\+I\+L\+E D\+E\+S\+C\+R\+I\+P\+T\+O\+R\+S. That can be exploited to break out of chroot() jails.
\end{DoxyWarning}
\hypertarget{c_messaging_c_messagingFutureEnhancements}{}\subsection{Future Enhancements}\label{c_messaging_c_messagingFutureEnhancements}
As an optimization to reduce the number of copies in cases where the sender of a message already has the message payload of their message assembled somewhere (perhaps as static data or in another message buffer received earlier from somewhere), a pointer to the payload could be passed to the message, instead of having to copy the payload into the message.


\begin{DoxyCode}
msgRef = \hyperlink{le__messaging_8h_a8293a69f256b98cbce5b9990ea3520f3}{le\_msg\_CreateMsg}(sessionRef);
le\_msg\_SetPayloadBuff(msgRef, &msgPayload, \textcolor{keyword}{sizeof}(msgPayload));
msgRef = \hyperlink{le__messaging_8h_a5440ae06a89b60ed04e9de5601496608}{le\_msg\_RequestResponse}(msgRef, ResponseHandlerFunc, contextPtr);
\end{DoxyCode}


Perhaps an \char`\"{}iovec\char`\"{} version could be added to do scatter-\/gather too?\hypertarget{c_messaging_c_messagingDesignNotes}{}\subsection{Design Notes}\label{c_messaging_c_messagingDesignNotes}
We explored the option of having asynchronous messages automatically released when their handler function returns, unless the handler calls an \char`\"{}\+Add\+Ref\char`\"{} function before returning. That would reduce the amount of code required in the common case. However, we chose to require that the client release the message explicitly in all cases, because the consequences of using an invalid reference can be catastrophic and much more difficult to debug than forgetting to release a message (which will generate pool growth warning messages in the log).\hypertarget{c_messaging_c_messagingTroubleshooting}{}\subsection{Troubleshooting}\label{c_messaging_c_messagingTroubleshooting}
If you are running as the super-\/user (root), you can trace messaging traffic using {\bfseries T\+B\+D}. You can also inspect message queues and view lists of outstanding message objects within processes using the Process Inspector tool.

If you are leaking messages by forgetting to release them when you are finished with them, you will see warning messages in the log indicating that your message pool is growing. You should be able to tell by the name of the expanding pool which messaging service it is related to.





Copyright (C) Sierra Wireless Inc. Use of this work is subject to license. \hypertarget{c_mutex}{}\section{Mutex A\+P\+I}\label{c_mutex}
\hyperlink{le__mutex_8h}{A\+P\+I Reference}





The Mutex A\+P\+I provides standard mutex functionality with added diagnostics capabilities. These mutexes can be shared by threads within the same process, but can\textquotesingle{}t be shared by threads in different processes.

\begin{DoxyWarning}{Warning}
Multithreaded programming is an advanced subject with many pitfalls. A general discussion of why and how mutexes are used in multithreaded programming is beyond the scope of this documentation. If you are not familiar with these concepts {\itshape please} seek out training and mentorship before attempting to work on multithreaded production code.
\end{DoxyWarning}
Two kinds of mutex are supported by Legato\+:
\begin{DoxyItemize}
\item {\bfseries Normal} 
\item {\bfseries Traceable} 
\end{DoxyItemize}

Normal mutexes are faster than traceable mutexes and consume less memory, but still offer some diagnosic capabilities. Traceable mutexes generally behave the same as Normal mutexes, but can also log their activities.

In addition, both Normal and Traceable mutexes can be either
\begin{DoxyItemize}
\item {\bfseries Recursive} or
\item {\bfseries Non-\/\+Recursive} 
\end{DoxyItemize}

All mutexes can be locked and unlocked. The same lock, unlock, and delete functions work for all the mutexes, regardless of what type they are. his means that a mutex can be changed from Normal to Traceable (or vice versa) by changing the function you use to create it. This helps to troubleshoot race conditions or deadlocks because it\textquotesingle{}s easy to switch one mutex or a select few mutexes to Traceable without suffering the runtime cost of switching {\itshape all} mutexes to the slower Traceable mutexes.

A recursive mutex can be locked again by the same thread that already has the lock, but a non-\/recursive mutex can only be locked once before being unlocked.

If a thread grabs a non-\/recursive mutex lock and then tries to grab that same lock again, a deadlock occurs. Legato\textquotesingle{}s non-\/recursive mutexes will detect this deadlock, log a fatal error and terminate the process.

If a thread grabs a recursive mutex, and then the same thread grabs the same lock again, the mutex\textquotesingle{}s \char`\"{}lock count\char`\"{} is incremented. When the thread unlocks that mutex, the lock count is decremented. Only when the lock count reaches zero will the mutex actually unlock.

There\textquotesingle{}s a limit to the number of times the same recursive mutex can be locked by the same thread without ever unlocking it, but that limit is so high (at least 2 billion), if that much recursion is going on, there are other, more serious problems with the program.\hypertarget{c_mutex_c_mutex_create}{}\subsection{Creating a Mutex}\label{c_mutex_c_mutex_create}
In Legato, mutexes are dynamically allocated objects. Functions that create them return references to them (of type le\+\_\+mutex\+\_\+\+Ref\+\_\+t).

These are the functions to create mutexes\+:
\begin{DoxyItemize}
\item {\ttfamily \hyperlink{le__mutex_8h_ac7dd2b69f4b905d56df969c9085a570b}{le\+\_\+mutex\+\_\+\+Create\+Recursive()}} -\/ creates a {\bfseries normal}, {\bfseries recursive} mutex.
\item {\ttfamily \hyperlink{le__mutex_8h_a602e2c18e646db7af0d68bb5fb103207}{le\+\_\+mutex\+\_\+\+Create\+Non\+Recursive()}} -\/ creates a {\bfseries normal}, {\bfseries non-\/recursive} mutex.
\item {\ttfamily \hyperlink{le__mutex_8h_a17bcd94aaf2c29e10cd90f949b1e13a7}{le\+\_\+mutex\+\_\+\+Create\+Traceable\+Recursive()}} -\/ creates a {\bfseries traceable}, {\bfseries recursive} mutex.
\item {\ttfamily \hyperlink{le__mutex_8h_abe16eb57e75131afe47d06c0530d5ee9}{le\+\_\+mutex\+\_\+\+Create\+Traceable\+Non\+Recursive()}} -\/ creates a {\bfseries traceable}, {\bfseries non-\/recursive} mutex.
\end{DoxyItemize}

All mutexes have names, required for diagnostic purposes. See \hyperlink{c_mutex_c_mutex_diagnostics}{Diagnostics} below.\hypertarget{c_mutex_c_mutex_locking}{}\subsection{Using a Mutex}\label{c_mutex_c_mutex_locking}
These are the functions to lock and unlock mutexes\+:
\begin{DoxyItemize}
\item {\ttfamily \hyperlink{le__mutex_8h_ad5b7d94710f420cd945229648e7a80e7}{le\+\_\+mutex\+\_\+\+Lock()}} 
\item {\ttfamily \hyperlink{le__mutex_8h_aae68b71222e20c55ff3bf2d7b52e3009}{le\+\_\+mutex\+\_\+\+Unlock()}} 
\item {\ttfamily \hyperlink{le__mutex_8h_a43864999f70f0a825cf8ca87f9a2ee2c}{le\+\_\+mutex\+\_\+\+Try\+Lock()}} 
\end{DoxyItemize}

It doesn\textquotesingle{}t matter what type of mutex you are using, you still use the same functions for locking and unlocking your mutex.\hypertarget{c_mutex_c_mutex_locking_tips}{}\subsubsection{Tip}\label{c_mutex_c_mutex_locking_tips}
A common case is often where a module has a single mutex it uses for some data structure that may get accessed by multiple threads. To make the locking and unlocking of that mutex jump out at readers of the code (and to make coding a little easier too), the following can be created in that module\+:


\begin{DoxyCode}
\textcolor{keyword}{static} \hyperlink{le__mutex_8h_ab2af11e2077e6bed9962eb7dfd54eb03}{le\_mutex\_Ref\_t} MyMutexRef;
\textcolor{keyword}{static} \textcolor{keyword}{inline} Lock(\textcolor{keywordtype}{void}) \{ \hyperlink{le__mutex_8h_ad5b7d94710f420cd945229648e7a80e7}{le\_mutex\_Lock}(MyMutexRef); \}
\textcolor{keyword}{static} \textcolor{keyword}{inline} Unlock(\textcolor{keywordtype}{void}) \{ \hyperlink{le__mutex_8h_aae68b71222e20c55ff3bf2d7b52e3009}{le\_mutex\_Unlock}(MyMutexRef); \}
\end{DoxyCode}


This results in code that looks like this\+:


\begin{DoxyCode}
\textcolor{keyword}{static} \textcolor{keywordtype}{void} SetParam(\textcolor{keywordtype}{int} param)
\{
    Lock();

    MyObjPtr->param = param;

    Unlock();
\}
\end{DoxyCode}


To make this easier, the Mutex A\+P\+I provides the \hyperlink{le__mutex_8h_a7bada3ca8908be93ba5b393e460f6e80}{L\+E\+\_\+\+M\+U\+T\+E\+X\+\_\+\+D\+E\+C\+L\+A\+R\+E\+\_\+\+R\+E\+F()} macro. Using that macro, the three declaration lines


\begin{DoxyCode}
\textcolor{keyword}{static} \hyperlink{le__mutex_8h_ab2af11e2077e6bed9962eb7dfd54eb03}{le\_mutex\_Ref\_t} MyMutexRef;
\textcolor{keyword}{static} \textcolor{keyword}{inline} Lock(\textcolor{keywordtype}{void}) \{ \hyperlink{le__mutex_8h_ad5b7d94710f420cd945229648e7a80e7}{le\_mutex\_Lock}(MyMutexRef); \}
\textcolor{keyword}{static} \textcolor{keyword}{inline} Unlock(\textcolor{keywordtype}{void}) \{ \hyperlink{le__mutex_8h_aae68b71222e20c55ff3bf2d7b52e3009}{le\_mutex\_Unlock}(MyMutexRef); \}
\end{DoxyCode}


can be replaced with one\+:


\begin{DoxyCode}
\hyperlink{le__mutex_8h_a7bada3ca8908be93ba5b393e460f6e80}{LE\_MUTEX\_DECLARE\_REF}(MyMutexRef);
\end{DoxyCode}
\hypertarget{c_mutex_c_mutex_delete}{}\subsection{Deleting a Mutex}\label{c_mutex_c_mutex_delete}
When you are finished with a mutex, you must delete it by calling \hyperlink{le__mutex_8h_a38571fa1d9c15d5f30ea9c480d8810c6}{le\+\_\+mutex\+\_\+\+Delete()}.

There must not be anyone using the mutex when it is deleted (i.\+e., no one can be holding it).\hypertarget{c_mutex_c_mutex_diagnostics}{}\subsection{Diagnostics}\label{c_mutex_c_mutex_diagnostics}
Both Normal and Traceable mutexes have some diagnostics capabilities.

The command-\/line diagnostic tool lsmutex can be used to list the mutexes that currently exist inside a given process. The state of each mutex can be seen, including a list of any threads that might be waiting for that mutex.

The tool threadlook will report if a given thread is currently holding the lock on a mutex or waiting for a mutex along with the mutex name.

If there are Traceable mutexes in a process, it\textquotesingle{}s possible to use the log tool to enable or disable tracing on that mutex. The trace keyword name is the name of the process, the name of the component, and the name of the mutex, separated by slashes (e.\+g., \char`\"{}process/component/mutex\char`\"{}).





Copyright (C) Sierra Wireless Inc. Use of this work is subject to license. \hypertarget{c_path}{}\section{Path A\+P\+I}\label{c_path}
\hyperlink{le__path_8h}{A\+P\+I Reference}





Paths are text strings that contain nodes separated by character separators. Paths are used in many common applications like file system addressing, U\+R\+Ls, etc. so being able to parse them is quite important.

The Path A\+P\+I is intended for general purpose use and supports U\+T\+F-\/8 null-\/terminated strings and multi-\/character separators.\hypertarget{c_path_c_path_dirAndBasename}{}\subsection{Directory and Basename}\label{c_path_c_path_dirAndBasename}
The function {\ttfamily \hyperlink{le__path_8h_a0b0ff4c06db44de9bf4f40b9f5388785}{le\+\_\+path\+\_\+\+Get\+Dir()}} is a convenient way to get the path\textquotesingle{}s directory without having to create an iterator. The directory is the portion of the path up to and including the last separator. \hyperlink{le__path_8h_a0b0ff4c06db44de9bf4f40b9f5388785}{le\+\_\+path\+\_\+\+Get\+Dir()} does not modify the path in anyway (i.\+e., consecutive paths are left as is), except to drop the node after the last separator.

The function \hyperlink{le__path_8h_aa58d208512dd5b9b2dc0ea6d5c963c25}{le\+\_\+path\+\_\+\+Get\+Basename\+Ptr()} is an efficient and convenient function for accessing the last node in the path without having to create an iterator. The returned pointer points to the character following the last separator in the path. Because the basename is actually a portion of the path string, not a copy, any changes to the returned basename will also affect the original path string.\hypertarget{c_path_c_path_threads}{}\subsection{Thread Safety}\label{c_path_c_path_threads}
All the functions in this A\+P\+I are thread safe and reentrant unless of course the path iterators or the buffers passed into the functions are shared between threads. If the path iterators or buffers are shared by multiple threads then some other mechanism must be used to ensure these functions are thread safe.





Copyright (C) Sierra Wireless Inc. Use of this work is subject to license. license. \hypertarget{c_pathIter}{}\section{Path Iterator A\+P\+I}\label{c_pathIter}
\hyperlink{le__path_iter_8h}{A\+P\+I Reference}





Paths are text strings that contain nodes separated by character separators. Paths are used in many common applications like file system addressing, U\+R\+Ls, etc. so being able to parse them is quite important.

The Path Iterator A\+P\+I is intended for general purpose use and supports U\+T\+F-\/8 null-\/terminated strings and multi-\/character separators.

This A\+P\+I can be used to iterate over paths, traversing the path node-\/by-\/node. Or creating and combining paths together while ensuring that the resultant paths are properly normalized. For instance the following path\+:

\begin{DoxyVerb} /a//path/to/a///some/../place
\end{DoxyVerb}


Would be normalized to the path\+:

\begin{DoxyVerb} /a/path/to/a/place
\end{DoxyVerb}
\hypertarget{c_path_iter_c_pathIter_create}{}\subsection{Create a Path Iterator}\label{c_path_iter_c_pathIter_create}
Before iterating over a path, a path object must first be created by calling either {\ttfamily \hyperlink{le__path_iter_8h_a73fac1b657b752b17395c66fb1ae324b}{le\+\_\+path\+Iter\+\_\+\+Create()}}, or {\ttfamily \hyperlink{le__path_iter_8h_a35a38b307f9fdc0de82552e96a5a2d1d}{le\+\_\+path\+Iter\+\_\+\+Create\+For\+Unix()}}. {\ttfamily \hyperlink{le__path_iter_8h_a73fac1b657b752b17395c66fb1ae324b}{le\+\_\+path\+Iter\+\_\+\+Create()}} will allow you to create an iterator for one of many different path styles. While {\ttfamily \hyperlink{le__path_iter_8h_a35a38b307f9fdc0de82552e96a5a2d1d}{le\+\_\+path\+Iter\+\_\+\+Create\+For\+Unix()}} will create an iterator preconfigured for Unix style paths.

All strings to this A\+P\+I must be formatted as U\+T\+F-\/8 null-\/terminated strings.

When the path object is no longer needed, it can be deleted by calling \hyperlink{le__path_iter_8h_a6b57267a2c0db0210aab96c66459f9a1}{le\+\_\+path\+Iter\+\_\+\+Delete()}.\hypertarget{c_path_iter_c_pathIter_iterate}{}\subsection{Iterating a Path}\label{c_path_iter_c_pathIter_iterate}
Once an object is created, the nodes in it can be accessed using {\ttfamily \hyperlink{le__path_iter_8h_ad83a619dcc34ecf03da1859b3da2f57f}{le\+\_\+path\+Iter\+\_\+\+Go\+To\+Next()}}, or {\ttfamily \hyperlink{le__path_iter_8h_a92a740759fe5c3b0a18e39dd8c73466b}{le\+\_\+path\+Iter\+\_\+\+Go\+To\+Prev()}}. To start over at the beginning of the path call {\ttfamily \hyperlink{le__path_iter_8h_af4352480ab3c9ffb09e740f2899d504e}{le\+\_\+path\+Iter\+\_\+\+Go\+To\+Start()}}. To position the iterator at the end of the path, use {\ttfamily \hyperlink{le__path_iter_8h_ab1c0b90132171b3f3cf5cfb614329b13}{le\+\_\+path\+Iter\+\_\+\+Go\+To\+End()}}. On creation, the default position of the iterator is at the end of the path.

Code sample, iterate over an entire path\+:


\begin{DoxyCode}
\textcolor{comment}{// Create an iterator object, and move it to the front of the path.}
\hyperlink{le__path_iter_8h_a0facb15e56e7ef896384eca415a7147a}{le\_pathIter\_Ref\_t} iteratorRef = \hyperlink{le__path_iter_8h_a35a38b307f9fdc0de82552e96a5a2d1d}{le\_pathIter\_CreateForUnix}(
      myPathPtr);

\textcolor{keywordflow}{if} (\hyperlink{le__path_iter_8h_ab4ceddae696158d04fdbc1802614c5d6}{le\_pathIter\_IsEmpty}(iteratorRef))
\{
    \textcolor{keywordflow}{return};
\}

\hyperlink{le__path_iter_8h_af4352480ab3c9ffb09e740f2899d504e}{le\_pathIter\_GoToStart}(iteratorRef);

\textcolor{comment}{// Now go through all of the path nodes and print out each one.}
\textcolor{keywordflow}{do}
\{
    \textcolor{keywordtype}{char} buffer[BUFFER\_SIZE] = \{ 0 \};

    \textcolor{keywordflow}{if} (\hyperlink{le__path_iter_8h_ab00916d853b3a869748b0195cc2a8f11}{le\_pathIter\_GetCurrentNode}(iteratorRef, buffer, BUFFER\_SIZE) == 
      \hyperlink{le__basics_8h_a1cca095ed6ebab24b57a636382a6c86ca5066a4bcec691c6b67843b8f79656422}{LE\_OK})
    \{
        printf(\textcolor{stringliteral}{"%s\(\backslash\)n"}, buffer);
    \}
\}
\textcolor{keywordflow}{while} (\hyperlink{le__path_iter_8h_ad83a619dcc34ecf03da1859b3da2f57f}{le\_pathIter\_GoToNext}(iteratorRef) != \hyperlink{le__basics_8h_a1cca095ed6ebab24b57a636382a6c86ca77a7505b0443df2fa1bab375c7267637}{LE\_NOT\_FOUND});

\textcolor{comment}{// All done with the iterator, so free it now.}
\hyperlink{le__path_iter_8h_a6b57267a2c0db0210aab96c66459f9a1}{le\_pathIter\_Delete}(iteratorRef);
\end{DoxyCode}


\begin{DoxyNote}{Note}
{\ttfamily le\+\_\+path\+Iter\+\_\+\+Get\+Next\+Node()} and {\ttfamily le\+\_\+path\+Iter\+\_\+\+Get\+Previous\+Node()} treat consecutive separators as a single separator.
\end{DoxyNote}
\hypertarget{c_path_iter_c_pathIter_absoluteRelative}{}\subsection{Absolute versus Relative Paths}\label{c_path_iter_c_pathIter_absoluteRelative}
Absolute paths begin with one or more separators. Relative paths do not begin with a separator. {\ttfamily \hyperlink{le__path_iter_8h_a657f779873a2220f463f705298c1399f}{le\+\_\+path\+Iter\+\_\+\+Is\+Absolute()}} can be used to determine if the path is absolute or relative.\hypertarget{c_path_iter_c_pathIter_modifyPath}{}\subsection{Modifying Paths}\label{c_path_iter_c_pathIter_modifyPath}
In addition to pure iteration, the path iterator can allow you to modify a path. For instance, you can iterate to a node in the path and use {\ttfamily \hyperlink{le__path_iter_8h_a04be1341536a3e330a815171e7cdbf7a}{le\+\_\+path\+Iter\+\_\+\+Truncate()}} to truncate everything at and after that point. While you can use {\ttfamily \hyperlink{le__path_iter_8h_ae6aa59696c54d2523009037cc78f9725}{le\+\_\+path\+Iter\+\_\+\+Append()}} to add new path nodes at the end of the path.

Take the following code\+:


\begin{DoxyCode}
\hyperlink{le__path_iter_8h_a0facb15e56e7ef896384eca415a7147a}{le\_pathIter\_Ref\_t} iteratorRef = \hyperlink{le__path_iter_8h_a35a38b307f9fdc0de82552e96a5a2d1d}{le\_pathIter\_CreateForUnix}(\textcolor{stringliteral}{"
      /a/path/to/a/place"});
\textcolor{keywordtype}{char} fullPath[PATH\_SIZE] = \{ 0 \};

\hyperlink{le__path_iter_8h_af4352480ab3c9ffb09e740f2899d504e}{le\_pathIter\_GoToStart}(iteratorRef);

\hyperlink{le__path_iter_8h_ad83a619dcc34ecf03da1859b3da2f57f}{le\_pathIter\_GoToNext}(iteratorRef);
\hyperlink{le__path_iter_8h_ad83a619dcc34ecf03da1859b3da2f57f}{le\_pathIter\_GoToNext}(iteratorRef);
\hyperlink{le__path_iter_8h_ad83a619dcc34ecf03da1859b3da2f57f}{le\_pathIter\_GoToNext}(iteratorRef);

\hyperlink{le__path_iter_8h_a04be1341536a3e330a815171e7cdbf7a}{le\_pathIter\_Truncate}(iteratorRef);

\hyperlink{le__path_iter_8h_ae6aa59696c54d2523009037cc78f9725}{le\_pathIter\_Append}(iteratorRef, \textcolor{stringliteral}{"nowhere"});

\hyperlink{le__path_iter_8h_a4a1c39584a779518395b41f957765283}{le\_pathIter\_GetPath}(iteratorRef, fullPath, PATH\_SIZE);

\hyperlink{le__log_8h_ac0dbbef91dc0fed449d0092ff0557b39}{LE\_ASSERT}(strcmp(fullPath, \textcolor{stringliteral}{"/a/path/to/nowhere"}) == 0);
\end{DoxyCode}


Note that {\ttfamily \hyperlink{le__path_iter_8h_ae6aa59696c54d2523009037cc78f9725}{le\+\_\+path\+Iter\+\_\+\+Append()}} will also normalize paths as it appends. So, the following example has the same effect as the previous one.


\begin{DoxyCode}
\hyperlink{le__path_iter_8h_a0facb15e56e7ef896384eca415a7147a}{le\_pathIter\_Ref\_t} iteratorRef = \hyperlink{le__path_iter_8h_a35a38b307f9fdc0de82552e96a5a2d1d}{le\_pathIter\_CreateForUnix}(\textcolor{stringliteral}{"
      /a/path/to/a/place"});
\textcolor{keywordtype}{char} fullPath[PATH\_SIZE] = \{ 0 \};

\hyperlink{le__path_iter_8h_ae6aa59696c54d2523009037cc78f9725}{le\_pathIter\_Append}(iteratorRef, \textcolor{stringliteral}{"../../nowhere"});
\hyperlink{le__path_iter_8h_a4a1c39584a779518395b41f957765283}{le\_pathIter\_GetPath}(iteratorRef, fullPath, PATH\_SIZE);

\hyperlink{le__path_iter_8h_a4a1c39584a779518395b41f957765283}{le\_pathIter\_GetPath}(iteratorRef, fullPath, PATH\_SIZE);

\hyperlink{le__log_8h_ac0dbbef91dc0fed449d0092ff0557b39}{LE\_ASSERT}(strcmp(fullPath, \textcolor{stringliteral}{"/a/path/to/nowhere"}) == 0);
\end{DoxyCode}






Copyright (C) Sierra Wireless Inc. Use of this work is subject to license. license. \hypertarget{c_print}{}\section{Print A\+P\+Is}\label{c_print}
\hyperlink{le__print_8h}{A\+P\+I Reference}





Copyright (C) Sierra Wireless Inc. Use of this work is subject to license. \hypertarget{c_clock}{}\section{System Clock A\+P\+I}\label{c_clock}
\hyperlink{le__clock_8h}{A\+P\+I Reference}





This module provides an A\+P\+I for getting/setting date and/or time values, and performing conversions between these values.\hypertarget{c_clock_clk_time}{}\subsection{Getting/\+Setting Time}\label{c_clock_clk_time}
Time values can either be absolute or relative. Time is expressed in seconds plus microseconds, and does not stop when the system is suspended (i.\+e., the clock continues to run even when the system is suspended).

Absolute time is given as time since the Epoch, 1970-\/01-\/01 00\+:00\+:00 +0000 (U\+T\+C) and is provided by \hyperlink{le__clock_8h_a33197dbd676a37b8c4d5de8f93edc1ee}{le\+\_\+clk\+\_\+\+Get\+Absolute\+Time()}. By definition, it is U\+T\+C time. The absolute time may jump forward or backward if a new value is set for the absolute time.

Relative time is a monotonic time from a fixed but unspecified starting point and is provided by \hyperlink{le__clock_8h_a298619d8c1d8f107cb03b8600f09a42b}{le\+\_\+clk\+\_\+\+Get\+Relative\+Time()}. The relative time is independent of the absolute time. The starting point is fixed during system boot, and cannot be changed, but is reset on each system boot. Since the relative time is monotonic, it is guaranteed to never go backwards. With these characteristics, the relative time is useful for measuring the time between two or more events. For example, at event 1, relative time A is stored, and at some later event 2, relative time B is stored. The relative time between these two events can always be calculated as B-\/\+A, and will always be an accurate measure of the relative time between these two events.\hypertarget{c_clock_clk_values}{}\subsection{Operations on Time Values}\label{c_clock_clk_values}
These operations can be performed on time values\+:
\begin{DoxyItemize}
\item \hyperlink{le__clock_8h_a6f4bda0268cd9349b6eb81ae350575fc}{le\+\_\+clk\+\_\+\+Add}
\item \hyperlink{le__clock_8h_ac4a550ee8aa5fad9c81a33024946660a}{le\+\_\+clk\+\_\+\+Greater\+Than}
\item \hyperlink{le__clock_8h_ac5b5ec6f462d598f4e5aa081224725ac}{le\+\_\+clk\+\_\+\+Sub}
\item \hyperlink{le__clock_8h_a77961175ee422b1418a18eece5192c9c}{le\+\_\+clk\+\_\+\+Multiply}
\end{DoxyItemize}

The functions use these assumptions\+:
\begin{DoxyItemize}
\item All input time values are normalized (i.\+e., the usec value is less than 1 sec). All time values returned are normalized.
\item All input time values or scale factors are positive; a negative time value will not be returned.
\item All input time values or scale factors are expected to have reasonable values (i.\+e., they will not be so large as to cause an overflow of the time value structure).
\end{DoxyItemize}\hypertarget{c_clock_clk_convert}{}\subsection{Converting Time to Other Formats}\label{c_clock_clk_convert}
The current absolute time can be converted to a formatted string in either U\+T\+C time or local time, using \hyperlink{le__clock_8h_a5392b8eb7d45ce86c0842a0a69975059}{le\+\_\+clk\+\_\+\+Get\+U\+T\+C\+Date\+Time\+String()} or \hyperlink{le__clock_8h_a6f37c2a2171eac23ddc306de1fd55f5c}{le\+\_\+clk\+\_\+\+Get\+Local\+Date\+Time\+String()} respectively. These functions use the format specification defined for strftime(), with the following additional conversion specifications\+:
\begin{DoxyItemize}
\item \%J \+: milliseconds, as a 3 digit zero-\/padded string, e.\+g. \char`\"{}015\char`\"{}
\item \%K \+: microseconds, as a 6 digit zero-\/padded string, e.\+g. \char`\"{}001015\char`\"{}
\end{DoxyItemize}





Copyright (C) Sierra Wireless Inc. Use of this work is subject to license. \hypertarget{c_safeRef}{}\section{Safe References A\+P\+I}\label{c_safeRef}
\hyperlink{le__safe_ref_8h}{A\+P\+I Reference}





The term \char`\"{}reference\char`\"{} is used to mean \char`\"{}opaque data that refers to some conceptual object\char`\"{}. It is intentionally vague to support \char`\"{}information hiding\char`\"{}. Behind the scenes, different implementations can use almost anything that fits into a pointer as a \char`\"{}reference\char`\"{}. Often, they are indexes into arrays or actual pointers to memory objects. When passing those references through an A\+P\+I to outside clients, the implementation becomes exposed to crash bugs when clients pass those references back into the A\+P\+I damaged or stale (\char`\"{}stale\char`\"{} meaning something that has been deleted).

{\bfseries  Safe References } are designed to help protect against damaged or stale references being used by clients.\hypertarget{c_safe_ref_c_safeRef_create}{}\subsection{Create Safe Reference}\label{c_safe_ref_c_safeRef_create}
Client calls an A\+P\+I\textquotesingle{}s \char`\"{}\+Create\char`\"{} function\+:
\begin{DoxyItemize}
\item \char`\"{}\+Create\char`\"{} function creates an object.
\item \char`\"{}\+Create\char`\"{} function creates a \char`\"{}\+Safe Reference\char`\"{} for the new object {\ttfamily \hyperlink{le__safe_ref_8h_a458597757cbce48e03413b49f52ec240}{le\+\_\+ref\+\_\+\+Create\+Ref()}} 
\item \char`\"{}\+Create\char`\"{} function returns the Safe Reference.
\end{DoxyItemize}\hypertarget{c_safe_ref_c_safeRef_lookup}{}\subsection{Lookup Pointer}\label{c_safe_ref_c_safeRef_lookup}
Followed by\+:

Client calls another A\+P\+I function, passing in the Safe Reference\+:
\begin{DoxyItemize}
\item A\+P\+I function translates the Safe Reference back into an object pointer {\ttfamily \hyperlink{le__safe_ref_8h_a488dddfd579f4a20f39be392c4d7d2e0}{le\+\_\+ref\+\_\+\+Lookup()}} 
\item A\+P\+I function acts on the object.
\end{DoxyItemize}\hypertarget{c_safe_ref_c_safeRef_delete}{}\subsection{Delete Safe Reference}\label{c_safe_ref_c_safeRef_delete}
Finishing with\+:

Client calls A\+P\+I\textquotesingle{}s \char`\"{}\+Delete\char`\"{} function, passing in the Safe Reference\+:
\begin{DoxyItemize}
\item \char`\"{}\+Delete\char`\"{} function translates the Safe Reference back into a pointer to its object.
\item \char`\"{}\+Delete\char`\"{} function invalidates the Safe Reference {\ttfamily \hyperlink{le__safe_ref_8h_a438e18b8ace1d4dda3ca5144a27bd424}{le\+\_\+ref\+\_\+\+Delete\+Ref()}} 
\item \char`\"{}\+Delete\char`\"{} function deletes the object.
\end{DoxyItemize}

At this point, if the Client calls an A\+P\+I function and passes that same (now invalid) Safe Reference (or if the client accidentally passes in some garbage value, like a pointer or zero), the A\+P\+I function will try to translate that into an object pointer. But it\textquotesingle{}ll be told that it\textquotesingle{}s an invalid Safe Reference. The A\+P\+I function can then handle it gracefully, rather than just acting as if it were a valid reference and clobbering the object\textquotesingle{}s deallocated memory or some other object that\textquotesingle{}s reusing the old object\textquotesingle{}s memory.\hypertarget{c_safe_ref_c_safeRef_map}{}\subsection{Create Referece Map}\label{c_safe_ref_c_safeRef_map}
A {\bfseries  Reference Map } object can be used to create Safe References and keep track of the mappings from Safe References to pointers. At start-\/up, a Reference Map is created by calling {\ttfamily \hyperlink{le__safe_ref_8h_a85faf3c75723a1af0e1adf720d9c9dca}{le\+\_\+ref\+\_\+\+Create\+Map()}}. It takes a single argument, the maximum number of mappings expected to track of at any time.\hypertarget{c_safe_ref_c_safeRef_multithreading}{}\subsection{Multithreading}\label{c_safe_ref_c_safeRef_multithreading}
This A\+P\+I\textquotesingle{}s functions are reentrant, but not thread safe. If there\textquotesingle{}s the slightest possibility the same Reference Map will be accessed by two threads at the same time, use a mutex or some other thread synchronization mechanism to protect the Reference Map from concurrent access.\hypertarget{c_safe_ref_c_safeRef_example}{}\subsection{Sample Code}\label{c_safe_ref_c_safeRef_example}
Here\textquotesingle{}s an A\+P\+I Definition sample\+:


\begin{DoxyCode}
\textcolor{comment}{// Opaque reference to Foo objects.}
\textcolor{keyword}{typedef} \textcolor{keyword}{struct }xyz\_foo\_Obj* xyz\_foo\_ObjRef\_t;

xyz\_foo\_Ref\_t xyz\_foo\_CreateObject
(
    \textcolor{keywordtype}{void}
);

\textcolor{keywordtype}{void} xyz\_foo\_DoSomething
(
    xyz\_foo\_Ref\_t objRef
);

\textcolor{keywordtype}{void} xyz\_foo\_DeleteObject
(
    xyz\_foo\_Ref\_t objRef
);
\end{DoxyCode}


Here\textquotesingle{}s an A\+P\+I Implementation sample\+:


\begin{DoxyCode}
\textcolor{comment}{// Maximum number of Foo objects we expect to have at one time.}
\textcolor{preprocessor}{#define MAX\_FOO\_OBJECTS  27}

\textcolor{comment}{// Actual Foo objects.}
\textcolor{keyword}{typedef} \textcolor{keyword}{struct}
\{
    ...
\}
Foo\_t;

\textcolor{comment}{// Pool from which Foo objects are allocated.}
le\_mem\_PoolRef\_t FooPool;

\textcolor{comment}{// Safe Reference Map for Foo objects.}
\hyperlink{le__safe_ref_8h_aaf8fb3412fb840cb50366856affaf69b}{le\_ref\_MapRef\_t} FooRefMap;

\hyperlink{le__event_loop_8h_abdb9187a56836a93d19cc793cbd4b7ec}{COMPONENT\_INIT}
\{
    \textcolor{comment}{// Create the Foo object pool.}
    FooPool = \hyperlink{le__mem_8h_ab91efaa2978c9c1c7b2427d25b33241c}{le\_mem\_CreatePool}(\textcolor{stringliteral}{"FooPool"}, \textcolor{keyword}{sizeof}(Foo\_t));
    \hyperlink{le__mem_8h_a79a4321ffa0345f267eaf3b7d3d3528a}{le\_mem\_ExpandPool}(FooPool, MAX\_FOO\_OBJECTS);

    \textcolor{comment}{// Create the Safe Reference Map to use for Foo object Safe References.}
    FooRefMap = \hyperlink{le__safe_ref_8h_a85faf3c75723a1af0e1adf720d9c9dca}{le\_ref\_CreateMap}(\textcolor{stringliteral}{"FooMap"}, MAX\_FOO\_OBJECTS);
\};

xyz\_foo\_Ref\_t xyz\_foo\_CreateObject
(
    \textcolor{keywordtype}{void}
)
\{
    Foo\_t* fooPtr = \hyperlink{le__mem_8h_af7c289c73d4182835a26a9099f3db359}{le\_mem\_ForceAlloc}(FooPool);

    \textcolor{comment}{// Initialize the new Foo object.}
    ...

    \textcolor{comment}{// Create and return a Safe Reference for this Foo object.}
    \textcolor{keywordflow}{return} \hyperlink{le__safe_ref_8h_a458597757cbce48e03413b49f52ec240}{le\_ref\_CreateRef}(FooRefMap, fooPtr);
\}

\textcolor{keywordtype}{void} xyz\_foo\_DoSomething
(
    xyz\_foo\_Ref\_t objRef
)
\{
    Foo\_t* fooPtr = \hyperlink{le__safe_ref_8h_a488dddfd579f4a20f39be392c4d7d2e0}{le\_ref\_Lookup}(FooRefMap, objRef);

    \textcolor{keywordflow}{if} (fooPtr == NULL)
    \{
        \hyperlink{le__log_8h_a5efa1e4b6292c820c8555b4066a5c10d}{LE\_CRIT}(\textcolor{stringliteral}{"Invalid reference (%p) provided!"}, objRef);
        \textcolor{keywordflow}{return};
    \}

    \textcolor{comment}{// Do something to the object.}
    ...
\}

\textcolor{keywordtype}{void} xyz\_foo\_DeleteObject
(
    xyz\_foo\_Ref\_t objRef
)
\{
    Foo\_t* fooPtr = \hyperlink{le__safe_ref_8h_a488dddfd579f4a20f39be392c4d7d2e0}{le\_ref\_Lookup}(FooRefMap, objRef);

    \textcolor{keywordflow}{if} (fooPtr == NULL)
    \{
        \hyperlink{le__log_8h_a5efa1e4b6292c820c8555b4066a5c10d}{LE\_CRIT}(\textcolor{stringliteral}{"Invalid reference (%p) provided!"}, objRef);
        \textcolor{keywordflow}{return};
    \}

    \textcolor{comment}{// Invalidate the Safe Reference.}
    \hyperlink{le__safe_ref_8h_a438e18b8ace1d4dda3ca5144a27bd424}{le\_ref\_DeleteRef}(FooRefMap, objRef);

    \textcolor{comment}{// Release the Foo object.}
    \hyperlink{le__mem_8h_a6d8e3fe430bcb81efe97b57ce30ef2de}{le\_mem\_Release}(fooPtr);
\}
\end{DoxyCode}






Copyright (C) Sierra Wireless Inc. Use of this work is subject to license. \hypertarget{c_semaphore}{}\section{Semaphore A\+P\+I}\label{c_semaphore}
\hyperlink{le__semaphore_8h}{A\+P\+I Reference}



 This A\+P\+I provides standard semaphore functionality, but with added diagnostic capabilities. These semaphores can only be shared by threads within the same process.

Two kinds of semaphore are supported by Legato\+:
\begin{DoxyItemize}
\item {\bfseries Normal} 
\item {\bfseries Traceable} 
\end{DoxyItemize}

Normal semaphores are faster than traceable semaphores and consume less memory, but still offer some diagnosic capabilities. Traceable semaphores generally behave the same as Normal sempahores, but can also log their activities.

All semaphores can wait (decrease value by one) and post (increase value by one). The same wait, post, and delete functions work for all the semaphores, regardless of what type they are. This means that a semaphore can be changed from Normal to Traceable or vice versa just by changing the function you use to create it. This helps when troubleshooting race conditions or deadlocks because it\textquotesingle{}s easy to switch one semaphore or a select few semaphores to Traceable, without suffering the runtime cost of switching {\itshape all} semaphores to the slower Traceable semaphores.\hypertarget{c_semaphore_create_semaphore}{}\subsection{Creating a Semaphore}\label{c_semaphore_create_semaphore}
In Legato, semaphores are dynamically allocated objects. Functions that create them return references to them (of type le\+\_\+sem\+\_\+\+Ref\+\_\+t). The functions for creating semaphores are\+:
\begin{DoxyItemize}
\item \hyperlink{le__semaphore_8h_add9fab5440abcff5a8bc3b8bd1126d99}{le\+\_\+sem\+\_\+\+Create()} -\/ creates a {\bfseries normal}, {\bfseries semaphore}.
\item \hyperlink{le__semaphore_8h_a27f805b9351bb5bb4a5fd6db6cbff982}{le\+\_\+sem\+\_\+\+Create\+Traceable()} -\/ creates a {\bfseries traceable}, {\bfseries semaphore}.
\end{DoxyItemize}

Note that all semaphores have names. This is required for diagnostic purposes. See \hyperlink{c_semaphore_diagnostics_semaphore}{Diagnostics} below.\hypertarget{c_semaphore_use_semaphore}{}\subsection{Using a Semaphore}\label{c_semaphore_use_semaphore}
Functions to increase and decrease semaphores are\+:
\begin{DoxyItemize}
\item {\ttfamily \hyperlink{le__semaphore_8h_abb859411cc58fbcc576c986ef52083b2}{le\+\_\+sem\+\_\+\+Post()}} 
\item {\ttfamily \hyperlink{le__semaphore_8h_aecdf87fe330dd008771b7530edbb1f2b}{le\+\_\+sem\+\_\+\+Wait()}} 
\item {\ttfamily \hyperlink{le__semaphore_8h_a6a6c435042dd37a3c78ebbab6ec72689}{le\+\_\+sem\+\_\+\+Try\+Wait()}} 
\item {\ttfamily \hyperlink{le__semaphore_8h_a14475f0c2f5483427279d39220f55eaa}{le\+\_\+sem\+\_\+\+Wait\+With\+Time\+Out()}} 
\end{DoxyItemize}

Function to get semaphores values is\+:
\begin{DoxyItemize}
\item {\ttfamily \hyperlink{le__semaphore_8h_ac4858ccb0ba748ca463bb29807b75c05}{le\+\_\+sem\+\_\+\+Get\+Value()}} 
\end{DoxyItemize}

It doesn\textquotesingle{}t matter what type of semaphore you are using, you still use the same functions for increasing, decreasing, getting value your semaphore.\hypertarget{c_semaphore_delete_semaphore}{}\subsection{Deleting a Semaphore}\label{c_semaphore_delete_semaphore}
When you are finished with a semaphore, you must delete it by calling \hyperlink{le__semaphore_8h_a96361b126f59934354ca17bf8b74b8f6}{le\+\_\+sem\+\_\+\+Delete()}.

There must not be anything using the semaphore when it is deleted (i.\+e., no one can be waiting on it).\hypertarget{c_semaphore_diagnostics_semaphore}{}\subsection{Diagnostics}\label{c_semaphore_diagnostics_semaphore}
Both Normal and Traceable semaphores have some diagnostics capabilities.

The command-\/line diagnostic tool lssemaphore can be used to list the semaphores that currently exist inside a given process. The state of each semaphore can be seen, including a list of any threads that might be waiting for that semaphore.

The tool threadlook will show if a given thread is currently waiting for a semaphore, and will name that semaphore.

If there are Traceable semaphores in a process, then it will be possible to use the log tool to enable or disable tracing on that semaphore. The trace keyword name is the name of the process, the name of the component, and the name of the semaphore, separated by slashes (e.\+g., \char`\"{}process/component/semaphore\char`\"{}).





Copyright (C) Sierra Wireless Inc. Use of this work is subject to license. \hypertarget{c_signals}{}\section{Signals A\+P\+I}\label{c_signals}
\hyperlink{le__signals_8h}{A\+P\+I Reference}





Signals are software interrupts that can be sent to a running process or thread to indicate exceptional situations. The action taken when an event is received depends on the current settings for the signal and may be set to either\+:


\begin{DoxyItemize}
\item Operating systems default action.
\item Ignore the signal.
\item A custom handler.
\end{DoxyItemize}

When a signal is received, unless it is ignored or blocked the action for the signal will preempt any code that is currently executing. Also, signals are asynchronous and may arrive at any time. See \href{http://man7.org/linux/man-pages/man7/signal.7.html}{\tt http\+://man7.\+org/linux/man-\/pages/man7/signal.\+7.\+html} for more details.

The asynchronous and preemptive nature of signals can be difficult to deal with and is often a source of race conditions. Moreover asynchronous and preemptive signal handling is often unnecessary so code often looks something like this\+:


\begin{DoxyCode}
\textcolor{comment}{// A global volatile atomic flag.}
\textcolor{keyword}{static} \textcolor{keyword}{volatile} \textcolor{keywordtype}{int} GotSignal;

\textcolor{keywordtype}{void} sigHandler(\textcolor{keywordtype}{int} sigNum)
\{
     \textcolor{comment}{// Must only use asynch-signal-safe functions in this handler, see signals(7) man page for}
     \textcolor{comment}{// more details.}

     \textcolor{comment}{// Set the flag.}
     GotSignal = 1;
\}

\textcolor{keywordtype}{int} main (\textcolor{keywordtype}{void})
\{
     \textcolor{keywordflow}{while}(1)
     \{
         \textcolor{comment}{// Do something.}
         ...

         \textcolor{keywordflow}{if} (GotSignal = 1)
         \{
             \textcolor{comment}{// Clear the flag.}
             GotSignal = 0;

             \textcolor{comment}{// Process the signal.}
             ...
         \}
     \}
\}
\end{DoxyCode}


In this code sample, the signal handler is only used to set a flag, while the main loop handles the actual signal processing. But handling signals this way requires the main loop to run continuously. This code is also prone to errors. For example, if the clearing of the flags was done after processing of the signal, any signals received during processing of the signals will be lost.\hypertarget{c_signals_c_signals_eventHandlers}{}\subsection{Signal Event Handlers}\label{c_signals_c_signals_eventHandlers}
The Legato signals A\+P\+I provides a simpler alternative, called signal events. Signal events can be used to receive and handle signals synchronously without the need for a sit-\/and-\/wait loop or even a block-\/and-\/wait call.

To use signal events, the desired signals must first be blocked using \hyperlink{le__signals_8h_a095ec12deab6b6ed0475583586a6c4d7}{le\+\_\+sig\+\_\+\+Block()} (see \hyperlink{c_signals_c_signals_blocking}{Blocking signals}). Then set a signal event handler for the desired signal using {\ttfamily \hyperlink{le__signals_8h_a421910132f193dae70e8309dc86a86c4}{le\+\_\+sig\+\_\+\+Set\+Event\+Handler()}}. Once a signal to the thread is received, the signal event handler is called by the thread\textquotesingle{}s Legato event loop (see \hyperlink{c_eventLoop}{Event Loop A\+P\+I} for more details). The handler is called synchronously in the context of the thread that set the handler. Be aware that if the thread\textquotesingle{}s event loop is not called or is blocked by some other code, the signal event handler will also be blocked.

Here is an example using signal events to handle the S\+I\+G\+C\+H\+L\+D signals\+:


\begin{DoxyCode}
\textcolor{comment}{// SIGCHILD event handler that will be called as a synchronous event.}
\textcolor{keyword}{static} \textcolor{keywordtype}{void} SigChildEventHandler(\textcolor{keywordtype}{int} sigNum)
\{
     \textcolor{comment}{// Handle SIGCHLD event.}
     ...

     \textcolor{comment}{// There is no need to limit ourselves to async-signal-safe functions because we are now in}
     \textcolor{comment}{// a synchronous event handler.}
\}


\hyperlink{le__event_loop_8h_abdb9187a56836a93d19cc793cbd4b7ec}{COMPONENT\_INIT}
\{
     \textcolor{comment}{// Block Signals that we are going to set event handlers for.}
     \hyperlink{le__signals_8h_a095ec12deab6b6ed0475583586a6c4d7}{le\_sig\_Block}(SIGCHLD);

     \textcolor{comment}{// Setup the signal event handler.}
     \hyperlink{le__signals_8h_a421910132f193dae70e8309dc86a86c4}{le\_sig\_SetEventHandler}(SIGCHILD, SigChildEventHandler);
\}
\end{DoxyCode}
\hypertarget{c_signals_c_signals_mixedHandlers}{}\subsection{Mixing Asynchronous Signal Handlers with Synchronous Signal Event Handlers}\label{c_signals_c_signals_mixedHandlers}
Signal events work well when dealing with signals synchronously, but when signals must be dealt with in an asynchronously, traditional signal handlers are still preferred. In fact, signal event handlers are not allowed for certain signals like program error signals (ie. S\+I\+G\+F\+P\+E, etc.) because they indicate a serious error in the program and all code outside of signal handlers are considered unreliable. This means that asynchronous signal handlers are the only option when dealing with program error signals.

Signal event handlers can be used in conjunction with asynchronous signal handlers but only if they do not deal with the same signals. In fact all signals that use signal events must be blocked for every thread in the process. The Legato framework takes care of this for you when you set the signals you want to use in the Legato build system.

If your code explicitly unblocks a signal where you currently have signal event handlers, the signal event handlers will no longer be called until the signal is blocked again.\hypertarget{c_signals_c_signals_multiThread}{}\subsection{Multi-\/\+Threading Support}\label{c_signals_c_signals_multiThread}
In a multi-\/threaded system, signals can be sent to either the process or a specific thread. Signals directed at a specific thread will be received by that thread; signals directed at the process are received by one of the threads in the process that has a handler for the signal.

It is unspecified which thread will actually receive the signal so it\textquotesingle{}s recommended to only have one signal event handler per signal.\hypertarget{c_signals_c_signals_limitations}{}\subsection{Limitations and Warnings}\label{c_signals_c_signals_limitations}
A limitation of signals in general (not just with signal events) is called signal merging. Signals that are received but not yet handled are said to be pending. If another signal of the same type is received while the first signal is pending, then the two signals will merge into a single signal and there will be only one handler function call. Consequently, it is not possible to reliably know how many signals arrived.

\begin{DoxyWarning}{Warning}
Signals are difficult to deal with in general because of their asynchronous nature and although, Legato has simplified the situation with signal events certain limitations still exist. If possible, avoid using them.
\end{DoxyWarning}
\hypertarget{c_signals_c_signals_blocking}{}\subsection{Blocking signals}\label{c_signals_c_signals_blocking}
Signals that are to be used with a signal event handlers must be blocked for the entire process. To ensure this use \hyperlink{le__signals_8h_a095ec12deab6b6ed0475583586a6c4d7}{le\+\_\+sig\+\_\+\+Block()} to block signals in the process\textquotesingle{} first thread. All other threads will inherit the signal mask from the first thread.

The example below shows how to use a signal event in a separate thread.


\begin{DoxyCode}
\textcolor{comment}{// SIGCHILD event handler that will be called as a synchronous event in the context of the}
\textcolor{comment}{// workThread.}
\textcolor{keyword}{static} \textcolor{keywordtype}{void} SigChildEventHandler(\textcolor{keywordtype}{int} sigNum)
\{
     \textcolor{comment}{// Handle SIGCHILD event.}
     ...

     \textcolor{comment}{// There is no need to limit ourselves to async-signal-safe functions because we are now in}
     \textcolor{comment}{// a synchronous event handler.}
\}


\textcolor{comment}{// Work thread's main function.}
\textcolor{keyword}{static} \textcolor{keywordtype}{void}* WorkThreadMain(\textcolor{keywordtype}{void}* context)
\{
     \textcolor{comment}{// Setup the signal event handler.}
     \hyperlink{le__signals_8h_a421910132f193dae70e8309dc86a86c4}{le\_sig\_SetEventHandler}(SIGCHILD, SigChildEventHandler);

     \textcolor{comment}{// Start this thread's event loop.}
     \hyperlink{le__event_loop_8h_ae313b457994371c658be9fe0494a01ff}{le\_event\_RunLoop}();

     \textcolor{keywordflow}{return} NULL;
\}


\textcolor{comment}{// Main thread code.}
\hyperlink{le__event_loop_8h_abdb9187a56836a93d19cc793cbd4b7ec}{COMPONENT\_INIT}
\{
     \textcolor{comment}{// Block Signals that we are going to set event handlers for in the main thread so that all}
     \textcolor{comment}{// subsequent threads will inherit the same signal mask.}
     \hyperlink{le__signals_8h_a095ec12deab6b6ed0475583586a6c4d7}{le\_sig\_Block}(SIGCHLD);

     \textcolor{comment}{// Create and start a work thread that will actually handle the signal.}
     \hyperlink{le__thread_8h_a32121104c6b4ca39008eb79a4d6862f2}{le\_thread\_Ref\_t} workThread = \hyperlink{le__thread_8h_a87e02a46f92e9e3e11ed28a2b265872f}{le\_thread\_Create}(\textcolor{stringliteral}{"workThread"}, WorkThread,
       NULL);

     \hyperlink{le__thread_8h_a38df3877ee5ab9fac17b2fc0be46c27e}{le\_thread\_Start}(workThread);
\}
\end{DoxyCode}






Copyright (C) Sierra Wireless Inc. Use of this work is subject to license. \hypertarget{c_singlyLinkedList}{}\section{Singly Linked List A\+P\+I}\label{c_singlyLinkedList}
\hyperlink{le__singly_linked_list_8h}{A\+P\+I Reference}





A singly linked list is a data structure consisting of a group of nodes linked together linearly. Each node consists of data elements and a link to the next node. The main advantage of linked lists over simple arrays is that the nodes can be inserted anywhere in the list without reallocating the entire array because the nodes in a linked list do not need to be stored contiguously in memory. However, nodes in the list cannot be accessed by index but must be accessed by traversing the list.\hypertarget{c_singly_linked_list_sls_createList}{}\subsection{Creating and Initializing Lists}\label{c_singly_linked_list_sls_createList}
To create and initialize a linked list, {\ttfamily create} a \hyperlink{structle__sls___list__t}{le\+\_\+sls\+\_\+\+List\+\_\+t} typed list and assign L\+E\+\_\+\+S\+L\+S\+\_\+\+L\+I\+S\+T\+\_\+\+I\+N\+I\+T to it. The assignment of L\+E\+\_\+\+S\+L\+S\+\_\+\+L\+I\+S\+T\+\_\+\+I\+N\+I\+T can be done either when the list is declared or after it\textquotesingle{}s declared. The list {\bfseries must} be initialized before it can be used.


\begin{DoxyCode}
\textcolor{comment}{// Create and initialized the list in the declaration.}
\hyperlink{structle__sls___list__t}{le\_sls\_List\_t} MyList = \hyperlink{le__singly_linked_list_8h_a2e1013c24e2c826dbba37a761c5d9f44}{LE\_SLS\_LIST\_INIT};
\end{DoxyCode}


Or


\begin{DoxyCode}
\textcolor{comment}{// Create list.}
\hyperlink{structle__sls___list__t}{le\_sls\_List\_t} MyList;

\textcolor{comment}{// Initialize the list.}
MyList = \hyperlink{le__singly_linked_list_8h_a2e1013c24e2c826dbba37a761c5d9f44}{LE\_SLS\_LIST\_INIT};
\end{DoxyCode}


{\bfseries  The elements of \hyperlink{structle__sls___list__t}{le\+\_\+sls\+\_\+\+List\+\_\+t} M\+U\+S\+T N\+O\+T be accessed directly by the user. }\hypertarget{c_singly_linked_list_sls_createNode}{}\subsection{Creating and Accessing Nodes}\label{c_singly_linked_list_sls_createNode}
Nodes can contain any data in any format and is defined and created by the user. The only requirement for nodes is that it must contain a \hyperlink{structle__sls___link__t}{le\+\_\+sls\+\_\+\+Link\+\_\+t} link member. The link member must be initialized by assigning L\+E\+\_\+\+S\+L\+S\+\_\+\+L\+I\+N\+K\+\_\+\+I\+N\+I\+T to it before it can be used. Nodes can then be added to the list by passing their links to the add functions (\hyperlink{le__singly_linked_list_8h_aca4266a87d4c5e3dca130cd5d48b99af}{le\+\_\+sls\+\_\+\+Stack()}, \hyperlink{le__singly_linked_list_8h_afb5e8ffc3fb5c0d86eb47d0885a6f546}{le\+\_\+sls\+\_\+\+Queue()}, etc.). For example\+:


\begin{DoxyCode}
\textcolor{comment}{// The node may be defined like this.}
\textcolor{keyword}{typedef} \textcolor{keyword}{struct}
\{
     dataType someUserData;
     ...
     \hyperlink{structle__sls___link__t}{le\_sls\_Link\_t} myLink;

\}
MyNodeClass\_t;

\textcolor{comment}{// Create and initialize the list.}
\hyperlink{structle__sls___list__t}{le\_sls\_List\_t} MyList = \hyperlink{le__singly_linked_list_8h_a2e1013c24e2c826dbba37a761c5d9f44}{LE\_SLS\_LIST\_INIT};

\textcolor{keywordtype}{void} foo (\textcolor{keywordtype}{void})
\{
    \textcolor{comment}{// Create the node.  Get the memory from a memory pool previously created.}
    MyNodeClass\_t* myNodePtr = \hyperlink{le__mem_8h_af7c289c73d4182835a26a9099f3db359}{le\_mem\_ForceAlloc}(MyNodePool);

    \textcolor{comment}{// Initialize the node's link.}
    myNodePtr->myLink = \hyperlink{le__singly_linked_list_8h_aa8375976bebc74107b4d026dfcf3e94a}{LE\_SLS\_LINK\_INIT};

    \textcolor{comment}{// Add the node to the head of the list by passing in the node's link.}
    \hyperlink{le__singly_linked_list_8h_aca4266a87d4c5e3dca130cd5d48b99af}{le\_sls\_Stack}(&MyList, &(myNodePtr->myLink));
\}
\end{DoxyCode}


The links in the nodes are added to the list and not the nodes themselves. This allows a node to be simultaneously part of multiple lists simply by having multiple links and adding the links into differently lists. This also means that nodes in a list can be of different types.

Because the links and not the nodes are in the list, the user must have a way to obtain the node itself from the link. This is achieved using the {\ttfamily C\+O\+N\+T\+A\+I\+N\+E\+R\+\_\+\+O\+F} macro defined in \hyperlink{le__basics_8h}{le\+\_\+basics.\+h}. This code sample shows using C\+O\+N\+T\+A\+I\+N\+E\+R\+\_\+\+O\+F to obtain the node\+:


\begin{DoxyCode}
\textcolor{comment}{// Assuming mylist has been created and initialized and is not empty.}
le\_sls\_link\_t* linkPtr = \hyperlink{le__singly_linked_list_8h_a63e301829d4a513c97dbda3943efa791}{le\_sls\_Peek}(&MyList);

\textcolor{comment}{// Now we have the link but we want the node so we can access the user data.}
\textcolor{comment}{// We use CONTAINER\_OF to get a pointer to the node given the node's link.}
\textcolor{keywordflow}{if} (linkPtr != NULL)
\{
    MyNodeClass\_t* myNodePtr = \hyperlink{le__basics_8h_a3616d3fd5b502150b643ddc769f71188}{CONTAINER\_OF}(linkPtr, MyNodeClass\_t, myLink);
\}
\end{DoxyCode}


The user is responsible for creating and freeing memory for all nodes, the linked list module simply manages the links in the nodes. The node must first be removed from all lists before its memory is freed.

{\bfseries The elements of \hyperlink{structle__sls___link__t}{le\+\_\+sls\+\_\+\+Link\+\_\+t} M\+U\+S\+T N\+O\+T be accessed directly by the user.}\hypertarget{c_singly_linked_list_sls_add}{}\subsection{Adding Links to a List}\label{c_singly_linked_list_sls_add}
To add nodes to a list pass the node\textquotesingle{}s link to one of the following functions\+:


\begin{DoxyItemize}
\item \hyperlink{le__singly_linked_list_8h_aca4266a87d4c5e3dca130cd5d48b99af}{le\+\_\+sls\+\_\+\+Stack()} -\/ Adds the link to the head of the list.
\item \hyperlink{le__singly_linked_list_8h_afb5e8ffc3fb5c0d86eb47d0885a6f546}{le\+\_\+sls\+\_\+\+Queue()} -\/ Adds the link to the tail of the list.
\item \hyperlink{le__singly_linked_list_8h_a69f5c789a64d372bb61b173c4418b14b}{le\+\_\+sls\+\_\+\+Add\+After()} -\/ Adds the link to a list after another specified link.
\end{DoxyItemize}\hypertarget{c_singly_linked_list_sls_remove}{}\subsection{Removing Links from a List}\label{c_singly_linked_list_sls_remove}
To remove nodes from a list, use {\ttfamily \hyperlink{le__singly_linked_list_8h_a85f5132147870e260b3b142665ec587e}{le\+\_\+sls\+\_\+\+Pop()}} to remove and return the link at the head of the list.\hypertarget{c_singly_linked_list_sls_peek}{}\subsection{Accessing Links in a List}\label{c_singly_linked_list_sls_peek}
To access a link in a list without removing the link, use one of the following functions\+:


\begin{DoxyItemize}
\item {\ttfamily \hyperlink{le__singly_linked_list_8h_a63e301829d4a513c97dbda3943efa791}{le\+\_\+sls\+\_\+\+Peek()}} -\/ Returns the link at the head of the list without removing it.
\item {\ttfamily \hyperlink{le__singly_linked_list_8h_aed118352d62f4df6301040bc1cd26431}{le\+\_\+sls\+\_\+\+Peek\+Next()}} -\/ Returns the link next to a specified link without removing it.
\end{DoxyItemize}\hypertarget{c_singly_linked_list_sls_query}{}\subsection{Querying List Status}\label{c_singly_linked_list_sls_query}
The following functions can be used to query a list\textquotesingle{}s current status\+:


\begin{DoxyItemize}
\item \hyperlink{le__singly_linked_list_8h_a8f59b1a42d967d018e301f7b7f2d55ae}{le\+\_\+sls\+\_\+\+Is\+Empty()} -\/ Checks if a given list is empty or not.
\item \hyperlink{le__singly_linked_list_8h_a253a443ee79c169462f64551b40f2dd7}{le\+\_\+sls\+\_\+\+Is\+In\+List()} -\/ Checks if a specified link is in the list.
\item \hyperlink{le__singly_linked_list_8h_a33feea925d3a247531c4796aba93f6fa}{le\+\_\+sls\+\_\+\+Num\+Links()} -\/ Checks the number of links currently in the list.
\item \hyperlink{le__singly_linked_list_8h_a871361a091418784f524090d5515a98f}{le\+\_\+sls\+\_\+\+Is\+List\+Corrupted()} -\/ Checks if the list is corrupted.
\end{DoxyItemize}\hypertarget{c_singly_linked_list_sls_fifo}{}\subsection{Queues and Stacks}\label{c_singly_linked_list_sls_fifo}
This implementation of linked lists can easily be used as either queues or stacks.

To use the list as a queue, restrict additions to the list to {\ttfamily \hyperlink{le__singly_linked_list_8h_afb5e8ffc3fb5c0d86eb47d0885a6f546}{le\+\_\+sls\+\_\+\+Queue()}} and removals from the list to {\ttfamily \hyperlink{le__singly_linked_list_8h_a85f5132147870e260b3b142665ec587e}{le\+\_\+sls\+\_\+\+Pop()}}.

To use the list as a stack, restrict additions to the list to {\ttfamily \hyperlink{le__singly_linked_list_8h_aca4266a87d4c5e3dca130cd5d48b99af}{le\+\_\+sls\+\_\+\+Stack()}} and removals from the list to {\ttfamily \hyperlink{le__singly_linked_list_8h_a85f5132147870e260b3b142665ec587e}{le\+\_\+sls\+\_\+\+Pop()}}.\hypertarget{c_singly_linked_list_sls_synch}{}\subsection{Thread Safety and Re-\/\+Entrancy}\label{c_singly_linked_list_sls_synch}
All linked list function calls are re-\/entrant and thread safe. But if the nodes and/or list object is shared by multiple threads, then explicit steps must be taken to maintain mutual exclusion of access.





Copyright (C) Sierra Wireless Inc. Use of this work is subject to license. \hypertarget{c_threading}{}\section{Thread Control A\+P\+I}\label{c_threading}
\hyperlink{le__thread_8h}{A\+P\+I Reference}





Generally, using single-\/threaded, event-\/driven programming (registering callbacks to be called by an event handling loop running in a single thread) is more efficient than using multiple threads. With single-\/threaded, event driven designs\+:
\begin{DoxyItemize}
\item there\textquotesingle{}s no C\+P\+U time spent switching between threads.
\item there\textquotesingle{}s only one copy of thread-\/specific memory objects, like the procedure call stack.
\item there\textquotesingle{}s no need to use thread synchronization mechanisms, like mutexes, to prevent race conditions between threads.
\end{DoxyItemize}

Sometimes, this style doesn\textquotesingle{}t fit well with a problem being solved, so you\textquotesingle{}re forced to implement workarounds that severely complicate the software design. In these cases, it is far better to take advantage of multi-\/threading to simplify the design, even if it means that the program uses more memory or more C\+P\+U cycles. In some cases, the workarounds required to avoid multi-\/threading will cost more memory and/or C\+P\+U cycles than using multi-\/threading would.

But you must {\bfseries  be careful } with multi-\/threading. Some of the most tenacious, intermittent defects known to humankind have resulted from the misuse of multi-\/threading. Ensure you know what you are doing.\hypertarget{c_threading_threadCreating}{}\subsection{Creating a Thread}\label{c_threading_threadCreating}
To create a thread, call {\ttfamily \hyperlink{le__thread_8h_a87e02a46f92e9e3e11ed28a2b265872f}{le\+\_\+thread\+\_\+\+Create()}}.

All threads are {\bfseries named} for two reasons\+:
\begin{DoxyEnumerate}
\item To make it possible to address them by name.
\item For diagnostics.
\end{DoxyEnumerate}

Threads are created in a suspended state. In this state, attributes like scheduling priority and stack size can use the appropriate \char`\"{}\+Set\char`\"{} functions. All attributes have default values so it is not necessary to set any attributes (other than the name and main function address, which are passed into \hyperlink{le__thread_8h_a87e02a46f92e9e3e11ed28a2b265872f}{le\+\_\+thread\+\_\+\+Create()} ). When all attributes have been set, the thread can be started by calling \hyperlink{le__thread_8h_a38df3877ee5ab9fac17b2fc0be46c27e}{le\+\_\+thread\+\_\+\+Start()}.

\begin{DoxyWarning}{Warning}
It is assumed that if a thread {\itshape T1} creates another thread {\itshape T2} then {\bfseries only} thread {\itshape T1} will set the attributes and start thread {\itshape T2}. No other thread should try to set any attributes of {\itshape T2} or try to start it.
\end{DoxyWarning}
\hypertarget{c_threading_threadTerminating}{}\subsection{Terminating a Thread}\label{c_threading_threadTerminating}
Threads can terminate themselves by\+:
\begin{DoxyItemize}
\item returning from their main function
\item calling \hyperlink{le__thread_8h_a6b8e349107ae6628ed8807588f044faa}{le\+\_\+thread\+\_\+\+Exit()}.
\end{DoxyItemize}

Threads can also tell other threads to terminate by \char`\"{}canceling\char`\"{} them; done through a call to {\ttfamily \hyperlink{le__thread_8h_a0f1c1b98f354a96e6e31e55a71b58f6a}{le\+\_\+thread\+\_\+\+Cancel()}}.

If a thread terminates itself, and it is \char`\"{}joinable\char`\"{}, it can pass a {\ttfamily void$\ast$} value to another thread that \char`\"{}joins\char`\"{} with it. See \hyperlink{c_threading_threadJoining}{Joining} for more information.

Canceling a thread may not cause the thread to terminate immediately. If it is in the middle of doing something that can\textquotesingle{}t be interrupted, it will not terminate until it is finished. See \textquotesingle{}man 7 pthreads\textquotesingle{} for more information on cancellation and cancellation points.\hypertarget{c_threading_threadJoining}{}\subsection{Joining}\label{c_threading_threadJoining}
Sometimes, you want single execution thread split (fork) into separate threads of parallel execution and later join back together into one thread later. Forking is done by creating and starting a thread. Joining is done by a call to \hyperlink{le__thread_8h_adf7f24fec4859ca12a52b16ce43fd9b8}{le\+\_\+thread\+\_\+\+Join()}. le\+\_\+thread\+\_\+\+Join(\+T) blocks the calling thread until thread T exits.

For a thread to be joinable, it must have its \char`\"{}joinable\char`\"{} attribute set (using \hyperlink{le__thread_8h_a8959f09f66f365916a6a4fbdaf36cf65}{le\+\_\+thread\+\_\+\+Set\+Joinable()}) prior to being started. Normally, when a thread terminates, it disappears. But, a joinable thread doesn\textquotesingle{}t disappear until another thread \char`\"{}joins\char`\"{} with it. This also means that if a thread is joinable, someone must join with it, or its resources will never get cleaned up (until the process terminates).

\hyperlink{le__thread_8h_adf7f24fec4859ca12a52b16ce43fd9b8}{le\+\_\+thread\+\_\+\+Join()} fetches the return/exit value of the thread that it joined with.\hypertarget{c_threading_threadLocalData}{}\subsection{Thread-\/\+Local Data}\label{c_threading_threadLocalData}
Often, you want data specific to a particular thread. A classic example of is the A\+N\+S\+I C variable {\ttfamily errno}. If one instance of {\ttfamily errno} was shared by all the threads in the process, then it would essentially become useless in a multi-\/threaded program because it would be impossible to ensure another thread hadn\textquotesingle{}t killed {\ttfamily errno} before its value could be read. As a result, P\+O\+S\+I\+X has mandated that {\ttfamily errno} be a {\itshape thread-\/local} variable; each thread has its own unique copy of {\ttfamily errno}.

If a component needs to make use of other thread-\/local data, it can do so using the pthread functions pthread\+\_\+key\+\_\+create(), pthread\+\_\+getspecific(), pthread\+\_\+setspecific(), pthread\+\_\+key\+\_\+delete(). See the pthread man pages for more details.\hypertarget{c_threading_threadSynchronization}{}\subsection{Thread Synchronization}\label{c_threading_threadSynchronization}
Nasty multi-\/threading defects arise as a result of thread synchronization, or a lack of synchronization. If threads share data, they {\bfseries M\+U\+S\+T} be synchronized with each other to avoid destroying that data and incorrect thread behaviour.

\begin{DoxyWarning}{Warning}
This documentation assumes that the reader is familiar with multi-\/thread synchronization techniques and mechanisms.
\end{DoxyWarning}
The Legato C A\+P\+Is provide the following thread synchronization mechanisms\+:
\begin{DoxyItemize}
\item \hyperlink{c_mutex}{Mutex A\+P\+I}
\item \hyperlink{c_semaphore}{Semaphore A\+P\+I}
\item \hyperlink{c_messaging}{Low-\/\+Level Messaging A\+P\+I}
\end{DoxyItemize}\hypertarget{c_threading_threadDestructors}{}\subsection{Thread Destructors}\label{c_threading_threadDestructors}
When a thread dies, some clean-\/up action is needed (e.\+g., a connection needs to be closed or some objects need to be released). If a thread doesn\textquotesingle{}t always terminate the same way (e.\+g., if it might be canceled by another thread or exit in several places due to error detection code), then a clean-\/up function (destructor) is probably needed.

Legato threads use {\ttfamily \hyperlink{le__thread_8h_aa85ee32cc06f219f3c619104f4d97932}{le\+\_\+thread\+\_\+\+Add\+Destructor()}} for clean-\/up functions. It registers a function to be called by a specified thread just before it terminates. A parent thread can also call {\ttfamily \hyperlink{le__thread_8h_a671dbe2927a3b2a13c5150476398f34f}{le\+\_\+thread\+\_\+\+Add\+Child\+Destructor()}} to register a destructor for a child thread before it starts the child thread.

Multiple destructors can be registered for the same thread. They will be called in reverse order of registration (i.\+e, the last destructor to be registered will be called first).\hypertarget{c_threading_threadLegatoizing}{}\subsection{Using Legato A\+P\+Is from Non-\/\+Legato Threads}\label{c_threading_threadLegatoizing}
If a thread is started using some other means besides \hyperlink{le__thread_8h_a38df3877ee5ab9fac17b2fc0be46c27e}{le\+\_\+thread\+\_\+\+Start()} (e.\+g., if pthread\+\_\+create() is used directly), then the Legato thread-\/specific data will not have been initialized for that thread. Therefore, if that thread tries to call some Legato A\+P\+Is, a fatal error message like, \char`\"{}\+Legato threading A\+P\+I used in non-\/\+Legato thread!\char`\"{} may be seen.

To work around this, a \char`\"{}non-\/\+Legato thread\char`\"{} can call \hyperlink{le__thread_8h_a3e35d530ce76e97a627dc60100fc1475}{le\+\_\+thread\+\_\+\+Init\+Legato\+Thread\+Data()} to initialize the thread-\/specific data that the Legato framework needs.

If you have done this for a thread, and that thread will die before the process it is inside dies, then that thread must call \hyperlink{le__thread_8h_aef59d0ded85da6ddd169a661824670d0}{le\+\_\+thread\+\_\+\+Cleanup\+Legato\+Thread\+Data()} before it exits. Otherwise the process will leak memory. Furthermore, if the thread will ever be cancelled by another thread before the process dies, a cancellation clean-\/up handler can be used to ensure that the clean-\/up is done, if the thread\textquotesingle{}s cancellation type is set to \char`\"{}deferred\char`\"{}. See \textquotesingle{}man 7 pthreads\textquotesingle{} for more information on cancellation and cancellation points.





Copyright (C) Sierra Wireless Inc. Use of this work is subject to license. \hypertarget{c_timer}{}\section{Timer A\+P\+I}\label{c_timer}
\hyperlink{le__timer_8h}{A\+P\+I Reference}





This module provides an A\+P\+I for managing and using timers.

\begin{DoxyNote}{Note}
This is an initial version of the A\+P\+I that only provides support for relative timers (e.\+g., expires in 10 seconds). Absolute timers allow a specific time/date to be used, and will be supported in a future version of this A\+P\+I.
\end{DoxyNote}
\hypertarget{c_timer_timer_objects}{}\subsection{Creating/\+Deleting Timer Objects}\label{c_timer_timer_objects}
Timers are created using \hyperlink{le__timer_8h_aee41169a210378b369f440cf99146522}{le\+\_\+timer\+\_\+\+Create}. The timer name is used for logging purposes only.

The following attributes of the timer can be set\+:
\begin{DoxyItemize}
\item \hyperlink{le__timer_8h_abbf8d4c3c78d7bf5801b94071adcb6c6}{le\+\_\+timer\+\_\+\+Set\+Handler}
\item \hyperlink{le__timer_8h_a0a103d5cef5e83fc9088859d527bbd43}{le\+\_\+timer\+\_\+\+Set\+Interval}
\item \hyperlink{le__timer_8h_a292b0a7d6dc0796a36a54fd04c6a7eeb}{le\+\_\+timer\+\_\+\+Set\+Repeat}
\item \hyperlink{le__timer_8h_af6900bdb4653ff95f7f7be918b9e482d}{le\+\_\+timer\+\_\+\+Set\+Context\+Ptr}
\end{DoxyItemize}

The repeat count defaults to 1, so that the timer is initially a one-\/shot timer. All the other attributes must be explicitly set. At a minimum, the interval must be set before the timer can be used. Note that these attributes can only be set if the timer is not currently running; otherwise, an error will be returned.

A timer is deleted using \hyperlink{le__timer_8h_ae103f6736bf855e77e5e59bbad1e27a7}{le\+\_\+timer\+\_\+\+Delete}. If the timer is currently running, then it will be stopped first, before being deleted.\hypertarget{c_timer_timer_usage}{}\subsection{Using Timers}\label{c_timer_timer_usage}
A timer can be started using \hyperlink{le__timer_8h_ada2ce7f8cb1e76ed959e323ae94bbfc0}{le\+\_\+timer\+\_\+\+Start}. If it\textquotesingle{}s already running, then it won\textquotesingle{}t be modified; instead an error will be returned. To restart a currently running timer, use \hyperlink{le__timer_8h_ab6b83d6302095a46b6046160c0a479bb}{le\+\_\+timer\+\_\+\+Restart}.

A timer can be stopped using \hyperlink{le__timer_8h_af310daa378bd6ca39373a47e073f2243}{le\+\_\+timer\+\_\+\+Stop}. If it\textquotesingle{}s not currently running, an error will be returned, and nothing more will be done.

To determine if the timer is currently running, use \hyperlink{le__timer_8h_ab33b8568fd394d38274b778130111f70}{le\+\_\+timer\+\_\+\+Is\+Running}.

When a timer expires, if the timer expiry handler is set by \hyperlink{le__timer_8h_abbf8d4c3c78d7bf5801b94071adcb6c6}{le\+\_\+timer\+\_\+\+Set\+Handler}, the handler will be called with a reference to the expired timer. If additional data is required in the handler, \hyperlink{le__timer_8h_af6900bdb4653ff95f7f7be918b9e482d}{le\+\_\+timer\+\_\+\+Set\+Context\+Ptr} can be used to set the appropriate context before starting the timer, and \hyperlink{le__timer_8h_aa0432dbabb32b546c0c0e6ced5ba9d3d}{le\+\_\+timer\+\_\+\+Get\+Context\+Ptr} can be used to retrieve the context while in the handler.

The number of times that a timer has expired can be retrieved by \hyperlink{le__timer_8h_a554cff1d11525bb60115291248f3ff53}{le\+\_\+timer\+\_\+\+Get\+Expiry\+Count}. This count is independent of whether there is an expiry handler for the timer.\hypertarget{c_timer_le_timer_thread}{}\subsection{Thread Support}\label{c_timer_le_timer_thread}
A timer should only be used by the thread that created it. It\textquotesingle{}s not safe for a thread to use or manipulate a timer that belongs to another thread. The timer expiry handler is called by the event loop of the thread that starts the timer.

See \hyperlink{c_eventLoop}{Event Loop A\+P\+I} for details on running the event loop of a thread.\hypertarget{c_timer_timer_errors}{}\subsection{Fatal Errors}\label{c_timer_timer_errors}
The process will exit under any of the following conditions\+:
\begin{DoxyItemize}
\item If an invalid timer object is given to\+:
\begin{DoxyItemize}
\item \hyperlink{le__timer_8h_ae103f6736bf855e77e5e59bbad1e27a7}{le\+\_\+timer\+\_\+\+Delete}
\item \hyperlink{le__timer_8h_abbf8d4c3c78d7bf5801b94071adcb6c6}{le\+\_\+timer\+\_\+\+Set\+Handler}
\item \hyperlink{le__timer_8h_a0a103d5cef5e83fc9088859d527bbd43}{le\+\_\+timer\+\_\+\+Set\+Interval}
\item \hyperlink{le__timer_8h_a292b0a7d6dc0796a36a54fd04c6a7eeb}{le\+\_\+timer\+\_\+\+Set\+Repeat}
\item \hyperlink{le__timer_8h_ada2ce7f8cb1e76ed959e323ae94bbfc0}{le\+\_\+timer\+\_\+\+Start}
\item \hyperlink{le__timer_8h_af310daa378bd6ca39373a47e073f2243}{le\+\_\+timer\+\_\+\+Stop}
\item \hyperlink{le__timer_8h_ab6b83d6302095a46b6046160c0a479bb}{le\+\_\+timer\+\_\+\+Restart}
\item \hyperlink{le__timer_8h_af6900bdb4653ff95f7f7be918b9e482d}{le\+\_\+timer\+\_\+\+Set\+Context\+Ptr}
\item \hyperlink{le__timer_8h_aa0432dbabb32b546c0c0e6ced5ba9d3d}{le\+\_\+timer\+\_\+\+Get\+Context\+Ptr}
\item \hyperlink{le__timer_8h_a554cff1d11525bb60115291248f3ff53}{le\+\_\+timer\+\_\+\+Get\+Expiry\+Count}
\end{DoxyItemize}
\end{DoxyItemize}\hypertarget{c_timer_timer_troubleshooting}{}\subsection{Troubleshooting}\label{c_timer_timer_troubleshooting}
Timers can be traced by enabling the log trace keyword \char`\"{}timers\char`\"{} in the \char`\"{}framework\char`\"{} component.

See \hyperlink{c_logging_c_log_controlling}{Log Controls} for more information.





Copyright (C) Sierra Wireless Inc. Use of this work is subject to license. \hypertarget{c_test}{}\section{Unit Testing A\+P\+I}\label{c_test}
\hyperlink{le__test_8h}{A\+P\+I Reference}





Unit testing is an important aspect of a quantifiable quality assurance methodology. Although unit testing requires some extra overhead (the writing of the unit tests) during the development process it can provide enormous benefits during the project life cycle.

One benefit of writing unit tests is that it gets the developer using the interface to the unit they designed. This forces the developer to think about, and hopefully design for, usability of the interface early in the development cycle.

Another major benefit to unit testing is that it provides a documented and verifiable level of correctness for the designed unit. This allows the developer to refactor the code more aggressively and to quickly verify its correctness. Unit tests can also be used to perform regression testing when adding new features.

Despite the benefits of unit testing, unit tests are often omitted because of the initial overhead of writing the tests and the complexity of testing frameworks. Legato\textquotesingle{}s Unit Test Framework is simple to use and very flexible and lightweight consisting of some handy macros.\hypertarget{c_test_c_test_modes}{}\subsection{Modes of Operation}\label{c_test_c_test_modes}
The Legato Test Framework can run in one of two modes\+: {\bfseries pass through} or {\bfseries exit on failure}. In {\bfseries pass through} mode all tests are run even if some of the tests fail. Failed tests are counted and reported on completion. {\bfseries Exit on failure} mode also runs all tests but exits right away if any tests fail.

Mode selection is done through a command line argument. If the test program is run with the command line argument \textquotesingle{}-\/p\textquotesingle{} or \textquotesingle{}--pass-\/through\textquotesingle{} then {\bfseries pass through} mode is selected. If the neither the \textquotesingle{}-\/p\textquotesingle{} or \textquotesingle{}--pass-\/through\textquotesingle{} arguments are present then {\bfseries exit on failure} mode is selected.\hypertarget{c_test_c_test_setup}{}\subsection{Setting Up the Test Framework}\label{c_test_c_test_setup}
To setup the Legato Test Framework, call the {\ttfamily L\+E\+\_\+\+T\+E\+S\+T\+\_\+\+I\+N\+I\+T} macro, once before any tests are started.\hypertarget{c_test_c_test_testing}{}\subsection{Performing Tests}\label{c_test_c_test_testing}
To perform tests, call the {\ttfamily L\+E\+\_\+\+T\+E\+S\+T} macro and pass in your test function to L\+E\+\_\+\+T\+E\+S\+T as a parameter. The test function must have a bool return type which indicates that the test passed (true) or failed (false).

For example\+:


\begin{DoxyCode}
\textcolor{preprocessor}{#include "\hyperlink{legato_8h}{legato.h}"}

\textcolor{comment}{// Returns true if the test passes, otherwise returns false.}
\textcolor{keywordtype}{bool} Test1(\textcolor{keywordtype}{void})
\{
    \textcolor{keywordtype}{int} expectedValue;

    \textcolor{comment}{// Do some initializations and/or calculations.}
    ...

    \textcolor{comment}{// Call one of the unit-under-test's interface function and check it's return value against}
    \textcolor{comment}{// an expected value that was calculated earlier.}
    \textcolor{keywordflow}{return} (unitUnderTest\_foo() == expectedValue);
\}


\textcolor{comment}{// Returns true if the test passes, otherwise returns false.}
\textcolor{keywordtype}{bool} Test2(\textcolor{keywordtype}{void})
\{
    \textcolor{keywordtype}{int} expectedValue;

    \textcolor{comment}{// Do some initializations and/or calculations.}
    ...

    \textcolor{comment}{// Call one of the unit-under-test's interface function and check it's return value against}
    \textcolor{comment}{// an expected value that was calculated earlier.}
    \textcolor{keywordflow}{return} (unitUnderTest\_foo2() == expectedValue);
\}


\textcolor{keywordtype}{int} main (\textcolor{keywordtype}{void})
\{
    \textcolor{comment}{// Setup the Legato Test Framework.}
    \hyperlink{le__test_8h_a219d49d27a2a5a1a6dc63f31b0b66c23}{LE\_TEST\_INIT};

    \textcolor{comment}{// Run the tests.}
    \hyperlink{le__test_8h_a555300c0c5b3442572fbcafa6dd7994b}{LE\_TEST}(Test1());
    \hyperlink{le__test_8h_a555300c0c5b3442572fbcafa6dd7994b}{LE\_TEST}(Test2());

    \textcolor{comment}{// Exit with the number of failed tests as the exit code.}
    \hyperlink{le__test_8h_a5f0517641049c368d0a55658c4f4ddec}{LE\_TEST\_EXIT};
\}
\end{DoxyCode}
\hypertarget{c_test_c_test_exit}{}\subsection{Exiting a Test Program}\label{c_test_c_test_exit}
When a test program is finished executing tests and needs to exit, it should always exit using the L\+E\+\_\+\+T\+E\+S\+T\+\_\+\+E\+X\+I\+T macro.

It\textquotesingle{}s also okay to exit using \hyperlink{le__log_8h_a54b4b07f5396e19a8d9fca74238f4795}{L\+E\+\_\+\+F\+A\+T\+A\+L()}, \hyperlink{le__log_8h_a7a3e66a87026cc9e57bcb748840ab41b}{L\+E\+\_\+\+F\+A\+T\+A\+L\+\_\+\+I\+F()} or \hyperlink{le__log_8h_ac0dbbef91dc0fed449d0092ff0557b39}{L\+E\+\_\+\+A\+S\+S\+E\+R\+T()}, if the test must be halted immeditately due to some failure that cannot be recovered from.\hypertarget{c_test_c_test_result}{}\subsection{Test Results}\label{c_test_c_test_result}
The L\+E\+\_\+\+T\+E\+S\+T\+\_\+\+E\+X\+I\+T macro will cause the process to exit with the number of failed tests as the exit code.

Also, L\+E\+\_\+\+T\+E\+S\+T will log an error message if the test fails and will log an info message if the test passes.

If the unit test in the example above was run in \char`\"{}pass-\/through mode\char`\"{} (continue even when a test fails) and Test1 failed and Test2 passed, the logs will contain the messages\+:

\begin{DoxyVerb} =ERR= | Unit Test Failed: 'Test1()'
  INFO | Unit Test Passed: 'Test2()'
\end{DoxyVerb}


And the return code would be 1.

\begin{DoxyNote}{Note}
The log message format depends on the current log settings.
\end{DoxyNote}
\hypertarget{c_test_c_test_multiThread}{}\subsection{Multi-\/\+Threaded Tests}\label{c_test_c_test_multiThread}
For unit tests that contain multiple threads run the various tests, the normal testing procedure will work because all the macros in this test framework are thread safe.\hypertarget{c_test_c_test_multiProcess}{}\subsection{Multi-\/\+Process Tests}\label{c_test_c_test_multiProcess}
For unit tests that require the use of multiple concurrent processes, a single process can fork the other processes using \hyperlink{le__test_8h_ac0c538ca3dfcd072464385b2f7e2776f}{L\+E\+\_\+\+T\+E\+S\+T\+\_\+\+F\+O\+R\+K()} and then wait for them to terminate using \hyperlink{le__test_8h_ad6045dfae96da7c326024c784de3d207}{L\+E\+\_\+\+T\+E\+S\+T\+\_\+\+J\+O\+I\+N()}.

When a child that is being waited for terminates, \hyperlink{le__test_8h_ad6045dfae96da7c326024c784de3d207}{L\+E\+\_\+\+T\+E\+S\+T\+\_\+\+J\+O\+I\+N()} will look at the child\textquotesingle{}s termination status and add the results to the running test summary.

If the child process exits normally with a non-\/negative exit code, that exit code will be considered a count of the number of test failures that occurred in that child process.

If the child exits normally with a negative exit code or if the child is terminated due to a signal (which can be caused by a segmentation fault, etc.), \hyperlink{le__test_8h_ad6045dfae96da7c326024c784de3d207}{L\+E\+\_\+\+T\+E\+S\+T\+\_\+\+J\+O\+I\+N()} will count one test failure for that child process.


\begin{DoxyCode}
\hyperlink{le__event_loop_8h_abdb9187a56836a93d19cc793cbd4b7ec}{COMPONENT\_INIT}
\{
    \textcolor{comment}{// Setup the Legato Test Framework.}
    \hyperlink{le__test_8h_a219d49d27a2a5a1a6dc63f31b0b66c23}{LE\_TEST\_INIT};

    \textcolor{comment}{// Run the test programs.}
    \hyperlink{le__test_8h_ab2de7b7c51414646bad2bc4ec2dda444}{le\_test\_ChildRef\_t} test1 = \hyperlink{le__test_8h_ac0c538ca3dfcd072464385b2f7e2776f}{LE\_TEST\_FORK}(\textcolor{stringliteral}{"test1"});
    \hyperlink{le__test_8h_ab2de7b7c51414646bad2bc4ec2dda444}{le\_test\_ChildRef\_t} test2 = \hyperlink{le__test_8h_ac0c538ca3dfcd072464385b2f7e2776f}{LE\_TEST\_FORK}(\textcolor{stringliteral}{"test2"});

    \textcolor{comment}{// Wait for the test programs to finish and tally the results.}
    \hyperlink{le__test_8h_ad6045dfae96da7c326024c784de3d207}{LE\_TEST\_JOIN}(test1);
    \hyperlink{le__test_8h_ad6045dfae96da7c326024c784de3d207}{LE\_TEST\_JOIN}(test2);

    \textcolor{comment}{// Exit with the number of failed tests as the exit code.}
    \hyperlink{le__test_8h_a5f0517641049c368d0a55658c4f4ddec}{LE\_TEST\_EXIT};
\}
\end{DoxyCode}






Copyright (C) Sierra Wireless Inc. Use of this work is subject to license. \hypertarget{c_utf8}{}\section{U\+T\+F-\/8 String Handling A\+P\+I}\label{c_utf8}
\hyperlink{le__utf8_8h}{A\+P\+I Reference}





This module implements safe and easy to use string handling functions for null-\/terminated strings with U\+T\+F-\/8 encoding.

U\+T\+F-\/8 is a variable length character encoding that supports every character in the Unicode character set. U\+T\+F-\/8 has become the dominant character encoding because it is self synchronizing, compatible with A\+S\+C\+I\+I, and avoids the endian issues that other encodings face.\hypertarget{c_utf8_utf8_encoding}{}\subsection{U\+T\+F-\/8 Encoding}\label{c_utf8_utf8_encoding}
U\+T\+F-\/8 uses between one and four bytes to encode a character as illustrated in the following table.

\begin{TabularC}{4}
\hline
\rowcolor{lightgray}{\bf Byte 1  }&{\bf Byte 2  }&{\bf Byte 3  }&{\bf Byte 4   }\\\cline{1-4}
0xxxxxxx  &&&\\\cline{1-4}
110xxxxx  &10xxxxxx  &&\\\cline{1-4}
1110xxxx  &10xxxxxx  &10xxxxxx  &\\\cline{1-4}
11110xxx  &10xxxxxx  &10xxxxxx  &10xxxxxx   \\\cline{1-4}
\end{TabularC}


Single byte codes are used only for the A\+S\+C\+I\+I values 0 through 127. In this case, U\+T\+F-\/8 has the same binary value as A\+S\+C\+I\+I, making A\+S\+C\+I\+I text valid U\+T\+F-\/8 encoded Unicode. All A\+S\+C\+I\+I strings are U\+T\+F-\/8 compatible.

Character codes larger than 127 have a multi-\/byte encoding consisting of a leading byte and one or more continuation bytes.

The leading byte has two or more high-\/order 1\textquotesingle{}s followed by a 0 that can be used to determine the number bytes in the character without examining the continuation bytes.

The continuation bytes have \textquotesingle{}10\textquotesingle{} in the high-\/order position.

Single bytes, leading bytes and continuation bytes can\textquotesingle{}t have the same values. This means that U\+T\+F-\/8 strings are self-\/synchronized, allowing the start of a character to be found by backing up at most three bytes.\hypertarget{c_utf8_utf8_copy}{}\subsection{Copy and Append}\label{c_utf8_utf8_copy}
{\ttfamily \hyperlink{le__utf8_8h_aa5ae72c01396c106fdf3b4741ead7477}{le\+\_\+utf8\+\_\+\+Copy()}} copies a string to a specified buffer location.

{\ttfamily \hyperlink{le__utf8_8h_ade7dfb60b18574dc62c49b86c025579b}{le\+\_\+utf8\+\_\+\+Append()}} appends a string to the end of another string by copying the source string to the destination string\textquotesingle{}s buffer starting at the null-\/terminator of the destination string.

The {\ttfamily le\+\_\+uft8\+\_\+\+Copy\+Up\+To\+Sub\+Str()} function is like \hyperlink{le__utf8_8h_aa5ae72c01396c106fdf3b4741ead7477}{le\+\_\+utf8\+\_\+\+Copy()} except it copies only up to, but not including, a specified string.\hypertarget{c_utf8_utf8_trunc}{}\subsection{Truncation}\label{c_utf8_utf8_trunc}
Because U\+T\+F-\/8 is a variable length encoding, the number of characters in a string is not necessarily the same as the number bytes in the string. When using functions like \hyperlink{le__utf8_8h_aa5ae72c01396c106fdf3b4741ead7477}{le\+\_\+utf8\+\_\+\+Copy()} and \hyperlink{le__utf8_8h_ade7dfb60b18574dc62c49b86c025579b}{le\+\_\+utf8\+\_\+\+Append()}, the size of the destination buffer, in bytes, must be provided to avoid buffer overruns.

The copied string is truncated because of limited space in the destination buffer, and the destination buffer may not be completely filled. This can occur during the copy processf the last character to copy is more than one byte long and will not fit within the buffer.

The character is not copied and a null-\/terminator is added. Even though we have not filled the destination buffer,we have truncated the copied string. Essentially, functions like \hyperlink{le__utf8_8h_aa5ae72c01396c106fdf3b4741ead7477}{le\+\_\+utf8\+\_\+\+Copy()} and \hyperlink{le__utf8_8h_ade7dfb60b18574dc62c49b86c025579b}{le\+\_\+utf8\+\_\+\+Append()} only copy complete characters, not partial characters.

For \hyperlink{le__utf8_8h_aa5ae72c01396c106fdf3b4741ead7477}{le\+\_\+utf8\+\_\+\+Copy()}, the number of bytes actually copied is returned in the num\+Bytes\+Ptr parameter. This parameter can be set to N\+U\+L\+L if the number of bytes copied is not needed. \hyperlink{le__utf8_8h_ade7dfb60b18574dc62c49b86c025579b}{le\+\_\+utf8\+\_\+\+Append()} and le\+\_\+utf8\+\_\+\+Copy\+Up\+To\+Ascii\+Char() work similarly.


\begin{DoxyCode}
\textcolor{comment}{// In this code sample, we need the number of bytes actually copied:}
\textcolor{keywordtype}{size\_t} numBytes;

\textcolor{keywordflow}{if} (\hyperlink{le__utf8_8h_aa5ae72c01396c106fdf3b4741ead7477}{le\_utf8\_Copy}(destStr, srcStr, \textcolor{keyword}{sizeof}(destStr), &numBytes) == 
      \hyperlink{le__basics_8h_a1cca095ed6ebab24b57a636382a6c86cae42c9d785827fc3a9c47fb55baca7879}{LE\_OVERFLOW})
\{
    \hyperlink{le__log_8h_a0201b2f60ee0e945479f91e181bf04b6}{LE\_WARN}(\textcolor{stringliteral}{"'%s' was truncated when copied.  Only %d bytes were copied."}, srcStr, numBytes);
\}

\textcolor{comment}{// In this code sample, we don't care about the number of bytes copied:}
\hyperlink{le__log_8h_ac0dbbef91dc0fed449d0092ff0557b39}{LE\_ASSERT}(\hyperlink{le__utf8_8h_aa5ae72c01396c106fdf3b4741ead7477}{le\_utf8\_Copy}(destStr, srcStr, \textcolor{keyword}{sizeof}(destStr), NULL) != 
      \hyperlink{le__basics_8h_a1cca095ed6ebab24b57a636382a6c86cae42c9d785827fc3a9c47fb55baca7879}{LE\_OVERFLOW});
\end{DoxyCode}
\hypertarget{c_utf8_utf8_length}{}\subsection{String Lengths}\label{c_utf8_utf8_length}
String length may mean either the number of characters in the string or the number of bytes in the string. These two meanings are often used interchangeably because in A\+S\+C\+I\+I-\/only encodings the number of characters in a string is equal to the number of bytes in a string. But this is not necessarily true with variable length encodings such as U\+T\+F-\/8. Legato provides both a \hyperlink{le__utf8_8h_af8f61f1aa523b03d02d6a89cb61449e2}{le\+\_\+utf8\+\_\+\+Num\+Chars()} function and a \hyperlink{le__utf8_8h_a2541a26cade8cef93db889194a430008}{le\+\_\+utf8\+\_\+\+Num\+Bytes()} function.

{\ttfamily \hyperlink{le__utf8_8h_a2541a26cade8cef93db889194a430008}{le\+\_\+utf8\+\_\+\+Num\+Bytes()}} must be used when determining the memory size of a string. {\ttfamily \hyperlink{le__utf8_8h_af8f61f1aa523b03d02d6a89cb61449e2}{le\+\_\+utf8\+\_\+\+Num\+Chars()}} is useful for counting the number of characters in a string (ie. for display purposes).\hypertarget{c_utf8_utf8_format}{}\subsection{Checking U\+T\+F-\/8 Format}\label{c_utf8_utf8_format}
As can be seen in the \hyperlink{c_utf8_utf8_encoding}{U\+T\+F-\/8 Encoding} section, U\+T\+F-\/8 strings have a specific byte sequence. The {\ttfamily \hyperlink{le__utf8_8h_acffd959e1c6dcf9841217c1c0f6d09e5}{le\+\_\+utf8\+\_\+\+Is\+Format\+Correct()}} function can be used to check if a string conforms to U\+T\+F-\/8 encoding. Not all valid U\+T\+F-\/8 characters are valid for a given character set; \hyperlink{le__utf8_8h_acffd959e1c6dcf9841217c1c0f6d09e5}{le\+\_\+utf8\+\_\+\+Is\+Format\+Correct()} does not check for this.\hypertarget{c_utf8_utf8_parsing}{}\subsection{String Parsing}\label{c_utf8_utf8_parsing}
To assist with converting integer values from U\+T\+F-\/8 strings to binary numerical values, \hyperlink{le__utf8_8h_a680a92fafea1ed72dedb80b52be32a06}{le\+\_\+utf8\+\_\+\+Parse\+Int()} is provided.

More parsing functions may be added as required in the future.\hypertarget{c_utf8_utf8_monotonic}{}\subsection{Monotonic Strings}\label{c_utf8_utf8_monotonic}
Occasionally, when creating identifiers for a set of objects it is useful to be able to generate a set of mutually unique strings. The identifiers may not have any meanings themselves but what is important is that they uniquely identify the object. The license plate number of cars is a good example of this.

The function \hyperlink{le__utf8_8h_a84c9f386331804e20c756ef386ae03dd}{le\+\_\+utf8\+\_\+\+Get\+Monotonic\+String()} in this module can be used to generate a series of mutually exclusive strings. The strings generated by this function differ from our license plate example in that the generated strings are variable length and are ordered. Nevertheless, the important property of these strings are that they are mutually unique and can be used as identifiers for a set of objects.

Passing an empty string to the \hyperlink{le__utf8_8h_a84c9f386331804e20c756ef386ae03dd}{le\+\_\+utf8\+\_\+\+Get\+Monotonic\+String()} function will generate the first string in the series. Passing the first string back into \hyperlink{le__utf8_8h_a84c9f386331804e20c756ef386ae03dd}{le\+\_\+utf8\+\_\+\+Get\+Monotonic\+String()} will generate the next string in the series. Continuing to pass the previously generated string to \hyperlink{le__utf8_8h_a84c9f386331804e20c756ef386ae03dd}{le\+\_\+utf8\+\_\+\+Get\+Monotonic\+String()} will produce a series of unique strings.

For example, the following function creates a number of files with unique names.


\begin{DoxyCode}
\textcolor{keyword}{static} \textcolor{keywordtype}{void} CreateFiles(\textcolor{keywordtype}{size\_t} numOfFiles)
\{
    \textcolor{keywordtype}{char} fileName[100] = \textcolor{stringliteral}{""};
    \textcolor{keywordtype}{char} prevFileName[100] = \textcolor{stringliteral}{""};

    \textcolor{keywordtype}{int} i;
    \textcolor{keywordflow}{for} (i = 0; i < numOfFiles; i++)
    \{
        \textcolor{comment}{// Generate the fileName.}
        \hyperlink{le__log_8h_ac0dbbef91dc0fed449d0092ff0557b39}{LE\_ASSERT}(\hyperlink{le__utf8_8h_a84c9f386331804e20c756ef386ae03dd}{le\_utf8\_GetMonotonicString}(prevFileName, fileName, \textcolor{keyword}{
      sizeof}(fileName)) == \hyperlink{le__basics_8h_a1cca095ed6ebab24b57a636382a6c86ca5066a4bcec691c6b67843b8f79656422}{LE\_OK});

        \textcolor{comment}{// Create the file.}
        \textcolor{keywordtype}{int} fd;
        \textcolor{keywordflow}{do}
        \{
            fd = open(fileName, O\_RDWR | O\_CREAT, S\_IRWXU);
        \}
        \textcolor{keywordflow}{while} ( (fd == -1) && (errno == EINTR) );

        \hyperlink{le__log_8h_a7a3e66a87026cc9e57bcb748840ab41b}{LE\_FATAL\_IF}(fd == -1, \textcolor{stringliteral}{"Could not create file %s.  %m."}, fileName);

        \textcolor{comment}{// Save the file name to generate the next file name.}
        \hyperlink{le__log_8h_ac0dbbef91dc0fed449d0092ff0557b39}{LE\_ASSERT}(\hyperlink{le__utf8_8h_aa5ae72c01396c106fdf3b4741ead7477}{le\_utf8\_Copy}(prevFileName, fileName, \textcolor{keyword}{sizeof}(prevFileName), NULL) == 
      \hyperlink{le__basics_8h_a1cca095ed6ebab24b57a636382a6c86ca5066a4bcec691c6b67843b8f79656422}{LE\_OK});

        \textcolor{comment}{// Close the file.}
        fd\_Close(fd);
    \}
\}
\end{DoxyCode}






Copyright (C) Sierra Wireless Inc. Use of this work is subject to license. 